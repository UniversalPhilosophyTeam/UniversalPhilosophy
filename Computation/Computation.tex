\documentclass[11pt]{article}
\usepackage{amsfonts}
\usepackage[T1]{fontenc}
\usepackage{mathabx,graphicx}

\newcommand{\test}{\circlearrowright}
\def \loop {\ensuremath{\rotatebox[origin=c]{-90}{$\circlearrowright$}}}
\def \nestedloop {\ensuremath{\rotatebox[origin=c]{-90}{$\circlearrowright$}}^n}

\begin{document}

\section*{Computation}





% Definition of Program
\section{Programs}
\subsection{Logical Instructions}
Define $\mathcal{L}$; an ordered set of logical operations $s_i$
\begin{center}
$
\mathcal{L} := \{ s_1,s_2,...,s_{N}\}
$
\end{center}





\subsection{Memory}
Define Memory $\mathcal{M}$; a set of elements, magnitudes, or sets $b_i$
\begin{center}
$\mathcal{M} := \{b_1,b_2,...,b_M\}$
\end{center}




\subsection{State |}
Define state; the memory utilized to perform program P
\begin{center}
$
P := \{ s_1, s_2,...,s_{N} | b_1, b_2,...,b_M\} =
$
\\ \vspace{2mm}
$
\{ s_1, s_2,...,s_{N}, b_1, b_2,...,b_M\}
$
\end{center}





\subsection{Boolean Programs}
Define a boolean program; boolean programs can represent functions with inputs $x_i$, input set C, and boolean output $y_o$
\begin{center}
$
X = \{x_1,...,x_n,C\}; \hspace{2mm} C = \{u_1,u_2,...,u_c\}
$
\\ \vspace{2mm}
$P = P\lbrack X \rbrack := \{ s_1,s_2,...,s_{N}\hspace{1mm}|\hspace{1mm} b_1, b_2,...,b_M,y_o\}=$
\\ \vspace{2mm}
$
P\lbrack X \rbrack \rightarrow y_o \in \{ \mathbb{T},\mathbb{F}\}
$
\end{center}





\subsection{Void Programs}
Define a void program; a program with inputs $x_i$, input set C, and no output
\begin{center}
$
X = \{x_1,...,x_n,C\}
$
\\ \vspace{2mm}
$P = P\lbrack X \rbrack := \{ s_1,s_2,...,s_{N}\hspace{1mm}|\hspace{1mm} b_1, b_2,...,b_M\}$
\end{center}





\subsection{Numerical Programs}
Define a numerical program; a program with inputs $x_i$, input set C, and real, rational output $y_o$
\begin{center}
$
X = \{x_1,...,x_n,C\}
$
\\ \vspace{2mm}
$P = P\lbrack X \rbrack := \{ s_1,s_2,...,s_{N}\hspace{1mm}|\hspace{1mm} b_1, b_2,...,b_M,y_o\}=$
\\ \vspace{2mm}
$
P\lbrack X \rbrack \rightarrow y_o \in \mathbb{Q} \hspace{2mm} y_o \geq 0
$
\end{center}





\subsection{System Programs}
Define a system program; a program with inputs $x_i$, input set C, and real, output set $Y_o$
\begin{center}
$
X = \{x_1,...,x_n,C\}
$
\\ \vspace{2mm}
$P = P\lbrack X \rbrack := \{ s_1,s_2,...,s_{N}\hspace{1mm}|\hspace{1mm} b_1, b_2,...,b_M,Y_o\}=$
\\ \vspace{2mm}
$
P\lbrack X \rbrack \rightarrow Y_o = \{y_1,y_2,...,y_K\}
$
\end{center}





\subsection{Mathematical Programs}
Define a mathematical program; a program with inputs $x_i$, input set C and numerical output $y_o$
\begin{center}
$
X = \{x_1,...,x_n,C\}
$
\\ \vspace{2mm}
$P = P\lbrack X \rbrack := \{ s_1,s_2,...,s_{N}\hspace{1mm}|\hspace{1mm} b_1, b_2,...,b_M,y_o\}=$
\\ \vspace{2mm}
$
P\lbrack X \rbrack \rightarrow y_o \in \mathbb{Q}
$
\end{center}




% No-op; While Loop; For Loop
\newpage
\section{No-op ;}

\subsection{Definition}
\begin{center}
$
; := \emptyset
$
\end{center}

\subsection{Property of No-op}
No-op can be inserted into any set with equality
\begin{center}
$
S= \{s_1,s_2,...,s_N\}
$
\\ \vspace{2mm}
$
S_; = insert \lbrack S,;, i \rbrack
$
\\ \vspace{2mm}
$
S_; = S_1 \hspace{2mm} \forall i
$
\\ \vspace{2mm}
$
|S_;| = |S| \hspace{2mm} \forall i
$
\end{center}


\subsection{Proof}
by definition of magnitude of null = 0 with Set And





%\section{For Loop \loop}
%\loop \lbrack startindex, endindex, condition\rbrack
%\subsection{Definition}








%\section{Nested For Loop $\nestedloop$}
%\loop \lbrack startindex, endindex, condition1,...condition n\rbrack
%\subsection{Definition}








% Decision Problems and Solutions
\newpage
\section{Decision Problems}

\subsection{Definition}
Define decision problem; a function with inputs $x_i$ and boolean output "answer" $a_o$
\begin{center}
$
X_i = \{x_1,...,x_n,C\}
$
\\ \vspace{2mm}
$
D := f \lbrack X_i \rbrack \rightarrow a_{o} \in \{\mathbb{T}, \mathbb{F}\} \hspace{3mm} \forall X_i
$
\end{center}







\section{General Solutions}

\subsection{Definition}
Program P is a general solution $s^{+}$ to decision problem D if \\\\
1. P outputs answer $a_o$ for all inputs $X_i \hspace{2mm} \forall i$\\
and \\
2. $s^{+}[X_i]$ is a subset of $s^{+}[\hat{X}_i]$
\begin{center}
\vspace{2mm}
$
X_i = \{x_1,...,x_n,C\}; \hspace{2mm} \hat{X}_i = \{x_1,...,x_n,x_{n+1},C\}
$
\\ \vspace{2mm}
$
D := f \lbrack X_i \rbrack \rightarrow a_{o} \in \{\mathbb{T}, \mathbb{F}\} \hspace{3mm} \forall X_i
$
\\ \vspace{2mm}
$
s^+ = s^+\lbrack X_i \rbrack := P :
$
\\ \vspace{2mm}
$
(P \lbrack X_i \rbrack \rightarrow y_{o} == a_{o} \hspace{3mm} \forall X_i) \hspace{2mm} \cap \hspace{2mm} (P[\hat{X}_i] \supseteq P[X_i] \hspace{3mm} \forall X_i,\hat{X}_i)
$
\\ \vspace{4mm}
$
P \lbrack X_i \rbrack = \{ s_1,s_2,...,s_N| b_1, b_2,...,b_M,y_o\}
$
\\ \vspace{3mm}
$
s^+ = P \lbrack X_i \rbrack = \{ s_1,s_2,...,s_{O_T \lbrack n \rbrack }, b_1, b_2,...,b_{O_S \lbrack n \rbrack},y_o \} \hspace{3mm} \forall X_i
$
\end{center}





\subsubsection{Property of No-op ;}
No-op ; can be added to any solution $S_i$ without modifying the output $y_o$
\begin{center}
$
s^+ = \{ s_1,s_2,...,s_{O_T \lbrack n \rbrack }, b_1, b_2,...,b_{O_S \lbrack n \rbrack},y_o\}
$
\\ \vspace{2mm}
$
\hat{s}^+ \rightarrow \hat{y}_o = insert \lbrack s^+,;,k \rbrack
$
\\ \vspace{2mm}
$
\hat{y}_o = y_o \hspace{2.5mm} \forall k
$
\end{center}





\subsection{Definition of $S^+$}
Define $S^+$; the set of solutions to decision problem D
\begin{center}
$
X_i = \{x_1,...,x_n,C\}; \hspace{2mm} \hat{X}_i = \{x_1,...,x_n,x_{n+1},C\}
$
\\ \vspace{2mm}
$
D := f \lbrack X_i \rbrack \rightarrow a_{o} \in \{\mathbb{T}, \mathbb{F}\} \hspace{3mm} \forall X_i
$
\\ \vspace{2mm}
$
s_j^+ = s_j^+[X_i] := P :
$
\\ \vspace{2mm}
$
(P \lbrack X_i \rbrack \rightarrow y_{o} == a_{o} \hspace{3mm} \forall X_i) \hspace{2mm} \cap \hspace{2mm} (P[\hat{X}_i] \supseteq P[X_i] \hspace{3mm} \forall X_i,\hat{X}_i)
$
\\ \vspace{2mm}
$
S^+ := \{s^+_j,...\} \hspace{3mm} \forall j
$
\end{center}





\subsection{Definition of Solvable}
Define solvable
\begin{center}
$
X_i = \{x_1,...,x_n,C\}; \hspace{2mm} \hat{X}_i = \{x_1,...,x_n,x_{n+1},C\}
$
\\ \vspace{2mm}
$
D := f \lbrack X_i \rbrack \rightarrow a_{o} \in \{\mathbb{T}, \mathbb{F}\} \hspace{3mm} \forall X_i
$
\\ \vspace{2mm}
$solvable : = solvable \lbrack D \rbrack \rightarrow b_o \in \{ \mathbb{T}, \mathbb{F} \} =$
\\ \vspace{2mm}
$\exists P : (P \lbrack X_i \rbrack \rightarrow y_{o} == a_{o} \hspace{3mm} \forall X_i) \hspace{2mm} \cap \hspace{2mm} (P[\hat{X}_i] \supseteq P[X_i] \hspace{3mm} \forall X_i,\hat{X}_i)
$
\end{center}





\section{The set of all Decision Problems $\mathbb{D}$}

\subsection{Definition}
Define the set of decision problems $\mathbb{D}$
\begin{center}
$
X_i = \{x_1,...,x_n,C\}
$
\\ \vspace{2mm}
$
D_j := f_j \lbrack X_i \rbrack \rightarrow a_{o} \in \{\mathbb{T}, \mathbb{F}\} \hspace{3mm} \forall X_i
$
\\ \vspace{2mm}
$\mathbb{D} := \{D_j,...\} \hspace{3mm} \forall j$
\end{center}





%\section{The Set of All Solutions to Decision Problems $\bold{S^+}$}

%\subsection{Definition}
%Define $\bold{S^+}$ the set of all solutions to decision problems
%\begin{center}
%$
%D_j \in \mathbb{D}
%$
%\\ \vspace{2mm}
%$
%D_j := f_j \lbrack X_i \rbrack \rightarrow a_{o} \in \{\mathbb{T}, \mathbb{F}\} \hspace{3mm} \forall X_i
%$
%\\ \vspace{2mm}
%$
%S^+_j := \{s^+_{1},s^+_{2},...\}
%$
%\\ \vspace{2mm}
%$
%\bold{S^+} := \{S^+_j,...\} \hspace{3mm} \forall j
%$
%\end{center}





\section{Instruction and Memory Notation}
Define $\mathcal{L}$ a set of logical operations\\
Define $\mathcal{M}$ a set of memory elements, magnitudes, and sets
\begin{center}
$
X_i = \{x_1,...,x_n,C\};
$
\\ \vspace{2mm}
$
P \lbrack X_i \rbrack \rightarrow y_o = \{ s_1,s_2,...,s_{O_T \lbrack n \rbrack }, b_1, b_2,...,b_{O_S \lbrack n \rbrack},y_o \}
$
\\ \vspace{2mm}
$
\mathcal{L} := \{ s_1,s_2,...,s_{O_T \lbrack n \rbrack}\}
$
\\ \vspace{2mm}
$
\mathcal{M} := \{ b_1,b_2,...,b_{O_S \lbrack n \rbrack}\}
$
\\ \vspace{2mm}
$
P \lbrack X_i \rbrack = \{ \mathcal{L},\mathcal{M},y_o\}
$
\end{center}





















% Complexity; Time Complexity; Space Complexity; Total Complexity Dimension > 1
\section{Complexity}

\subsection{Time Complexity of a Decision Problem $O_T \lbrack n \rbrack$}
Define Time Complexity $O_T [n]$ of solution $s^+$ to Decision Problem $D$ as the total number of logical operations
\begin{center}
\vspace{1mm}
$
X_i = \{x_1,...,x_n,C\}; \hspace{2mm} \hat{X}_i = \{x_1,...,x_{n+1},C\}
$
\\ \vspace{4mm}
$
D := f \lbrack X_i \rbrack \rightarrow a_{o} \in \{\mathbb{T}, \mathbb{F}\} \hspace{3mm} \forall X_i
$
\\ \vspace{4mm}
$
s^+[X_i] := P :
$
\\ \vspace{2mm}
$
(P \lbrack X_i \rbrack \rightarrow y_{o} == a_{o} \hspace{3mm} \forall X_i) \hspace{2mm} \cap \hspace{2mm} (P[\hat{X}_i] \supseteq P[X_i] \hspace{3mm} \forall X_i,\hat{X}_i)
$
\\ \vspace{4mm}
$
s^+ = \{ s_1,s_2,...,s_N|b_1,b_2,...,b_M,y_o\} = \{ s_1,s_2,...,s_{O_T \lbrack n \rbrack }, b_1, b_2,...,b_{O_S \lbrack n \rbrack},y_o \}
$
\\ \vspace{2mm}
$
= \{ \mathcal{L},\mathcal{M},y_o\}
$
\\ \vspace{3mm}
$
O_T[n] := |\mathcal{L}| = N
$
\end{center}

\subsection{Space Complexity $O_S \lbrack n \rbrack$}
Define Space Complexity $O_S \lbrack n \rbrack$ of solution $s^+$ to Decision Problem $D$ as the total number of memory elements
\begin{center}
\vspace{1mm}
$
X_i = \{x_1,...,x_n,C\}; \hspace{2mm} \hat{X}_i = \{x_1,...,x_{n+1},C\}
$
\\ \vspace{4mm}
$
D := f \lbrack X_i \rbrack \rightarrow a_{o} \in \{\mathbb{T}, \mathbb{F}\} \hspace{3mm} \forall X_i
$
\\ \vspace{4mm}
$
s^+[X_i] := P :
$
\\ \vspace{2mm}
$
(P \lbrack X_i \rbrack \rightarrow y_{o} == a_{o} \hspace{3mm} \forall X_i) \hspace{2mm} \cap \hspace{2mm} (P[\hat{X}_i] \supseteq P[X_i] \hspace{3mm} \forall X_i,\hat{X}_i)
$
\\ \vspace{4mm}
$
s^+ = \{ s_1,s_2,...,s_N|b_1,b_2,...,b_M,y_o\} = \{ s_1,s_2,...,s_{O_T \lbrack n \rbrack }, b_1, b_2,...,b_{O_S \lbrack n \rbrack},y_o \}
$
\\ \vspace{2mm}
$
= \{ \mathcal{L},\mathcal{M},y_o\}
$
\\ \vspace{3mm}
$
O_S[n] := |\mathcal{M}|+|y_o|^* = M + 1
$
\end{center}
\vspace{2mm}
*It is convention to reserve one memory element for output $y_o$.\\
Void programs do not require the $y_o$ memory element for output




\section{Definition of Complexity}
Define Complexity $O[n]$ as a vector of dimension Y
\begin{center}
$
\bold{O}[n] := \hspace{3mm} < O_T [n], O_S [n],O_3[n],O_4[n]...,O_V[n]>
$
\end{center}

\section{Total Complexity}
\begin{center}
$O[n] := O_T[n] + O_S[n] + \sum_{i=3}^{V} O_i[n]$
\end{center}





















% Simple Computational Complexity; Restate Time and Space Complexity; (Simple) Total Complexity
\newpage
\section{Simple Computational Complexity}
The remainder of this document assumes simple computational complexity of dimension 2

\subsection{Definition}
Define simple computational complexity of dimension 2
\begin{center}
$
\bold{O}[n] := \hspace{3mm} < O_T [n], O_S [n] >
$
\end{center}





\subsection{Time Complexity}
Restate definition of Time Complexity $O_T[n]$ of solution $s^+$
\begin{center}
$
s^+ = \{ \mathcal{L},\mathcal{M},y_o\}
$
\\ \vspace{3mm}
$
O_T[n] := |\mathcal{L}| = N
$
\end{center}



\subsection{Space Complexity}
Restate definition of Time Complexity $O_S[n]$ of solution $s^+$
\begin{center}
$
s^+ = \{ \mathcal{L},\mathcal{M},y_o\}
$
\\ \vspace{2mm}
$
O_S[n] := |\mathcal{M}| + |y_o| = M + 1
$
\end{center}





\subsection{Total Complexity}
\begin{center}
$O[n] := O_T[n] + O_S[n]$
\\ \vspace{2mm}
$= |\mathcal{L}| + |\mathcal{M}| + |y_o| = N + M + 1$
\end{center}


\subsection{$O_S[n] > 0$}
\subsubsection{Proof}
Assume $O_S[n]$ = 0
\begin{center}
$
O_S[n] := |\mathcal{M}| + |y_o| 
$
\\ \vspace{2mm}
$
O_S[n] = 0 \Rightarrow \mathcal{M} = y_o = \emptyset
$
\\ \vspace{2mm}
$
y_o = \emptyset; \hspace{2mm} y_o \in \{\mathbb{T},\mathbb{F}\}$ by definition of $s^+$
\\ \vspace{4mm}
$
\therefore O_S[n] = 0$ contradicts the definition of solution $s^+$ of a decision problem
\\ \vspace{2mm}
$
O_S[n] \geq 0$ by definition of magnitude
\\ \vspace{2mm}
$
\therefore O_S[n] > 0
$
\end{center}


\subsection{$O_T[n] > 0$}
\subsubsection{Proof}
Assume $O_T[n] = 0$
\begin{center}
$
O_T[n] := |\mathcal{L}| 
$
\\ \vspace{2mm}
$
O_T[n] = 0 \Rightarrow y_o \not \in \{\mathbb{T},\mathbb{F}\}
$
\\ \vspace{2mm}
$
y_o \not \in \{\mathbb{T},\mathbb{F}\}; \hspace{2mm} y_o \in \{\mathbb{T},\mathbb{F}\}$ by definition of $s^+$
\\ \vspace{4mm}
$
\therefore O_T[n] = 0$ contradicts the definition of solution $s^+$ of a decision problem
\\ \vspace{2mm}
$
O_T[n] \geq 0$ by definition of magnitude
\\ \vspace{2mm}
$
\therefore O_T[n] > 0
$
\end{center}




\subsection{O[n] > 0}
\subsubsection{Proof}
\begin{center}
$
O[n] := O_T[n] + O_S[n]
$
\\ \vspace{3mm}
$
O_T[n] > 0; \hspace{2mm} O_S[n] > 0
$
\\ \vspace{3mm}
$
\therefore O[n] > 0
$
\end{center}

\subsection{$O[n] > O_T[n]$}
\subsubsection{Proof}
\begin{center}
$
O[n] := O_T[n] + O_S[n]
$
\\ \vspace{3mm}
$
\hspace{2mm} O_S[n] > 0
$
\\ \vspace{3mm}
$
\therefore O[n] > O_T[n]
$
\end{center}

\subsection{$O[n] > O_S[n]$}
\subsubsection{Proof}
\begin{center}
$
O[n] := O_T[n] + O_S[n]
$
\\ \vspace{3mm}
$
O_T[n] > 0
$
\\ \vspace{3mm}
$
\therefore O[n] > O_S[n]
$
\end{center}


\subsection{$O[n+1] \geq O[n]$}
\subsubsection{Proof}
\begin{center}
$
X_i = \{x_1,...,x_n,C\}; \hspace{2mm} \hat{X}_i = \{x_1,...,x_{n+1},C\}
$
\\ \vspace{4mm}
$
O[n] = |\hspace{1mm} s^+[X_i]|
$
\\ \vspace{2mm}
$
O[n+1] = \hat{O}[n] = |\hspace{1mm} s^+[\hat{X}_i]|
$
\end{center}
\hspace{4mm}
For general solutions $s^+$
\begin{center}
$
s^+[\hat{X}_i] \supseteq s^+[X_i] 
$
\\ \vspace{2mm}
$
\Rightarrow |s^+[\hat{X}_i]| \geq |s^+[X_i]|
$
\\ \vspace{2mm}
$
\therefore \hat{O}[n] = O[n+1] \geq O[n]
$
\end{center}










%\newpage
%\section{Convergent Complexity}
%\subsection{Definition}
%Define Convergent Complexity; the set of solutions with complexity satisfying
%\begin{center}
%$
%limit_{n \rightarrow \infty} \frac{O[n+1]}{O[n]} = c
%$
%\end{center}
%where c is a constant


%\subsection{Derivative Property of Convergent Solutions}
%There exists an nth derivative equal to zero
%\begin{center}
%$
%limit_{n \rightarrow \infty} \frac{O[n+1]}{O[n]} = c
%$
%\end{center}









%\subsection{Show c=1 for all convergent solutions}
%O[n] is an integer valued, non-decreasing function
%\begin{center}
%$
%O[n] \in \mathbb{Z}
%$
%\\ \vspace{2mm}
%$
%O[n+1] \geq O[n]; \hspace{2mm} f_{n+1}[n] = O[n+1] - O[n] \geq 0
%$
%\\ \vspace{6mm}
%$
%limit_{n \rightarrow \infty} \frac{O[n+1]}{O[n]} = limit_{n \rightarrow \infty} \frac{O[n] + f_{n+1}[n]}{O[n]}
%$
%\\ \vspace{2mm}
%$
%=  limit_{n \rightarrow \infty}(\frac{O[n]}{O[n]} +   \frac{f_{n+1}[n]}{O[n]})
%$
%\\ \vspace{2mm}
%$
%= 1 +   limit_{n \rightarrow \infty} \frac{f_{n+1}[n]}{O[n]} = c
%$
%\end{center}









%\subsection{Convergent Inductive Function}
%\begin{center}
%$
%limit_{n \rightarrow \infty} \frac{O[n+1]}{O[n]} = limit_{n \rightarrow \infty} \frac{O[n] + f_{n+1}[n]}{O[n]}
%$
%\\ \vspace{2mm}
%$
%=  limit_{n \rightarrow \infty}(\frac{O[n]}{O[n]} +   \frac{f_{n+1}[n]}{O[n]})
%$
%\\ \vspace{2mm}
%$
%= 1 +   limit_{n \rightarrow \infty} \frac{f_{n+1}[n]}{O[n]} = c
%$
%\\ \vspace{2mm}
%$
%= limit_{n \rightarrow \infty} \frac{f_{n+1}[n]}{O[n]} = c - 1
%$
%\end{center}


















% Total Polynomial Complexity; Set of all Polynomial Problems; Polynomial Order of Complexity
\newpage
\section{Polynomial Complexity}

\subsection{Definition}
Decision problem $D$ with solution $s^+$ has polynomial total complexity $O[n]$ if
\begin{center}
$\exists K,C,\lambda_1...\lambda_K \hspace{2mm} :$
\\ \vspace{2mm}
$O[n] = (\lambda_K n)^K + (\lambda_{K-1} n)^{K-1}... + \lambda_1 n + C \hspace{3mm} \forall n$
\end{center}





\subsection{Polynomial Problems}
Define $\mathbb{P}$, the set of Decision Problems that can be solved with Polynomial Complexity
\begin{center}
$
\mathbb{P} := \{D_1,D_2,...\} : 
$
\\ \vspace{4mm}
$
\exists K,C,\lambda_1...\lambda_K : 
$
\\
$
O[n] = (\lambda_K n)^K + (\lambda_{K-1} n)^{K-1}... + \lambda_1 n + C \hspace{4mm} \forall n, D_i \in \mathbb{P}
$
\end{center}





\subsection{Polynomial Order of Complexity}
Solution $s^+$ with total complexity $O[n]$ is said to be of order $n^K$
\begin{center}
$
 O[n] \sim n^K
$
\\ \vspace{2mm}
$O[n] = (\lambda_{K} n)^{K} + (\lambda_{K-1} n)^{K-1}... + \lambda_1 n +  C \hspace{4mm} \forall n$
\end{center}








\subsection{Property of Polynomial  Complexity 1}
Solutions with polynomial complexity have convergent complexity
\begin{center}
$
lim_{n \rightarrow \infty} \frac{O[n+1]}{O[n]} = 1
$
\end{center}
\subsubsection{Proof}
\begin{center}
$
O[n] = (\lambda_K n)^K + (\lambda_{K-1} n)^{K-1}... + \lambda_1 n + C
$
\\ \vspace{2mm}
$
O[n+1] = (\lambda_K (n+1))^K + (\lambda_{K-1} (n+1))^{K-1}... + \lambda_1 (n+1) + C
$
\\ \vspace{2mm}
$ 
=  (\lambda_K n)^K + (\tilde{\lambda}_{K-1} n)^{K-1}... + \tilde{\lambda_1} n + \tilde{C}
$
\\ \vspace{5mm}
$
lim_{n \rightarrow \infty} \frac{O[n+1]}{O[n]}
$
\\ \vspace{3mm}
$
= lim_{n \rightarrow \infty} \frac{ (\lambda_K n)^K + (\tilde{\lambda}_{K-1} n)^{K-1}... + \tilde{\lambda_1} n + \tilde{C}}{(\lambda_K n)^K + (\lambda_{K-1} n)^{K-1}... + \lambda_1 n + C}
$
\\ \vspace{3mm}
$
= lim_{n \rightarrow \infty} \frac{ (\lambda_K n)^K}{(\lambda_K n)^K + (\lambda_{K-1} n)^{K-1}... + \lambda_1 n + C} + \frac{ (\tilde{\lambda}_{K-1} n)^{K-1}}{(\lambda_K n)^K + (\lambda_{K-1} n)^{K-1}... + \lambda_1 n + C} + ... + \frac{ \tilde{\lambda}_1 n}{(\lambda_K n)^K + (\lambda_{K-1} n)^{K-1}... + \lambda_1 n + C} +  \frac{ \tilde{C}}{(\lambda_K n)^K + (\lambda_{K-1} n)^{K-1}... + \lambda_1 n + C}
$
\\ \vspace{3mm}
$
= 1 = lim_{n \rightarrow \infty} \frac{O[n+1]}{O[n]}
$
\end{center}







\subsection{Property of Polynomial Complexity 2}
\begin{center}
$
\exists K,\hat{C},\hat{\lambda}_1,...,\hat{\lambda}_{K-1} :
$
\\ \vspace{2mm}
$
O[n+1] - O[n] = f_{n+1}[n] = (\hat{\lambda}_{K-1} n)^{K-1}... + \hat{\lambda}_1 n + \hat{C} \hspace{3mm} \forall n
$
\end{center}
\subsubsection{Proof}
\begin{center}
$
O[n] = (\lambda_K n)^K + (\lambda_{K-1} n)^{K-1}... + \lambda_1 n + C
$
\\ \vspace{2mm}
$
O[n+1] = (\lambda_K (n+1))^K + (\lambda_{K-1} (n+1))^{K-1}... + \lambda_1 (n+1) + C
$
\\ \vspace{2mm}
$ 
=  (\lambda_K n)^K + (\tilde{\lambda}_{K-1} n)^{K-1}... + \tilde{\lambda_1} n + \tilde{C}
$
\\ \vspace{2mm}
$
O[n+1] - O[n] = ((\tilde{\lambda}_{K-1}- \lambda_{K-1})n)^{K-1}... + (\tilde{\lambda_1} - \lambda_1) n + (\tilde{C} - C)
$
\\ \vspace{2mm}
$
O[n+1] - O[n] = (\hat{\lambda}_{K-1}n)^{K-1}... + \hat{\lambda}_1  n + \hat{C}
$
\end{center}







\subsection{Total Polynomial Complexity Implies Time bounded Polynomial Complexity}
\begin{center}
\vspace{1mm}
$
D \in \mathbb{P} \Longrightarrow O_T[n] < (\lambda_K n)^K + (\lambda_{K-1} n)^{K-1}... + \lambda_1 n + C
$
\end{center}

\subsubsection{Proof}
\begin{center}
$
O[n] = (\lambda_K n)^K + (\lambda_{K-1} n)^{K-1}... + \lambda_1 n + C \hspace{2mm} \forall n
$
\\ \vspace{2mm}
$
O[n] := O_T[n] + O_S[n]; \hspace{2mm} O_S[n] > 0
$
\\ \vspace{2mm}
$
\therefore O_T[n] < (\lambda_K n)^K + (\lambda_{K-1} n)^{K-1}... + \lambda_1 n + C \hspace{2mm} \forall n
$
\end{center}










\subsection{Total Polynomial Complexity Implies Space bounded Polynomial Complexity}
\begin{center}
\vspace{1mm}
$
D \in \mathbb{P} \Longrightarrow O_S[n] <  (\lambda_K n)^K + (\lambda_{K-1} n)^{K-1}... + \lambda_1 n + C
$
\end{center}

\subsubsection{Proof}
\begin{center}
$
O[n] = (\lambda_K n)^K + (\lambda_{K-1} n)^{K-1}... + \lambda_1 n + C \hspace{2mm} \forall n
$
\\ \vspace{2mm}
$
O[n] := O_T[n] + O_S[n]; \hspace{2mm} O_T[n] > 0
$
\\ \vspace{2mm}
$
\therefore O_S[n] < (\lambda_K n)^K + (\lambda_{K-1} n)^{K-1}... + \lambda_1 n + C \hspace{2mm} \forall n
$
\end{center}










%\subsection{Total Polynomial Complexity iff Time and Space bounded by Polynomial Complexity}
%Use limit definition











%\subsection{Theorem Either OT or OS is on the order of Oopt}
%Proof by contradiction








% Non Polynomial Problems
\newpage
\section{Non-Polynomial Complexity}
\subsection{Definition}
Decision problem $\tilde{D}$ with solution $s^+$ has non-polynomial total complexity $O[n]$ if
\begin{center}
$\not \exists K,C,\lambda_1...\lambda_K \hspace{1mm} :$
\\ \vspace{2mm}
$O[n] = (\lambda_K n)^K + (\lambda_{K-1} n)^{K-1}... + \lambda_1 n + C \hspace{3mm} \forall n$
\end{center}


\subsection{Non-Polynomial Problems}
Define $\mathcal{N}$, the set of Decision Problems that cannot be solved with Polynomial Complexity
\begin{center}
$
\mathcal{N} := \{\tilde{D}_1,\tilde{D}_2,...\} :
$
\\ \vspace{2mm}
$\not \exists K,C,\lambda_1...\lambda_K \hspace{1mm} :$
\\ \vspace{2mm}
$O[n] = (\lambda_K n)^K + (\lambda_{K-1} n)^{K-1}... + \lambda_1 n + C \hspace{3mm} \forall n, \hspace{1mm}  s^+ \in S^+_i, \hspace{1mm} \tilde{D}_i \in \mathcal{N}$
\end{center}

\subsection{$\mathbb{P}$ and $\mathcal{N}$ are disjoint}
\begin{center}
\vspace{2mm}
$
\mathbb{P} \cap \mathcal{N} = \emptyset
$
\end{center}

\subsubsection{Proof}
Let D $\in \mathcal{N}$
\begin{center}
$\not \exists K,C,\lambda_1...\lambda_K \hspace{1mm} :$
\\ \vspace{2mm}
$O[n] = (\lambda_K n)^K + (\lambda_{K-1} n)^{K-1}... + \lambda_1 n + C \hspace{3mm} \forall n$
\end{center}
\vspace{4mm}
Assume D $\in \mathbb{P}$
\begin{center}
$\exists K,C,\lambda_1...\lambda_K \hspace{1mm} :$
\\ \vspace{2mm}
$O[n] = (\lambda_K n)^K + (\lambda_{K-1} n)^{K-1}... + \lambda_1 n + C \hspace{3mm} \forall n$
\\ \vspace{4mm}
Contradicts the definition of $\mathcal{N}$
\\ \vspace{2mm}
$
\therefore D \in \mathcal{N} \Rightarrow D \not \in \mathbb{P}
$
\end{center}
\vspace{6mm}
Let D $\in \mathbb{P}$
\begin{center}
$\exists K,C,\lambda_1...\lambda_K \hspace{1mm} :$
\\ \vspace{2mm}
$O[n] = (\lambda_K n)^K + (\lambda_{K-1} n)^{K-1}... + \lambda_1 n + C \hspace{3mm} \forall n$
\end{center}
\vspace{4mm}
Assume D $\in \mathcal{N}$
\begin{center}
$\not \exists K,C,\lambda_1...\lambda_K \hspace{1mm} :$
\\ \vspace{2mm}
$O[n] = (\lambda_K n)^K + (\lambda_{K-1} n)^{K-1}... + \lambda_1 n + C \hspace{3mm} \forall n$
\\ \vspace{4mm}
Contradicts the definition of $\mathbb{P}$
\\ \vspace{2mm}
$
\therefore D \in \mathbb{P} \Rightarrow D \not \in \mathcal{N}
$
\\ \vspace{6mm}
$
D \in \mathcal{N} \Rightarrow D \not \in \mathbb{P}; D \in \mathbb{P} \Rightarrow D \not \in \mathcal{N}
$
\\ \vspace{2mm}
$
\therefore \mathbb{P} \cap \mathcal{N} = \emptyset
$
\end{center}


% Divergent Problems
\section{Divergent Complexity}
\subsection{Defintion}
Decision problem $\hat{D}$ with solution $s^+$ has divergent total complexity $O[n]$ if
\begin{center}
$
lim_{n \rightarrow \infty} \frac{O[n+1]}{O[n]} \hspace{2mm} diverges \hspace{3mm} \forall n
$
\end{center}




\subsection{Divergent Problems}
\begin{center}
$
\mathcal{\hat{D}} := \{ \hat{D}_1,\hat{D}_2,...\} :
$
\\ \vspace{2mm}
$
lim_{n \rightarrow \infty} \frac{O[n+1]}{O[n]} \hspace{2mm} diverges \hspace{3mm} \forall s^+ \in S^+_i, \hspace{1mm} \hat{D}_i \in \mathcal{\hat{D}}
$
\end{center}





%\subsection{Derivative Property of Divergent Solutions}
%\begin{center}
%$
%lim_{n \rightarrow \infty} \hspace{2mm} O[n+1] - O[n]$ diverges
%\end{center}

%\subsubsection{Proof}




%\subsection{Theorem of Divergent Subfunctions}
%If an (inductive) subfunction of $s^+$ diverges, the solution is divergent
%\begin{center}
%$
%f_{n+1} = \sum g_{n+1}
%$
%\\ \vspace{2mm}
%$
%limit \frac{g[n+1]}{g[n]} diverges
%$
%\\ \vspace{3mm}
%$
%\exists g_{n+1} : limit \frac{g_{n+1}}{O[n]} diverges \Longrightarrow limit \frac{O[n+1]}{O[n]} diverges
%$
%\end{center}

%\subsection{Proof}




\subsection{The Set of Polynomial Solutions and the Set of Divergent Solutions are disjoint}
\begin{center}
\vspace{2mm}
$
\mathbb{P} \cap \hat{D} = \emptyset
$
\end{center}

\subsection{Proof}
Let D $\in \hat{\mathcal{D}}$
\begin{center}
$
lim_{n \rightarrow \infty} \frac{O[n+1]}{O[n]} \hspace{2mm} diverges$ by definition
\end{center}
\vspace{4mm}
Assume D $\in \mathbb{P}$
\begin{center}
$
lim_{n \rightarrow \infty} \frac{O[n+1]}{O[n]} = 1
$
\\ \vspace{2mm}
$
 lim_{n \rightarrow \infty} \frac{O[n+1]}{O[n]}$ = 1 contradicts the definition of Divergent Problems
\\ \vspace{2mm}
$
\therefore D \in \hat{\mathcal{D}} \Rightarrow D \not \in \mathbb{P}
$
\end{center}
\vspace{12mm}
Let D $\in \mathbb{P}$
\begin{center}
$
lim_{n \rightarrow \infty} \frac{O[n+1]}{O[n]} = 1$ by property of Polynomial complexity
\end{center}
\vspace{4mm}
Assume D $\in \hat{D}$
\begin{center}
$
lim_{n \rightarrow \infty} \frac{O[n+1]}{O[n]}$ diverges
\\ \vspace{2mm}
$
 lim_{n \rightarrow \infty} \frac{O[n+1]}{O[n]}$ diverges contradicts a property of Polynomial complexity
\\ \vspace{2mm}
$
\therefore D \in \mathbb{P} \Rightarrow D \not \in \hat{\mathcal{D}}
$
\\ \vspace{2mm}
$
\therefore \mathbb{P} \cap \hat{\mathcal{D}} = \emptyset
$
\end{center}





% Theorem of Computational Duality; Polynomial in time and space
\newpage
\section{Inductive Functions}














% Inductive Function
\subsection{Inductive Function $f_{n+1}$}
\begin{center}
\vspace{2mm}
$
O[n] := O_T[n] + O_S[n]
$
\\ \vspace{2mm}
$
O[n+1] = O_T[n+1] + O_S[n+1]
$
\\ \vspace{4mm}
$
f_{n+1}[n] := O[n+1] - O[n]
$
\end{center}










\subsection{Inductive Space and Time Formulas}
\begin{center}
$
f^T_{n+1}[n] := O_T[n+1] - O_T[n]
$
\\ \vspace{2mm}
$
O_T[n+1] = O_T[n] + f^T_{n+1}[n]
$
\\ \vspace{2mm}
$
f^S_{n+1}[n] := O_S[n+1] - O_S[n]
$
\\ \vspace{2mm}
$
O_S[n+1] = O_S[n] + f^S_{n+1}[n]
$

\end{center}





% Theorem of Polynomia? Duality
\subsection{Inductive Function Expressions}
Relate $f_{n+1}[n]$ to equivalence functions
\begin{center}
$
D \in \mathbb{P}
$
\\ \vspace{2mm}
$
O[n] := O_T[n] + O_S[n]
$
\\ \vspace{2mm}
$
O[n+1] = O_T[n+1] + O_S[n+1] = O[n] + f_{n+1}[n]
$
\\ \vspace{2mm}
$
O_T[n] = O[n] - O_S[n]
$
\\ \vspace{2mm}
$
O_S[n] = O[n] - O_T[n]
$
\\ \vspace{8mm}
$
f_{n+1} = O[n+1] - O[n]
$
\\ \vspace{2mm}
$
f_{n+1} = O_T[n+1] + O_S[n+1] - O[n]
$
\\ \vspace{2mm}
$
f_{n+1} = O_T[n+1] - O_T[n] + O_S[n+1] - O_S[n]
$
\\ \vspace{2mm}
$
f_{n+1} = O[n+1] - O_T[n] - O_S[n]
$
\\ \vspace{2mm}
$
f_{n+1}[n] =  f^T_{n+1}[n] +  f^S_{n+1}[n]
$
\end{center}






\subsection{Zero Order Inductive Function}
\begin{center}
$
Let \hspace{2mm} O_S[n] \sim n^0
$
\\ \vspace{2mm}
$
f_{n+1} = O_T[n+1] - O_T[n] + O_S[n+1] - O_S[n] = O_T[n+1] - O_T[n]
$
\end{center}


\subsection{Property of Polynomial Complexity}
\begin{center}
$
limit_{n \rightarrow \infty } \frac{f_{n+1}[n]}{O[n]} = 0
$
\end{center}

\subsubsection{Proof}
\begin{center}
$
limit_{n \rightarrow \infty } \frac{O[n+1]}{O[n]} = 1
$
\\ \vspace{2mm}
$
limit_{n \rightarrow \infty } \frac{O[n] + f_{n+1}[n]}{O[n]} = 1
$
\\ \vspace{2mm}
$
limit_{n \rightarrow \infty } \frac{O[n]}{O[n]}+  \frac{f_{n+1}[n]]}{O[n]}  = 1
$
\\ \vspace{2mm}
$
limit_{n \rightarrow \infty } 1 +  \frac{f_{n+1}[n]]}{O[n]}  = 1
$
\\ \vspace{2mm}
$
limit_{n \rightarrow \infty }  \frac{f_{n+1}[n]]}{O[n]}  = 0
$
\end{center}
















% Subfunctions
\newpage
\section{Subfunctions}





\subsection{Restate the subfunction condition of general solutions}
Recall the definition of general solution $s^+$
\begin{center}
$
X_i = \{x_1,...,x_n,C\}; \hspace{2mm} \hat{X}_i = \{x_1,...,x_{n+1},C\}
$
\\ \vspace{2mm}
$
s^+ = s^+[X_i] := P :
$
\\ \vspace{2mm}
$
(P \lbrack X_i \rbrack \rightarrow y_{o} == a_{o} \hspace{3mm} \forall X_i) \hspace{2mm} \cap \hspace{2mm} (P[\hat{X}_i] \supseteq P[X_i] \hspace{3mm} \forall X_i,\hat{X}_i)
$
\end{center}
\vspace{3mm}
The subfunction condition is one of two conditions for a general solution
\begin{center}
$
P[\hat{X}_i] \supseteq P[X_i] \hspace{3mm} \forall X_i,\hat{X}_i
$
\end{center}






\subsection{Prove O[n] is a non-decreasing function}
Consider solution $s^+$ with complexity $O[n]$
\begin{center}
\vspace{1mm}
$
X_i = \{x_1,...,x_n,C\}; \hspace{2mm} \hat{X}_i = \{x_1,...,x_{n+1},C\}
$
\\ \vspace{2mm}
$
s^+ = s^+[X_i] := P :
$
\\ \vspace{2mm}
$
(P \lbrack X_i \rbrack \rightarrow y_{o} == a_{o} \hspace{3mm} \forall X_i) \hspace{2mm} \cap \hspace{2mm} (P[\hat{X}_i] \supseteq P[X_i] \hspace{3mm} \forall X_i,\hat{X}_i)
$
\\ \vspace{4mm}
$
s^+ = \{ s_1,s_2,...,s_N|b_1,b_2,...,b_M,y_o\} = \{ s_1,s_2,...,s_{O_T \lbrack n \rbrack }, b_1, b_2,...,b_{O_S \lbrack n \rbrack},y_o \}
$
\\ \vspace{2mm}
$
= \{ \mathcal{L},\mathcal{M},y_o\}
$
\\ \vspace{5mm}
$
O[n] := O_T[n] + O_S[n]
$
\\ \vspace{3mm}
$
O_T[n] := |\mathcal{L}| = N
$
\\ \vspace{2mm}
$
O_S[n] := |\mathcal{M}| + |y_o| = M + 1
$
\\ \vspace{6mm}
\end{center}
O[n+1] denotes the total complexity for solution $s^+[\hat{X}_i]$\\\\
\begin{center}
$
s^+[\hat{X}_i] = \hat{s}^+
$
\end{center}
Let
\begin{center}
$
O[n+1] < O[n]
$
\\ \vspace{2mm}
$
\Rightarrow \hat{N} + \hat{M} < N + M
$
\\ \vspace{2mm}
$
\hat{s}^+ = \{ s_1,s_2,...,s_{\hat{N}}|b_1,b_2,...,b_{\hat{M}},y_o\}
$
\\ \vspace{6mm}
$
\Rightarrow \hat{s}^+ \not \supseteq s^+
$
\\ \vspace{2mm}
$
P[\hat{X}_i] \not \supseteq P[X_i] \hspace{3mm} \forall X_i,\hat{X}_i
$
\\ \vspace{6mm}
$
\therefore O[n+1] < O[n]$ contradicts the definition of solution $s^+$
\\ \vspace{2mm}
$
O[n+1] \geq O[n]
$
\end{center}




% Definition of a subfunction
\subsection{Definition of Subfunction}
\vspace{1mm}
\begin{center}
$
X_i = \{x_1,...,x_n,C\}; \hspace{2mm} \hat{X}_i = \{x_1,...,x_{n+1},C\}
$
\\ \vspace{2mm}
$
s^+ = s^+[X_i] := P :
$
\\ \vspace{2mm}
$
(P \lbrack X_i \rbrack \rightarrow y_{o} == a_{o} \hspace{3mm} \forall X_i) \hspace{2mm} \cap \hspace{2mm} (P[\hat{X}_i] \supseteq P[X_i] \hspace{3mm} \forall X_i,\hat{X}_i)
$
\\ \vspace{4mm}
$
s^+ = \{ s_1,s_2,...,s_N|b_1,b_2,...,b_M,y_o\} = \{ s_1,s_2,...,s_{O_T \lbrack n \rbrack }, b_1, b_2,...,b_{O_S \lbrack n \rbrack},y_o \}
$
\\ \vspace{2mm}
$
= \{ \mathcal{L},\mathcal{M},y_o\}
$
\\ \vspace{6mm}
$
Sub[X_i] := S = \{s_j,...|b_k,...,y_o\}:
$
\\ \vspace{2mm}
$
s_j,b_k \in s^+ \hspace{3mm} \forall s_j,b_k \in S
$
\end{center}







\subsubsection{$s^+[X_i]$ is a subfunction of $s^+[\hat{X}_i]$}
\begin{center}
\vspace{1mm}
$
s^+ = \{ s_1,s_2,...,s_N|b_1,b_2,...,b_M,y_o\} = \{ s_1,s_2,...,s_{O_T \lbrack n \rbrack }, b_1, b_2,...,b_{O_S \lbrack n \rbrack},y_o \}
$
\\ \vspace{2mm}
$
\hat{s}^+ = \{ s_1,s_2,...,s_N,...,s_{\hat{N}}|b_1,b_2,...,b_M,...,b_{\hat{M}},y_o\}; \hspace{2mm} \hat{N} + \hat{M} \geq N + M
$
\end{center}
\vspace{2mm}
By definition of solution
\begin{center}
\vspace{1mm}
$
\hat{s}^+ = P[\hat{X}_i] \supseteq P[X_i] = s^+ \hspace{3mm} \forall X_i,\hat{X}_i
$
\\ \vspace{2mm}
$
\Rightarrow s_j,b_k \in \hat{s}^+ \hspace{2mm} \forall s_j,b_k \in s^+
$
\end{center}



% Subfunction Decomposition
\subsection{Subfunction Decomposition of Solutions}
FIX Double check conditions!!!
Solutions $s^+$ can be written as the union of subfunctions $Sub_k[X_i]$
\begin{center}
$
X_i = \{x_1,...,x_n,C\}; \hspace{2mm} \hat{X}_i = \{x_1,...,x_{n+1},C\}
$
\\ \vspace{2mm}
$
s^+ = s^+[X_i] := P :
$
\\ \vspace{2mm}
$
(P \lbrack X_i \rbrack \rightarrow y_{o} == a_{o} \hspace{3mm} \forall X_i) \hspace{2mm} \cap \hspace{2mm} (P[\hat{X}_i] \supseteq P[X_i] \hspace{3mm} \forall X_i,\hat{X}_i)
$
\\ \vspace{4mm}
$
s^+ = \{ s_1,s_2,...,s_N|b_1,b_2,...,b_M,y_o\} = \{ s_1,s_2,...,s_{O_T \lbrack n \rbrack }, b_1, b_2,...,b_{O_S \lbrack n \rbrack},y_o \}
$
\\ \vspace{2mm}
$
= \{ \mathcal{L},\mathcal{M},y_o\}
$
\\ \vspace{4mm}
$
s^+ = Sub_1[X_i] \cup Sub_2[X_i] \cup ... \cup Sub_z[X_i]
$
\\ \vspace{3mm}
$
= \{ \mathcal{L}_1| \mathcal{M}_1,y_o\} \cup \{  \mathcal{L}_2| \mathcal{M}_2,y_o\} \cup ... \cup \{  \mathcal{L}_z| \mathcal{M}_z,y_o\} : 
$
\\ \vspace{2mm}
$
\mathcal{L}_j \cap \mathcal{L}_k = \emptyset \hspace{2mm} \forall j,k\neq j
$
\\ \vspace{4mm}
$
s^+ = \{ s_{1}^1,...,s^1_{N_1}| b^1_{1},...,y_o\} \cup \{  s_{1}^2,...,s^2_{N_2}| b^2_{1},...,y_o\} \cup ... \cup \{ s_{1}^z,...,s^z_{N_z}| b^z_{1},...,y_o\}:
$
\\ \vspace{2mm}
$
\sum_{l=1}^z N_l = N = O_T[n]
$
\end{center}





\section{Subfunction Complexity}
%\begin{center}
%\vspace{2mm}
%$
%s^+[X_i] = \{ s_1,s_2,...,s_N|b_1,b_2,...,b_M,y_o\} = \{ \mathcal{L},\mathcal{M},y_o\}
%$
%\\ \vspace{6mm}
%$
%Sub[X_i] := S = \{s_j,...|b_k,...,y_o\}:
%$
%\\ \vspace{2mm}
%$
%s_j,b_k \in s^+ \hspace{3mm} \forall s_j,b_k \in S
%$
%\\ \vspace{4mm}
%$
%s^+ = Sub_1[X_i] \cup Sub_2[X_i] \cup ... \cup Sub_z[X_i]
%$
%\\ \vspace{2mm}
%$
%= \{ \mathcal{L}_1| \mathcal{M}_1,y_o\} \cup \{  \mathcal{L}_2| \mathcal{M}_2,y_o\} \cup ... \cup \{  \mathcal{L}_z| \mathcal{M}_z,y_o\} : 
%$
%\\ \vspace{2mm}
%$
%\mathcal{L}_j \cap \mathcal{L}_k = \emptyset \hspace{2mm} \forall j,k\neq j
%$
%\\ \vspace{4mm}
%$
%s^+ = \{ s_{1}^1,...,s^1_{N_1}| b^1_{1},...,y_o\} \cup \{  s_{1}^2,...,s^2_{N_2}| b^2_{1},...,y_o\} \cup ... \cup \{ s_{1}^z,...,s^z_{N_z}| b^z_{1},...,y_o\}:
%$
%\\ \vspace{2mm}
%$
%\sum_{l=1}^z N_l = N = O_T[n]
%$
%\\ \vspace{2mm}
%$
%= \{ s_1,...,s_{i_{N1}}| b_{i_1},...,y_o\} \cup \{ s_{i_2},...,s_{i_{N2}}| b_{i_2},...,y_o\} \cup ... \cup \{ s_{i_z},...,s_N| b_{i_z},...,y_o\}
%$
%\\ \vspace{6mm}
%$
%= \{ \mathcal{L}_1| \mathcal{M}_1\} \cup \{  \mathcal{L}_2| \mathcal{M}_2\} \cup ... \cup \{  \mathcal{L}_z| \mathcal{M}_z\} :
%$
%\\ \vspace{2mm}
%$
%(\cup_{i=1}^z \mathcal{L}_i = \mathcal{L}) \cap (\cup_{i=1}^z \mathcal{M}_i = \mathcal{M})
%$
%\\ \vspace{2mm}
%$
%(s_{i_x} \in s^+ \hspace{2mm} \forall s_{i_x}) \hspace{1mm} \cap \hspace{1mm} (b_{i_x} \in s^+ \hspace{2mm} \forall b_{i_x}) 
%$
%\end{center}



\subsection{Disjoint Subfunction Operations}
\begin{center}
$
\mathcal{L}_i \cap \mathcal{L}_j = \emptyset \hspace{2mm} \forall i,j\neq i
$
\end{center}

\subsection{Shared Subfunction Memory}
\begin{center}
$
|\mathcal{M}_i \cap \mathcal{M}_j| \geq 0 \hspace{2mm} \forall i,j\neq i
$
\end{center}



\subsubsection{Time Complexity of Subfunctions}
Subfunction time complexity is additive
\begin{center}
$
s^+ = \{ \mathcal{L},\mathcal{M},y_o\}
$
\\ \vspace{6mm}
$
Sub_i[X] := S_i = \{s_j,...|b_k,...,y_o\}:
$
\\ \vspace{2mm}
$
s_j,b_k \in s^+ \hspace{3mm} \forall s_j,b_k \in S_i
$
\\ \vspace{6mm}
$
s^+ = \{ \mathcal{L}_1| \mathcal{M}_1,y_o\} \cup \{  \mathcal{L}_2| \mathcal{M}_2,y_o\} \cup ... \cup \{  \mathcal{L}_z| \mathcal{M}_z,y_o\} :
$
\\ \vspace{2mm}
$
\mathcal{L}_i \cap \mathcal{L}_j = \emptyset \hspace{2mm} \forall i,j\neq i
$
\\ \vspace{6mm}
$
\mathcal{L} = \cup_{i=1}^z \mathcal{L}_i
$
\\ \vspace{2mm}
$
\mathcal{L}_i \cap \mathcal{L}_j = \emptyset \hspace{2mm} \forall i,j\neq i
$
\\ \vspace{6mm}
$
O_T[n] = |\mathcal{L}| = N
$
\\ \vspace{2mm}
$
O_T[n]  =  |\cup_{i=1}^z \mathcal{L}_i| = \sum_{i=1}^z |\mathcal{L}_i|^*=|\mathcal{L}_1| + |\mathcal{L}_2|  + ... + |\mathcal{L}_z| 
$
\\ \vspace{2mm}
$
= O_{T_1}[n] + O_{T_2}[n] + ... + O_{T_z}[n] =  N_1 + N_2 + ... + N_z
$
\end{center}
\vspace{4mm}
$^*$Due to the disjoint condition of subfunction operations $\mathcal{L}_i \cap \mathcal{L}_j = \emptyset \hspace{2mm} \forall i,j\neq i$



\subsubsection{Space Complexity of Subfunctions}
Subfunctions can access the full memory $\mathcal{M}$ with no added space complexity 
\begin{center}
\vspace{1mm}
$
s^+ = \{ \mathcal{L},\mathcal{M},y_o\}
$
\\ \vspace{6mm}
$
Sub_i[X] := S_i = \{s_j,...|b_k,...,y_o\}:
$
\\ \vspace{2mm}
$
s_j,b_k \in s^+ \hspace{3mm} \forall s_j,b_k \in S_i
$
\\ \vspace{6mm}
$
s^+ = \{ \mathcal{L}_1| \mathcal{M}_1,y_o\} \cup \{  \mathcal{L}_2| \mathcal{M}_2,y_o\} \cup ... \cup \{  \mathcal{L}_z| \mathcal{M}_z,y_o\} :
$
\\ \vspace{2mm}
$
\mathcal{L}_i \cap \mathcal{L}_j = \emptyset \hspace{2mm} \forall i,j\neq i
$
\\ \vspace{4mm}
$
s^+ = \{ \mathcal{L}_1| \mathcal{M},y_o\} \cup \{  \mathcal{L}_2| \mathcal{M},y_o\} \cup ... \cup \{  \mathcal{L}_z| \mathcal{M},y_o\} :
$
\\ \vspace{2mm}
$
\mathcal{L}_i \cap \mathcal{L}_j = \emptyset \hspace{2mm} \forall i,j\neq i
$
\\ \vspace{4mm}
$
\mathcal{M} = \cup_{i=1}^z \mathcal{M}_i  = \cup_{i=1}^z \mathcal{M}
$
\\ \vspace{4mm}
$
O_S[n] = |\mathcal{M}| = M
$
\\ \vspace{2mm}
$
O_S[n] = |\cup_{i=1}^z \mathcal{M}_i| = M
$
\end{center}






\subsubsection{Shared State Notation}
\begin{center}
\vspace{1mm}
$
s^+ = \{ \mathcal{L},\mathcal{M},y_o\}
$
\\ \vspace{6mm}
$
Sub_i[X] := S_i = \{s_j,...|b_k,...,y_o\}:
$
\\ \vspace{2mm}
$
s_j,b_k \in s^+ \hspace{3mm} \forall s_j,b_k \in S_i
$
\\ \vspace{6mm}
$
s^+ = \{ \mathcal{L}_1| \mathcal{M},y_o\} \cup \{  \mathcal{L}_2| \mathcal{M},y_o\} \cup ... \cup \{  \mathcal{L}_z| \mathcal{M},y_o\} :
$
\\ \vspace{2mm}
$
\mathcal{L}_i \cap \mathcal{L}_j = \emptyset \hspace{2mm} \forall i,j\neq i
$
\end{center}
% Theorem of Polynomial Subfunctions
%\subsection{Theorem of Polynomial Subfunctions}
%Consider solution $s^+$ with polynomial total complexity O[n] containing $z$ subfunctions $Sub_k[X_i]$ k = 1..z
%\vspace{1mm}
%\begin{center}
%$
%X_i = \{x_1,...,x_n,C\}; \hspace{2mm} \hat{X}_i = \{x_1,...,x_{n+1},C\}
%$
%\\ \vspace{2mm}
%$
%s^+ = s^+[X_i] := P :
%$
%\\ \vspace{2mm}
%$
%(P \lbrack X_i \rbrack \rightarrow y_{o} == a_{o} \hspace{3mm} \forall X_i) \hspace{2mm} \cap \hspace{2mm} (P[\hat{X}_i] \supseteq P[X_i] \hspace{3mm} \forall X_i,\hat{X}_i)
%$
%\\ \vspace{4mm}
%$
%s^+ = \{ s_1,s_2,...,s_N|b_1,b_2,...,b_M,y_o\} = \{ s_1,s_2,...,s_{O_T \lbrack n \rbrack }, b_1, b_2,...,b_{O_S \lbrack n \rbrack},y_o \}
%$
%\\ \vspace{2mm}
%$
%= \{ \mathcal{L},\mathcal{M},y_o\}
%$
%\\ \vspace{6mm}
%$
%Sub_h[X_i] := S_h = \{s_j,...|b_k,...,y_o\}:
%$
%\\ \vspace{2mm}
%$
%s_j, b_k \in s^+ \hspace{3mm} \forall s_j,b_k \in S_h
%$
%\\ \vspace{6mm}
%$
%s^+ = Sub_1[X_i] \cup Sub_2[X_i] \cup ... \cup Sub_z[X_i]
%$
%\\ \vspace{6mm}
%$
%O[n] = (\lambda_K n)^K + (\lambda_{K-1} n)^{K-1}... + \lambda_1 n + C \hspace{3mm} \forall n
%$
%\\ \vspace{2mm}
%$
%O[n] = O_{T_1}[n] + O_{T_2}[n] + ... + O_{T_z}[n] + |O_{S_1}[n] \cup O_{S_2}[n] \cup  ... \cup O_{S_z}[n]|
%$
%\\ \vspace{2mm}
%$
%= O_{T_1}[n] + O_{T_2}[n] + ... + O_{T_z}[n] + O_S[n]
%$
%\end{center}
%\vspace{6mm}
%By property of polynomial complexity
%\begin{center}
%\vspace{1mm}
%$
%limit_{n \rightarrow \infty} \frac{O[n+1]}{O[n]} = 1
%$
%\\ \vspace{2mm}
%$
%limit_{n \rightarrow \infty} \frac{O_{T_1}[n+1] + O_{T_2}[n+1] + ... + O_{T_z}[n+1] + O_S[n+1]}{O_{T_1}[n] + O_{T_2}[n] + ... + O_{T_z}[n] + O_S[n]]} = 1
%$
%\\ \vspace{2mm}
%$
%limit_{n \rightarrow \infty} \frac{O_h[n+1]}{O_h[n]} \leq 1 \hspace{2mm} \forall h
%$
%\end{center}



\newpage

\section{Polynomial Solution Subfunction Properties}

\subsection{Restate Definition of Subfunction}
\begin{center}
$
X_i = \{x_1,...,x_n,C\}; \hspace{2mm} \hat{X}_i = \{x_1,...,x_{n+1},C\}
$
\\ \vspace{2mm}
$
s^+ = s^+[X_i] := P :
$
\\ \vspace{2mm}
$
(P \lbrack X_i \rbrack \rightarrow y_{o} == a_{o} \hspace{3mm} \forall X_i) \hspace{2mm} \cap \hspace{2mm} (P[\hat{X}_i] \supseteq P[X_i] \hspace{3mm} \forall X_i,\hat{X}_i)
$
\\ \vspace{4mm}
$
s^+ = \{ s_1,s_2,...,s_N|b_1,b_2,...,b_M,y_o\} = \{ s_1,s_2,...,s_{O_T \lbrack n \rbrack }, b_1, b_2,...,b_{O_S \lbrack n \rbrack},y_o \}
$
\\ \vspace{2mm}
$
= \{ \mathcal{L},\mathcal{M},y_o\}
$
\\ \vspace{6mm}
$
Sub[X_i] := S = \{s_j,...|b_k,...,y_o\}:
$
\\ \vspace{2mm}
$
s_j,b_k \in s^+ \hspace{3mm} \forall s_j,b_k \in S
$
\end{center}


\subsection{Property of Polynomial Solution Subfunctions}
Let
\begin{center}
$
D \in \mathbb{P}
$
\\ \vspace{5mm}
$
X_i = \{x_1,...,x_n,C\}; \hspace{2mm} \hat{X}_i = \{x_1,...,x_{n+1},C\}
$
\\ \vspace{2mm}
$
s^+ = s^+[X_i] := P :
$
\\ \vspace{2mm}
$
(P \lbrack X_i \rbrack \rightarrow y_{o} == a_{o} \hspace{3mm} \forall X_i) \hspace{2mm} \cap \hspace{2mm} (P[\hat{X}_i] \supseteq P[X_i] \hspace{3mm} \forall X_i,\hat{X}_i)
$
\\ \vspace{4mm}
$
\exists K,C,\lambda_1...\lambda_K \hspace{2mm} :
$
\\ \vspace{2mm}
$
O[n] = (\lambda_K n)^K + (\lambda_{K-1} n)^{K-1}... + \lambda_1 n + C \hspace{3mm} \forall n
$
\\ \vspace{5mm}
$
s^+ = Sub_1[X_i] \cup Sub_2[X_i] \cup ... \cup Sub_z[X_i]
$
\\ \vspace{6mm}
$
limit_{n \rightarrow \infty} \frac{O[n+1]}{O[n]} = 1
$
\\ \vspace{2mm}
$
= limit_{n \rightarrow \infty} \frac{O^1_T[n+1] + O^2_T[n+1] + ... + O^z_T[n+1] + O_S[n+1]}{O^1_T[n] + O^2_T[n] + ... + O^z_T[n] + O_S[n]}
$
\\ \vspace{4mm}
$
= limit_{n \rightarrow \infty} \frac{O^1_T[n] + O^2_T[n] + ... + O^z_T[n] + O_S[n] + f^1_{T_{n+1}}[n] + f^2_{T_{n+1}}[n+1] + ... + f^z_{T_{n+1}}[n] + f_{S_{n+1}}[n]}{O^1_T[n] + O^2_T[n] + ... + O^z_T[n] + O_S[n]}
$
\\ \vspace{4mm}
$
= limit_{n \rightarrow \infty} 1 + \frac{f^1_{T_{n+1}}[n] + f^2_{T_{n+1}}[n+1] + ... + f^z_{T_{n+1}}[n] + f_{S_{n+1}}[n]}{O^1_T[n] + O^2_T[n] + ... + O^z_T[n] + O_S[n]} =1
$
\\ \vspace{4mm}
$
\Rightarrow limit_{n \rightarrow \infty} \frac{f^1_{T_{n+1}}[n] + f^2_{T_{n+1}}[n+1] + ... + f^z_{T_{n+1}}[n] + f_{S_{n+1}}[n]}{O^1_T[n] + O^2_T[n] + ... + O^z_T[n] + O_S[n]} = 0^*
$
\\ \vspace{3mm}
$
\Rightarrow  limit_{n \rightarrow \infty} \frac{f^i_{T_{n+1}}[n] + f_{S_{n+1}}[n]}{O^1_T[n] + O^2_T[n] + ... + O^z_T[n] + O_S[n]} = 0 \hspace{2mm} \forall i
$
\\ \vspace{2mm}
$
limit_{n \rightarrow \infty} \frac{f^i_{n+1}[n]}{O[n]} = 0 \hspace{2mm} \forall i
$
%$
%s^+ = Sub_1[X_i] \cup Sub_2[X_i] \cup ... \cup Sub_z[X_i]
%$
\end{center}
$^*$ O[n] is a positive, non-decreasing function




% N Sum Problem
\newpage
\section{Sum to N Problem with 2 integers}
% Zero Order Space Complexity Solution
\subsection{State formal definition of Sum to N : $x_i + x_j == N$}
\begin{center}
\vspace{1.5mm}
$
X_i = \{x_1,...,x_n,N\}
$
\\ \vspace{2mm}
$
D := f \lbrack X_i \rbrack \rightarrow a_{o} \in \{\mathbb{T}, \mathbb{F}\} \hspace{3mm} \forall X_i
$
\\ \vspace{2mm}
$
s^+ = P :
$
\\ \vspace{2mm}
$
(P \lbrack X_i \rbrack \rightarrow y_{o} == a_{o} \hspace{3mm} \forall X_i) \hspace{2mm} \cap \hspace{2mm} (P[\hat{X}_i] \supseteq P[X_i] \hspace{3mm} \forall X_i,\hat{X}_i)
$
\\ \vspace{2mm}
$
s^+ = \{ s_1,s_2,...,s_{O_T \lbrack n \rbrack }, b_1, b_2,...,b_{O_S \lbrack n \rbrack},y_o \} = \{ \mathcal{L},\mathcal{M},y_o\}
$
\\ \vspace{6mm}
$
D = f[X_i] = \exists x_j,x_k \in X_i : x_j + x_k == N
$
\end{center}



\subsection{Express a formal solution : $O_S[n] \sim n^0$}
\begin{center}
\vspace{1.5mm}
$
s^+ = \{ s_1,s_2,...,s_{O_T \lbrack n \rbrack }, b_1, b_2,...,b_{O_S \lbrack n \rbrack},y_o \} = \{ \mathcal{L},\mathcal{M},y_o\}
$
\\ \vspace{2mm}
$
s_1 = y_o \leftarrow \mathbb{F};
$
\\ \vspace{2mm}
$
\forall i,j > i
$
\\ \vspace{2mm}
$
s_2,s_3,s_8,s_9,...,s_{3ij-4},s_{3ij-3}...,s_{3n(n-1)-4},s_{3n(n-1)-3} =  b_1 \leftarrow x_i + x_j 
$
\\ \vspace{2mm}
$
s_4,s_5,s_{10},s_{11},...,s_{3ij-2},s_{3ij-1}...,s_{3n(n-1)-2},s_{3n(n-1)-1} = b_1 \leftarrow b_1 == N
$
\\ \vspace{2mm}
$
s_6,s_7,s_{12},s_{13}...,s_{3ij},s_{3ij+1}...,s_{3n(n-1)},s_{3n(n-1)+1} = y_o \leftarrow y_o \lor b_1
$
\\ \vspace{2mm}
$
s^+ = \{y_o \leftarrow \mathbb{F},y_o \leftarrow y_o \lor (x_i + x_j == N) \hspace{3mm} \forall i,j > i \hspace{1mm}| \hspace{1mm} b_1,y_o\}
$
\end{center}



\subsection{Show $s^+$ satisfies the subfunction condition of solutions: $P[\hat{X}_i] \supseteq P[X_i] \hspace{3mm} \forall \hat{X}_i,X_i$}
\begin{center}
\vspace{2mm}
$
X_i = \{x_1,...,x_n,N\}; \hspace{2mm} \hat{X}_i = \{x_1,...,x_n,x_{n+1},N\}
$
\\ \vspace{2mm}
$
s^+ = \{ s_1,s_2,...,s_{O_T \lbrack n \rbrack }, b_1, b_2,...,b_{O_S \lbrack n \rbrack},y_o \} = \{ \mathcal{L},\mathcal{M},y_o\}
$
\\ \vspace{2mm}
$
s_{n+1}^+ = s^+ \cup \hat{s}^+ 
$
\\ \vspace{2mm}
$
s_1 = y_o \leftarrow \mathbb{F};
$
\\ \vspace{2mm}
$
\forall i,j > i
$
\\ \vspace{2mm}
$
s_2,s_3,s_8,s_9,...,s_{3ij-4},s_{3ij-3}...,s_{3n(n-1)-4},s_{3n(n-1)-3} =  b_1 \leftarrow x_i + x_j 
$
\\ \vspace{2mm}
$
s_4,s_5,s_{10},s_{11},...,s_{3ij-2},s_{3ij-1}...,s_{3n(n-1)-2},s_{3n(n-1)-1} = b_1 \leftarrow b_1 == N
$
\\ \vspace{2mm}
$
s_6,s_7,s_{12},s_{13}...,s_{3ij},s_{3ij+1}...,s_{3n(n-1)},s_{3n(n-1)+1} = y_o \leftarrow y_o \lor b_1
$
\\ \vspace{2mm}
$
\forall k
$
\\ \vspace{2mm}
$
s... = b_1 \leftarrow x_k + x_{n+1}
$
\\ \vspace{2mm}
$
s... = b_1 \leftarrow b_1 == N
$
\\ \vspace{2mm}
$
s... = y_o \leftarrow y_o \lor b_1
$
\\ \vspace{8mm}
$
s^+ = \{y_o \leftarrow \mathbb{F},y_o \leftarrow y_o \lor (x_i + x_j == N) \hspace{3mm} \forall i,j > i \hspace{1mm}| \hspace{1mm} b_1,y_o\}
$
\\ \vspace{4mm}
$
\hat{s}^+ = \{y_o \leftarrow y_o \lor (x_k + x_{n+1} == N) \hspace{2mm} \forall k < n+1 | \hspace{1mm} b_1,y_o \}
$
\\ \vspace{4mm}
$
s^+_{n+1} = \{y_o \leftarrow \mathbb{F},y_o \leftarrow y_o \lor (x_i + x_j == N) \hspace{3mm} \forall i,j > i \hspace{1mm}| \hspace{1mm} b_1,y_o\} \hspace{2mm}\cup 
$
\\ \vspace{3mm}
$
\{y_o \leftarrow y_o \lor (x_k + x_{n+1} == N) \hspace{2mm} \forall k < n+1 | \hspace{1mm} b_1,y_o \}
$
\\ \vspace{6mm}
$
s_{n+1}^+ = P[\hat{X}_i] \supseteq P[X_i] = s^+
$
\end{center}











\subsection{Determine $O[n], O_S[n], O_T[n],\hat{O}[n],\hat{O}_T[n],\hat{O}_S[n]$ for the above solution}
\begin{center}
$
O_S[n] = |y_o| + |b_1| = 2
$
\\ \vspace{2mm}
$
O_T[n] = 3n(n-1) + 1 = 3n(n-1) -1 + O_S[n]
$
\\ \vspace{2mm}
$
O[n] = 3n(n-1) + 3 = 3n^2 -3n + 3
$
\\ \vspace{4mm}
$
\hat{O}_S[n] = 0
$
\\ \vspace{2mm}
$
\hat{O}_T[n] = 6n
$
\\ \vspace{2mm}
$
\hat{O}[n] = \hat{O}_S[n] + \hat{O}_T[n]
$
\end{center}


\subsection{Verify $O[n+1]$ = $O[n]$ + $\hat{O}[n]$}
\begin{center}
$
O[n+1] = O[n] + \hat{O}[n]
$
\\ \vspace{2mm}
$
3(n+1)^2 -3(n+1) + 3 =  3n^2 -3n + 3 + 6n
$
\\ \vspace{2mm}
$
3n^2 + 6n + 3 - 3n -3 + 3 = 3n^2 + 3n + 3
$
\\ \vspace{2mm}
$
3n^2 + 3n+ 3 = 3n^2 + 3n + 3
$
\end{center}







\subsection{Show $s^+$ has Polynomial Complexity by the definition of Total Polynomial Complexity}
\begin{center}
\vspace{2mm}
$
O[n] = 3n^2 -3n + 3
$
\end{center}

\subsection{Show $s^+$ has Polynomial Complexity by showing limit$_{n \rightarrow \infty}\frac{O[n+1]}{O[n]}$ = 1}
\begin{center}
$
limit_{n \rightarrow \infty}\frac{O[n+1]}{O[n]} =
$
\\ \vspace{2mm}
$
limit_{n \rightarrow \infty}\frac{3n^2 + 3n + 3}{3n^2 - 3n + 3} =
$
\\ \vspace{2mm}
$
limit_{n \rightarrow \infty}(\frac{3n^2 - 3n + 3}{3n^2 - 3n + 3} + \frac{6n}{3n^2 - 3n + 3}) =
$
\\ \vspace{2mm}
$
limit_{n \rightarrow \infty}(1 + \frac{6n }{3n^2 - 3n + 3}) = 1
$
\end{center}













% Proof of Non-Polynomial Problems Traveling Salesman
\newpage
\section*{Traveling Salesman Problem of Dimension 2}
\section{Proof of the existence of $\mathcal{N}$}

\subsection{The Traveling Salesman Problem of Dimension 2}
English description

\subsection{Formal Definition}
\begin{center}
$
X_i = \{l_1,l_2,...,l_n,C\}
$
\\ \vspace{5mm}
$
l_i = \{x_i,y_i\} \hspace{2mm} \forall i
$
\\ \vspace{2mm}
$l_i$ denotes the 2D coordinates of location i
\\ \vspace{5mm}
$
C = \{d_{proposed},p_{decimal}\}
$
\\ \vspace{2mm}
$
d_{proposed}$ denotes the suggested shortest distance
\\ \vspace{1.5mm}
$
p_{decimal}$ is the decimal precision
\\ \vspace{5mm}
$
L\lbrack l_i,l_j \rbrack := \sqrt{(y_j - y_i)^2 + (x_j - x_i)^2}
$
\\ \vspace{2mm}
Let $L\lbrack l_i,l_j \rbrack$ denote the distance between location $l_i$ and $l_j$
\\ \vspace{5mm}
$
\tilde{L} \lbrack l_i,l_j \rbrack :=d_{trunc} : -p_{decimal} < d_{trunc} - L\lbrack l_i,l_j \rbrack < p_{decimal}
$
\\ \vspace{2mm}
Let $\tilde{L}\lbrack l_i,l_j \rbrack$ denote a truncated decimal representation of $L\lbrack l_i,l_j \rbrack$
\\ \vspace{5mm}
$
R_i := \{r_1,r_2,...,r_n,r_1\} \hspace{1.5mm} : \hspace{1.5mm} r_i \in X_i \hspace{2mm} \forall i; \hspace{2mm} r_i \neq r_j
$
\\ \vspace{2mm}
Let $R_i$ denote route $i$
\\ \vspace{5mm}
$
L_{Total}\lbrack R_i \rbrack := (\sum_{i=1}^{n-1} \tilde{L}\lbrack r_i,r_{i+1}\rbrack) + \tilde{L}\lbrack r_n,r_1\rbrack
$
\\ \vspace{2mm}
Let $L_{Total}\lbrack R_i \rbrack$ denote the sum of truncated lengths of route $R_i$
\\ \vspace{5mm}
$
D := f \lbrack X_i \rbrack \rightarrow a_o \in \{\mathbb{T},\mathbb{F}\} \hspace{2mm} \forall X_i
$
\\ \vspace{2mm}
$
a_o =
$
\\ \vspace{2mm}
$
(\exists R_k : L_{total}\lbrack R_k\rbrack == d_{proposed}) \cap (\not \exists R_j :  L_{total}\lbrack R_j \rbrack < d_{proposed})
$
\end{center}





\newpage
\section*{Traveling Salesman Problem of Dimension 2}
\subsection{Define subpath, subpath distance, subpath storage}
\vspace{2mm}
$\tilde{L} \lbrack l_i,l_j \rbrack$ denotes "the distance of a subpath of length 1"
\begin{center}
$
\tilde{L} \lbrack l_i,l_j \rbrack :=d_{trunc} : -p_{decimal} < d_{trunc} - L\lbrack l_i,l_j \rbrack < p_{decimal} 
$
\\ \vspace{2mm}
$
= abs(d_{trunc} - L\lbrack l_i,l_j \rbrack) < p_{decimal} 
$
\end{center}
\vspace{4mm}
$
\tilde{R}$ denotes a subpath of length k
\begin{center}
$
\tilde{R} = \{\tilde{r}_1,\tilde{r}_2,...,\tilde{r}_k\} \hspace{1.5mm} : \hspace{1.5mm} \tilde{r}_i \in X_i \hspace{2mm} \forall i, r_i \neq r_j
$
\end{center}
\vspace{4mm}
$\tilde{L}_k\lbrack \tilde{R} \rbrack$ denotes "the distance of a subpath of length k"
\begin{center}
$
\tilde{L}_k\lbrack \tilde{R} \rbrack := \sum_{i=1}^{k} \tilde{L}\lbrack \tilde{r},\tilde{r}_{i+1}\rbrack
$ 
\end{center}
\vspace{4mm}
Let $\mathcal{M}_1$ denote the memory reserved for subpaths distances of length 1
\begin{center}
$
\mathcal{M}_1= \{\hat{b}_{1;1},\hat{b}_{1;2},\hat{b}_{1;3},...,\hat{b}_{startindex;finishindex},...,\hat{b}_{n-1;n}\}^*
$
\\ \vspace{2mm}
$
\mathcal{M} \supseteq \mathcal{M}_1
$
\end{center}
\vspace{3mm}* Note $\hat{b}_{i;j} = \hat{b}_{j;i}$\\
$\sqrt{(y_j - y_i)^2 + (x_j - x_i)^2} = \sqrt{(y_i - y_j)^2 + (x_i - x_j)^2}$



% Traveling Salesman First Order Space Complexity Solution
\subsection{Define the following functions}
\subsubsection{$sqrt\lbrack x,p_{decimal} \rbrack = \sqrt{x}$\hspace{2mm}[1]}
\subsubsection{$pow\lbrack x,2,p_{decimal} \rbrack = x^2$ \hspace{1mm}[2]}


\subsection{Define the following subfunctions}
\subsubsection{loadM1Subpaths$\lbrack X\rbrack$}
\vspace{1mm}
// Computes all subpaths of length 1 and stores in $\mathcal{M}_1 = \{\hat{b}_{1;1},\hat{b}_{1,;2},...,\hat{b}_{n-1;n}\}$
\begin{center}
\vspace{2mm}
$
// X_i = \{l_1,l_2,...,l_n,C\}
$
\\ \vspace{2mm}
$
//\hspace{.5mm} l_i = \{x_i,y_i\} \hspace{2mm} \forall i
$
\\ \vspace{5mm}
$
// \mathcal{M} = \{b_1,b_2,...,b_M,\hat{b}_{1;1},\hat{b}_{1,;2},...,\hat{b}_{n-1;n},y_o\} = \{b_1,b_2,...,b_M,\mathcal{M}_1,y_o\} = \{\mathcal{M},\mathcal{M}_1,y_o\}
$
\\ \vspace{5mm}
$
\forall i,j > i
$
\\ \vspace{2mm}
$
b_3 \leftarrow y_i - y_j
$
\\ \vspace{2mm}
$
b_4 \leftarrow x_i - x_j
$
\\ \vspace{2mm}
$
b_3 \leftarrow b_3^2
$
\\ \vspace{2mm}
$
b_4 \leftarrow b_4^2
$
\\ \vspace{2mm}
$
b_3 \leftarrow b_3 + b_4
$
\\ \vspace{2mm}
$
\hat{b}_{i;j}\leftarrow \sqrt{b_3}^*
$
\end{center}
\vspace{4mm}
$^* \hat{b}_{i;j} = \tilde{L}\lbrack l_i,l_j \rbrack$



\subsubsection{computeAllRoutes$\lbrack X \rbrack$}
\vspace{1mm} // Computes all complete routes, checks for a route == $d_{proposed}$, sets $y_o$ to false if the current route is shorter than $d_{proposed}$
\begin{center}
$
\forall i,j \neq i, k \neq i,j ...\hspace{.5mm},q \neq i,j,...,m
$
\\ \vspace{2mm}
$
b_3 \leftarrow \hat{b}_{1;j}+ \hat{b}_{j;k}
$
\\ \vspace{2mm}
$
b_3 \leftarrow b_3 + \hat{b}_{k;l}
$
\\ \vspace{2mm}
$
...
$
\\ \vspace{2mm}
$
b_3 \leftarrow b_3 + \hat{b}_{m;q}
$
\\ \vspace{2mm}
$
b_3 \leftarrow b_3 + \hat{b}_{q;1}
$
\\ \vspace{2mm}
$
b_4 \leftarrow b_3 == b_2
$
\\ \vspace{2mm}
$
b_1 \leftarrow b_1 \lor b_4
$
\\ \vspace{2mm}
$
b_4 \leftarrow b_2 \leq b_3
$
\\ \vspace{2mm}
$
y_o \leftarrow y_o \land b_4
$
\end{center}




\subsection{Express a solution using subfunctions, storing subpaths of length 1 in memory}
\begin{center}
\vspace{4mm}
// $d_{proposed}$ is the shortest path\\
$y_o \leftarrow \mathbb{T}$
\\ \vspace{3mm}
//$d_{proposed}$ exists as a total path length\\
$ b_1 \leftarrow \mathbb{F}$
\\ \vspace{3mm}
// shortest path register\\
$ b_2 \leftarrow d_{proposed}$
\\ \vspace{3mm}
$loadM1Subpaths\lbrack X\rbrack$
\\ \vspace{1mm}
$computeAllRoutes\lbrack X \rbrack$
\end{center}












\subsection{Show each subfunction satisfies the subfunction condition of solutions $: P[\hat{X}_i] \supseteq P[X_i] \hspace{3mm} \forall \hat{X}_i,X_i, \hspace{2mm} \hat{X}_i \supseteq X_i$}
Let
\begin{center}
$
\mathcal{M}_0 = \{b_1,b_2,b_3,b_4,y_o\}
$
\\ \vspace{2mm}
$
\mathcal{M}_1 = \{\hat{b}_{1;1},\hat{b}_{1,;2},...,\hat{b}_{n-1;n}\} 
$
\end{center}

\subsubsection{$loadM1Subpaths\lbrack X\rbrack \rightarrow \mathcal{M}_1$}
Let
\begin{center}
$
// X = \{l_1,l_2,...,l_n,C\}; \hspace{2mm} \hat{X} = \{l_1,l_2,...,l_n,l_{n+1},C\}
$
\\ \vspace{4mm}
$
loadM1Subpaths\lbrack X, \mathcal{M}\rbrack \rightarrow \mathcal{M}_1 = Sub_1\lbrack X, \mathcal{M} \rbrack \rightarrow \mathcal{M}_1
$
\\ \vspace{8mm}
$
Sub_1 \lbrack X, \mathcal{M} \rbrack = \{ \mathcal{L},\mathcal{M}\}
$
\\ \vspace{2mm}
$
= \{ \hat{b}_{i;j} \leftarrow \tilde{L}\lbrack l_i,l_j \rbrack \hspace{2mm} \forall i,j>i | b_3,b_4,\hat{b}_{1;1},\hat{b}_{1;2},...,\hat{b}_{n-1;n}\}
$
\\ \vspace{3mm}
$
Sub_1 \lbrack X_i, \mathcal{M} \rbrack = \{ \hat{b}_{i;j} \leftarrow \tilde{L}\lbrack l_i,l_j \rbrack \hspace{2mm} \forall i,j>i | b_3,b_4,\mathcal{M}_1\}
$
\\ \vspace{8mm}
$
Sub_1 \lbrack \hat{X}, \mathcal{M} \rbrack = \{ \hat{\mathcal{L}},\hat{\mathcal{M}}\}
$
\\ \vspace{2mm}
$
= \{ \hat{b}_{i;j} \leftarrow \tilde{L}\lbrack l_i,l_j \rbrack \hspace{2mm} \forall i,j>i | b_3,b_4,\hat{b}_{1;1},\hat{b}_{1;2},...,\hat{b}_{n;n+1}\}
$
\\ \vspace{2mm}
$
= \{ \mathcal{L}, \hat{b}_{i;j} \leftarrow \tilde{L}\lbrack l_i,l_j \rbrack \hspace{2mm} \forall i,j=n+1 | \mathcal{M}, \hat{b}_{1;n+1},\hat{b}_{2;n+1},...,\hat{b}_{n;n+1}\}
$
\\ \vspace{2mm}
$
Sub_1 \lbrack \hat{X}, \mathcal{M} \rbrack = \{ \mathcal{L}, \mathcal{L}_{n+1} | \mathcal{M}, \mathcal{M}_{n+1}\}
$
\\ \vspace{3mm}
$
Sub_1 \lbrack \hat{X}, \mathcal{M} \rbrack = \{ \mathcal{L}, \mathcal{L}_{n+1} | \mathcal{M}, \mathcal{M}_{n+1}\} \supseteq \{ \mathcal{L} | \mathcal{M}\} = Sub_1 \lbrack X, \mathcal{M} \rbrack 
$
\end{center}




\subsubsection{$computeAllRoutes\lbrack X\rbrack$}
Let
\begin{center}
$
computeAllRoutes\lbrack X\rbrack = Sub_2\lbrack X \rbrack
$
\\ \vspace{2mm}
$
Sub_2 \lbrack X \rbrack = \{ \mathcal{L},\mathcal{M},y_o\}
$
\\ \vspace{2mm}
$
= \{\hat{b}_{1;i_2}+\hat{b}_{i_2;i_3}+\hat{b}_{i_3;i_4}+...+\hat{b}_{i_{n};1} \hspace{2mm} \forall i_2,i_3 \neq i_2, i_4 \neq i_2,i_3 ... i_n \neq i_2,i_3...,i_{n-1} \hspace{2mm}
$
\\ \vspace{1mm}
$
| b_1,b_2,b_3,b_4,\mathcal{M}_1,y_o \}
$
\\ \vspace{6mm}
$
Sub_2 \lbrack\hat{X} \rbrack = \{ \hat{\mathcal{L}},\hat{\mathcal{M}},y_o\}
$
\\ \vspace{2mm}
$
= \{\hat{b}_{1;i_2}+\hat{b}_{i_2;i_3}+\hat{b}_{i_3;i_4}+...+\hat{b}_{i_{n+1};1} \hspace{2mm} \forall i_2,i_3 \neq i_2, i_4 \neq i_2,i_3 ... i_{n+1} \neq i_2,i_3...,i_{n} \hspace{2mm}
$
\\ \vspace{1mm}
$
| \mathcal{M},\mathcal{M}_{n+1},y_o \}
$
\end{center}
\vspace{6mm}Let
\begin{center}
$
insert\_subpath\lbrack \mathcal{L} \rbrack =
$
\\ \vspace{8mm}
$
Sub_2 \lbrack\hat{X} \rbrack =  \{ insert\_subpath\lbrack \mathcal{L},\hat{b}_{i_{n+1};j},j \rbrack \hspace{2mm} \forall j \neq n+1 | \mathcal{M},\mathcal{M}_{n+1},y_o \}
$
\end{center}




\subsubsection{Show the overall solution storing subpaths of length 1 satisfies the subfunction condition of solutions $: P[\hat{X}_i] \supseteq P[X_i] \hspace{3mm} \forall \hat{X}_i,X_i$}





\subsection{Express $O[n]$ in terms of subfunction complexities}
\begin{center}
$
O_{sub1}\lbrack n \rbrack = O_{T_{sub1}}\lbrack n \rbrack + O_{S_{sub1}}\lbrack n \rbrack
$
\\ \vspace{2mm}
$
O_{sub2}\lbrack n \rbrack = O_{T_{sub2}}\lbrack n \rbrack + O_{S_{sub2}}\lbrack n \rbrack
$
\\ \vspace{2mm}
$
O\lbrack n \rbrack = O_{sub1}\lbrack n \rbrack + O_{sub2}\lbrack n \rbrack + 3
$
\end{center}







\subsection{$\bold{Sub_+[X]}$}
Let $Sub_+[X]$ denote a subfunction that adds all subpaths of length 1\\
Let $O_+[n]$ denote the total complexity of subfunction $Sub_+[X]$

\subsubsection{Find an expression for $O_+[n]$ := the number of $\tilde{L} \lbrack l_i,l_j \rbrack$ + $\tilde{L} \lbrack l_j,l_k \rbrack$  length 1 subpath additions}
\begin{center}
\vspace{2mm}
$
O_+[n] = (\sum_{i=1}^n 1)\frac{(n P (n-1)}{2}
$
\\ \vspace{2mm}
$
O_+[n] = \frac{n(n-1)!}{2}
$
\\ \vspace{2mm}
$
O_+[n] = \frac{n!}{2}
$
\end{center}






\subsection{Prove $Sub_+[X]$ is a subfunction of all $s^+$ by contradiction}
suppose not all subpaths are considered\\
there could exist subpath resulting in an incorrect solution\\
contradicts definition of solution\\



\subsection{Show the solution storing subpaths of length 1 contains $\bold{Sub_+[X]}$}
\begin{center}
\vspace{4mm}
// $d_{proposed}$ is the shortest path\\
$y_o \leftarrow \mathbb{T}$
\\ \vspace{3mm}
//$d_{proposed}$ exists as a total path length\\
$ b_1 \leftarrow \mathbb{F}$
\\ \vspace{3mm}
// shortest path register\\
$ b_2 \leftarrow d_{proposed}$
\\ \vspace{3mm}
$loadM1Subpaths\lbrack X\rbrack$
\\ \vspace{1mm}
$computeAllRoutes\lbrack X \rbrack$
\end{center}








\subsection{Express $O[n]$ in terms of subfunction complexities including $O_+[n]$ as a subfunction complexity}
\begin{center}
$
O\lbrack n \rbrack = O_{sub1}\lbrack n \rbrack + O_{sub2}\lbrack n \rbrack + 3
$
\\ \vspace{2mm}
$
 O_{sub2}\lbrack n \rbrack = O_+[n] + 8\frac{nP(n-1)}{2}
$
\\ \vspace{2mm}
$
 O_{sub2}\lbrack n \rbrack = O_+[n] + 8\frac{(n-1)!}{2}
$
\\ \vspace{2mm}
$
O\lbrack n \rbrack = O_{sub1}\lbrack n \rbrack + O_+[n] + 8\frac{(n-1)!}{2} + 3
$
\end{center}






\subsection{Show $limit_{n \rightarrow \infty}\frac{O_+[n+1]}{O_+[n]}$ diverges}
\begin{center}
$
limit_{n \rightarrow \infty}\frac{O_+[n+1]}{O_+[n]}
$
\\ \vspace{2mm}
$
= limit_{n \rightarrow \infty}\frac{(n+1)!}{2} \frac{2}{n!}
$
\\ \vspace{2mm}
$
= limit_{n \rightarrow \infty}\hspace{1mm} n
$
\end{center}
There does not exist ... \\
therefore $limit_{n \rightarrow \infty}\frac{O_+[n+1]}{O_+[n]}$ diverges





\subsection{Prove D is not in $\mathbb{P}$ by the definition of Polynomial Complexity}
Let
\begin{center}
$
Sub_i[X]$ denote all subfunctions of a solution $s^+$ except $Sub_+[X]
$
\\ \vspace{4mm}
$
O[n] = O_+[n] + \sum_{i=1}^z O_{i_T}[n]
$
\\ \vspace{3mm}
$
O[n] = \frac{n!}{2} + \sum_{i=1}^z O_{i_T}[n]
$
\end{center}
\vspace{6mm}
Assume O[n] satisfies the condition of Polynomial total complexity for $n = \hat{n}$ Let n = K!!!
\begin{center}
$\exists K,C,\lambda_1...\lambda_K \hspace{2mm} :$
\\ \vspace{2mm}
$O[\hat{n}] = (\lambda_K \hat{n})^K + (\lambda_{K-1} \hat{n})^{K-1}... + \lambda_1 \hat{n} + C \hspace{3mm}
$
\\ \vspace{2mm}
$
= \frac{\hat{n}!}{2} + \sum_{i=1}^z O_{i_T}[\hat{n}]
$
\end{center}
\vspace{5mm}
Let $n = \hat{n}+1$
\begin{center}
$
O[\hat{n} + 1] =  (\lambda_K \hat{n})^K + (\hat{\lambda_{K-1}} \hat{n})^{K-1}... + \hat{\lambda_1} \hat{n} +\hat{C} \sim (\lambda_K \hat{n})^K 
$
\\ \vspace{2mm}
$
\neq \frac{(\hat{n}+1)!}{2} + \sum_{i=1}^z O_{i_T}[\hat{n}+1] \sim (\lambda_K \hat{n})^{K+1}
$
\end{center}
\vspace{1mm}
\begin{center}
$
\therefore D \in \mathbb{P}$ is a contradiction
\\ \vspace{3mm}
$
\therefore D \not \in \mathbb{P} \Rightarrow D \in \mathcal{N}
$
\end{center}


\subsection{Prove D is not in $\mathbb{P}$ by limit property of Polynomial Complexity}
Let
\begin{center}
$
Sub_i[X]$ denote all subfunctions of a solution $s^+$ except $Sub_+[X]
$
\\ \vspace{4mm}
$
O[n] = O_+[n] + \sum_{i=1}^z O_{i_T}[n]
$
\\ \vspace{3mm}
$
O[n] = \frac{n!}{2} + \sum_{i=1}^z O_{i_T}[n]
$
\end{center}
\vspace{6mm}
Assume O[n] satisfies the condition of Polynomial total complexity
\begin{center}
$\exists K,C,\lambda_1...\lambda_K \hspace{2mm} :$
\\ \vspace{2mm}
$O[n] = (\lambda_K n)^K + (\lambda_{K-1} n)^{K-1}... + \lambda_1 n + C \hspace{3mm}
$
\\ \vspace{8mm}
$
f_{+_{n+1}}[n] = O_+[n+1] - O_+[n] = \frac{(n+1)!}{2} - \frac{n!}{2}
$
\\ \vspace{2mm}
$
= \frac{n!(n+1 - 1)}{2} = \frac{n*n!}{2}
$
\\ \vspace{8mm}
$
limit_{n \rightarrow \infty} \frac{O[n+1]}{O[n]} = 1
$
\\ \vspace{2mm}
$
limit_{n \rightarrow \infty} \frac{\frac{(n+1)!}{2} + \sum_{i=1}^z O_{i_T}[n+1]}{ (\lambda_K n)^K + (\lambda_{K-1} n)^{K-1}... + \lambda_1 n + C}
$
\\ \vspace{2mm}
$
limit_{n \rightarrow \infty} \frac{\frac{(n+1)!}{2}}{(\lambda_K n)^K + (\lambda_{K-1} n)^{K-1}... + \lambda_1 n + C} +  \frac{\sum_{i=1}^z O_{i_T}[n+1]}{(\lambda_K n)^K + (\lambda_{K-1} n)^{K-1}... + \lambda_1 n + C}
$
\\ \vspace{6mm}
$
limit_{n \rightarrow \infty} \frac{\frac{(n+1)!}{2}}{(\lambda_K n)^K + (\lambda_{K-1} n)^{K-1}... + \lambda_1 n + C}$ diverges
\\ \vspace{4mm} 
$O_{i_T}[n+1]$ is a positive, non-decreasing function
\\ \vspace{2mm}
$
\therefore limit_{n \rightarrow \infty} \frac{O[n+1]}{O[n]}$ diverges
\end{center}
\vspace{1mm}
\begin{center}
$
\therefore D \in \mathbb{P}$ contradicts the limit property of Polynomial solutions
\\ \vspace{3mm}
$
\therefore D \not \in \mathbb{P} \Rightarrow D \in \mathcal{N}
$
\end{center}



%\subsection{Connection to "P $\neq$ NP"}



\newpage
\section*{Citations}
\lbrack1\rbrack \hspace{1mm} $chatgpt$\\
\lbrack2\rbrack \hspace{1mm} $https://stackoverflow.com/questions/3518973/floating-point-exponentiation-without-power-function$\\
\lbrack3\rbrack \hspace{1mm} $https://stackoverflow.com/questions/27086195/linear-index-upper-triangular-matrix$\\
\lbrack4\rbrack \hspace{1mm} $http://www.math.uchicago.edu/~may/VIGRE/VIGRE2011/REUPapers/Riffer-Reinert.pdf$



\end{document}