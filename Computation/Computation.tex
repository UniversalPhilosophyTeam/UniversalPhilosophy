\documentclass[11pt]{article}
\usepackage{amsfonts}
\usepackage[T1]{fontenc}
\usepackage{mathabx,graphicx}

\newcommand{\test}{\circlearrowright}
\def \loop {\ensuremath{\rotatebox[origin=c]{-90}{$\circlearrowright$}}}
\def \nestedloop {\ensuremath{\rotatebox[origin=c]{-90}{$\circlearrowright$}}^n}

\begin{document}

\section*{Ch. 5 Computation}





\section{Memory}

\subsection{Definition}
Define Memory; a set of elements
\begin{center}
$\mathcal{M} := \{b_1,b_2,...,b_M\}$
\end{center}

\subsection{Silicon Computation}
In silicon based computation memory is represented with bits 0/1
\begin{center}
$\mathcal{M} := \{b_1,b_2,...,b_m\}$
\\ \vspace{2mm}
$b_i \in \{0,1\} \hspace{2mm} \forall i$
\end{center}

\section{Logical Operations}
Working list; possibly any set operation/function\\
\\
+\\
-\\
*\\
/\\
exp\\
\loop\\
$\leftarrow$\\
delete\\
remove\\
insert\\
append\\
if\\
==\\
!\\




\section{Program}
\subsection{Logical Instructions}
Define $\mathcal{L}$; an ordered set of logical operations $s_i$
\begin{center}
$
\mathcal{L} := \{ s_1,s_2,...,s_{N}\}
$
\end{center}





\subsection{State |}
Define state; the memory required to perform program P
\begin{center}
$
P := \{ s_1, s_2,...,s_{N} | b_1, b_2,...,b_M\} =
$
\\ \vspace{2mm}
$
\{ s_1, s_2,...,s_{N}, b_1, b_2,...,b_M\}
$
\end{center}





\subsection{Boolean Programs}
Define a boolean program; boolean programs can represent functions with inputs $x_i$ and boolean output $y_o$
\begin{center}
$
X = \{x_1,...,x_n\}
$
\\ \vspace{2mm}
$P = P\lbrack X \rbrack := \{ s_1,s_2,...,s_{N}\hspace{1mm}|\hspace{1mm} b_1, b_2,...,b_M,X_i,y_o\}=$
\\ \vspace{2mm}
$
P\lbrack X \rbrack \rightarrow y_o \in \{ \mathbb{T},\mathbb{F}\}
$
\end{center}





\subsection{Void Programs}
Define a void program; a program with inputs $x_i$ and no output
\begin{center}
$
X = \{x_1,...,x_n\}
$
\\ \vspace{2mm}
$P = P\lbrack X \rbrack := \{ s_1,s_2,...,s_{N}\hspace{1mm}|\hspace{1mm} b_1, b_2,...,b_M,X_i\}$
\end{center}





\subsection{Numerical Programs}
Define a numerical program; a program with inputs $x_i$ and real, rational output $y_o$
\begin{center}
$
X = \{x_1,...,x_n\}
$
\\ \vspace{2mm}
$P = P\lbrack X \rbrack := \{ s_1,s_2,...,s_{N}\hspace{1mm}|\hspace{1mm} b_1, b_2,...,b_M,X_i,y_o\}=$
\\ \vspace{2mm}
$
P\lbrack X \rbrack \rightarrow y_o \in \mathbb{Q} \hspace{2mm} y_o \geq 0
$
\end{center}





\subsection{System Programs}
Naming convention to be formalized; a program that outputs one or more elements
\begin{center}
$
X = \{x_1,...,x_n\}
$
\\ \vspace{2mm}
$P = P\lbrack X \rbrack := \{ s_1,s_2,...,s_{N}\hspace{1mm}|\hspace{1mm} b_1, b_2,...,b_M,X_i,Y_o\}=$
\\ \vspace{2mm}
$
P\lbrack X \rbrack \rightarrow Y_o = \{y_1,y_2,...,y_K\}
$
\end{center}





\subsection{Mathematical Programs}
Define a mathematical program; a program with inputs $x_i$ and numerical output $y_o$
\begin{center}
$
X = \{x_1,...,x_n\}
$
\\ \vspace{2mm}
$P = P\lbrack X \rbrack := \{ s_1,s_2,...,s_{N}\hspace{1mm}|\hspace{1mm} b_1, b_2,...,b_M,X_i,y_o\}=$
\\ \vspace{2mm}
$
P\lbrack X \rbrack \rightarrow y_o \in \mathbb{Q}
$
\end{center}




% No-op; While Loop; For Loop
\newpage
\section{No-op ;}

\subsection{Definition}
\begin{center}
$
; := \emptyset
$
\end{center}

\subsection{Property of No-op}
No-op can be inserted into any set with equality
\begin{center}
$
S= \{s_1,s_2,...,s_N\}
$
\\ \vspace{2mm}
$
S_; = insert \lbrack S,;, i \rbrack
$
\\ \vspace{2mm}
$
S_; = S_1 \hspace{2mm} \forall i
$
\\ \vspace{2mm}
$
|S_;| = |S| \hspace{2mm} \forall i
$
\end{center}


\subsection{Proof}
by definition of magnitude of null = 0 with Set And





\section{For Loop \loop}
\loop \lbrack startindex, endindex, condition\rbrack
\subsection{Definition}








\section{Nested For Loop $\nestedloop$}
\loop \lbrack startindex, endindex, condition1,...condition n\rbrack
\subsection{Definition}








% Decision Problems and Solutions
\newpage
\section{Decision Problems}

\subsection{Definition}
Define decision problem; a function with inputs $x_i$ and boolean output "answer" $a_o$
\begin{center}
$
X_i = \{x_1,...,x_n\}
$
\\ \vspace{2mm}
$
D := f \lbrack X_i \rbrack \rightarrow a_{o} \in \{\mathbb{T}, \mathbb{F}\} \hspace{3mm} \forall X_i
$
\end{center}







\section{Solutions}

\subsection{Definition}
Program P is a solution $s^{+}$ if P outputs answer $a_o$ for all inputs $X_i \hspace{2mm} \forall i$\\
$s^+$ is a function of the number of inputs n
\begin{center}
\vspace{1mm}
$
X_i = \{x_1,...,x_n\}
$
\\ \vspace{2mm}
$
D := f \lbrack X_i \rbrack \rightarrow a_{o} \in \{\mathbb{T}, \mathbb{F}\} \hspace{3mm} \forall X_i
$
\\ \vspace{2mm}
$
s^+ = s^+\lbrack n \rbrack := P \lbrack X_i \rbrack \rightarrow y_{o} : y_o = a_{o} \hspace{3mm} \forall X_i
$
\\ \vspace{2mm}
$
P \lbrack X_i \rbrack = \{ s_1,s_2,...,s_N| b_1, b_2,...,b_M,X_i,y_o\}
$
\\ \vspace{3mm}
$
s^+ = P \lbrack X_i \rbrack = \{ s_1,s_2,...,s_{O_T \lbrack n \rbrack }, b_1, b_2,...,b_{O_S \lbrack n \rbrack},X_i,y_o \} \hspace{3mm} \forall X_i
$
\end{center}





\subsubsection{Property of No-op ;}
No-op ; can be added to any solution $S_i$ and remain a solution for all i anywhere in the order for all j
\begin{center}
$
s^+ = \{ s_1,s_2,...,s_{O_T \lbrack n \rbrack }, b_1, b_2,...,b_{O_S \lbrack n \rbrack},X_i,y_o\}
$
\\ \vspace{2mm}
$
\hat{s}^+ = insert \lbrack s^+,;,k \rbrack
$
\\ \vspace{2mm}
$
\hat{s}^+ = s^+ \hspace{2.5mm} \forall k
$
\end{center}





\subsection{Definition of $S^+$}
Define $S^+$; the set of solutions to decision problem D
\begin{center}
$
X_i = \{x_1,...,x_n\}
$
\\ \vspace{2mm}
$
D := f \lbrack X_i \rbrack \rightarrow a_{o} \in \{\mathbb{T}, \mathbb{F}\} \hspace{3mm} \forall X_i
$
\\ \vspace{2mm}
$
s_j^+ := P \lbrack X_i \rbrack \rightarrow y_{o} : y_o = a_{o} \hspace{3mm} \forall X_i
$
\\ \vspace{2mm}
$
S^+ := \{s^+_j,...\} \hspace{3mm} \forall j
$
\end{center}





\subsection{Definition of Solvable}
Define solvable
\begin{center}
$
X_i = \{x_1,...,x_n\}
$
\\ \vspace{2mm}
$
D := f \lbrack X_i \rbrack \rightarrow a_{o} \in \{\mathbb{T}, \mathbb{F}\} \hspace{3mm} \forall X_i
$
\\ \vspace{2mm}
$solvable : = solvable \lbrack D \rbrack \rightarrow b_o \in \{ \mathbb{T}, \mathbb{F} \} =$
\\ \vspace{2mm}
$\exists s^+ : s^+ = P \lbrack X_i \rbrack \rightarrow y_{o} : y_o = a_{o} \hspace{3mm} \forall X_i$
\end{center}





\section{The set of all Decision Problems $\mathbb{D}$}

\subsection{Definition}
Define the set of decision problems $\mathbb{D}$
\begin{center}
$
X_i = \{x_1,...,x_n\}
$
\\ \vspace{2mm}
$
D_j := f_j \lbrack X_i \rbrack \rightarrow a_{o} \in \{\mathbb{T}, \mathbb{F}\} \hspace{3mm} \forall X_i
$
\\ \vspace{2mm}
$\mathbb{D} := \{D_j,...\} \hspace{3mm} \forall j$
\end{center}





%\section{The Set of All Solutions to Decision Problems $\bold{S^+}$}

%\subsection{Definition}
%Define $\bold{S^+}$ the set of all solutions to decision problems
%\begin{center}
%$
%D_j \in \mathbb{D}
%$
%\\ \vspace{2mm}
%$
%D_j := f_j \lbrack X_i \rbrack \rightarrow a_{o} \in \{\mathbb{T}, \mathbb{F}\} \hspace{3mm} \forall X_i
%$
%\\ \vspace{2mm}
%$
%S^+_j := \{s^+_{1},s^+_{2},...\}
%$
%\\ \vspace{2mm}
%$
%\bold{S^+} := \{S^+_j,...\} \hspace{3mm} \forall j
%$
%\end{center}





\section{Instruction and Memory Notation}
Define $\mathcal{L}$ a set of logical operations\\
Define $\mathcal{M}$ a set of bits "memory"
\begin{center}
$
P \lbrack X_i \rbrack \rightarrow y_o = \{ s_1,s_2,...,s_{O_T \lbrack n \rbrack }, b_1, b_2,...,b_{O_S \lbrack n \rbrack},X_i,y_o \}
$
\\ \vspace{2mm}
$
\mathcal{L} := \{ s_1,s_2,...,s_{O_T \lbrack n \rbrack}\}
$
\\ \vspace{2mm}
$
\mathcal{M} := \{ b_1,b_2,...,b_{O_S \lbrack n \rbrack}\}
$
\\ \vspace{2mm}
$
P \lbrack X_i \rbrack = \{ \mathcal{L},\mathcal{M},X_i,y_o\}
$
\end{center}





















% Complexity; Time Complexity; Space Complexity; Total Complexity Dimension > 1
\section{Complexity}

\subsection{Time Complexity of a Decision Problem $O_T \lbrack n \rbrack$}
Define Time Complexity $O_T [n]$ of Decision Problem $D$ with solution $s^+$
\begin{center}
$
X_i = \{x_1,...,x_n\}
$
\\ \vspace{2mm}
$
D := f \lbrack X_i \rbrack \rightarrow a_{o} \in \{\mathbb{T}, \mathbb{F}\} \hspace{3mm} \forall X_i
$
\\ \vspace{2mm}
$
s^+ := P \lbrack X_i \rbrack \rightarrow y_{o} : y_o = a_{o} \hspace{3mm} \forall X_i = 
$
\\ \vspace{2mm}
$
\{ s_1,s_2,...,s_{O_T \lbrack n \rbrack }, b_1, b_2,...,b_{O_S \lbrack n \rbrack},X_i,y_o \} = \{ \mathcal{L},\mathcal{M},X_i,y_o\}
$
\\ \vspace{3mm}
$
O_T[n] := |\mathcal{L}| = N
$
\end{center}

\subsection{Space Complexity $O_S \lbrack n \rbrack$}
Define Space Complexity $O_S \lbrack n \rbrack$ of Decision Problem $D$ with solution $s^+$
\begin{center}
$
X_i = \{x_1,...,x_n\}
$
\\ \vspace{2mm}
$
D := f \lbrack X_i \rbrack \rightarrow a_{o} \in \{\mathbb{T}, \mathbb{F}\} \hspace{3mm} \forall X_i
$
\\ \vspace{2mm}
$
s^+ := P \lbrack X_i \rbrack \rightarrow y_{o} : y_o = a_{o} \hspace{3mm} \forall X_i =
$
\\ \vspace{2mm}
$
\{ s_1,s_2,...,s_{O_T \lbrack n \rbrack }, b_1, b_2,...,b_{O_S \lbrack n \rbrack},X_i,y_o \} = \{ \mathcal{L},\mathcal{M},X_i,y_o\}
$
\\ \vspace{2mm}
$
O_S[n] := |\mathcal{M}| = M
$
\end{center}





\section{Definition of Complexity}
Define Complexity $O[n]$ as a vector of dimension C
\begin{center}
$
\bold{O}[n] := \hspace{3mm} < O_T [n], O_S [n],O_3[n],O_4[n]...,O_C[n]>
$
\end{center}

\section{Total Complexity}
\begin{center}
$O[n] := O_T[n] + O_S[n] + \sum_{i=3}^{C} O_i[n]$
\end{center}
























% Simple Computational Complexity; Restate Time and Space Complexity; (Simple) Total Complexity
\newpage
\section{Simple Computational Complexity}
The remainder of this chapter assumes simple computational complexity of dimension 2

\subsection{Definition}
Define simple computational complexity of dimension 2
\begin{center}
$
\bold{O}[n] := \hspace{3mm} < O_T [n], O_S [n] >
$
\end{center}





\subsection{Time Complexity}
Restate definition of Time Complexity $O_T[n]$
\begin{center}
$
s^+ = \{ \mathcal{L},\mathcal{M},X_i,y_o\}
$
\\ \vspace{3mm}
$
O_T[n] := |\mathcal{L}| = N
$
\end{center}



\subsection{Space Complexity}
Restate definition of Time Complexity $O_S[n]$
\begin{center}
$
s^+ = \{ \mathcal{L},\mathcal{M},X_i,y_o\}
$
\\ \vspace{2mm}
$
O_S[n] := |\mathcal{M}| = M
$
\end{center}





\subsection{Total Complexity}
\begin{center}
$O[n] := O_T[n] + O_S[n]$
\\ \vspace{2mm}
$= |\mathcal{L}| + |\mathcal{M}| = N + M$
\end{center}




\subsection{Axiom O[n] $\neq$ 0}
\subsection{Theorem $O_T[n] + O_S[n] \neq$ 0}
\subsubsection{Proof}



\subsection{Theorem $O[n] \geq O_T[n]$}
\subsubsection{Proof}

\subsection{Theorem $O[n] \geq O_S[n]$}
\subsubsection{Proof}







% Optimal Complexity; Optimal Solutions; Optimal Time Solution; Optimal Space Solution
\newpage
\section{Optimal Complexity}

\subsection{Definition}
Define Optimal Complexity; the minimum total complexity required to solve a decision problem
\begin{center}
$O_{opt}[n] :=$
\\ \vspace{2mm}
$\not \exists \hat{O} \lbrack n \rbrack : \hat{O} \lbrack n \rbrack < O_{opt}[n] \hspace{2mm} \forall n$
\end{center}

\subsection{Proof of Existence}
Prove the existence of at least one $O_{min}[n]$ by induction/contradiction








\section{Optimal solution}
Define an optimal solution $s_{opt}^+$

\subsection{Definition}
\begin{center}
$
X_i = \{x_1,...,x_n\}
$
\\ \vspace{2mm}
$
D_j := f \lbrack X_i \rbrack \rightarrow a_{o} \in \{\mathbb{T}, \mathbb{F}\} \hspace{3mm} \forall X_i
$
\\ \vspace{2mm}
$
s^+ := P \lbrack X_i \rbrack \rightarrow y_{o} : y_o = a_{o} \hspace{3mm} \forall X_i
$
\\ \vspace{2mm}
$
s_{opt}^+ := s^+ :
$
\\ \vspace{2mm}
$
\not \exists \hat{O} \lbrack n \rbrack < O_{opt}[n] \hspace{3mm} \forall n, \hspace{1mm}  s^+ \in S_j^+
$
\end{center}





\subsection{Optimal Time Complexity Solution}
\begin{center}
$
X_i = \{x_1,...,x_n\}
$
\\ \vspace{2mm}
$
D_j := f \lbrack X_i \rbrack \rightarrow a_{o} \in \{\mathbb{T}, \mathbb{F}\} \hspace{3mm} \forall X_i
$
\\ \vspace{2mm}
$
s^+ := P \lbrack X_i \rbrack \rightarrow y_{o} : y_o = a_{o} \hspace{3mm} \forall X_i = 
$
\\ \vspace{2mm}
$
\{ s_1,s_2,...,s_{O_T \lbrack n \rbrack }, b_1, b_2,...,b_{O_S \lbrack n \rbrack},X_i,y_o \} = \{ \mathcal{L},\mathcal{M},X_i,y_o\}
$
\\ \vspace{3mm}
$
O_T[n] := |\mathcal{L}| = N
$
\\ \vspace{2mm}
$
s_{T}^+ := s^+ :
$
\\ \vspace{2mm}
$
\not \exists \hat{O_T} \lbrack n \rbrack < O_{T}[n] \hspace{3mm} \forall n, \hspace{1mm}  s^+ \in S_j^+
$
\end{center}




\subsection{Optimal Space Complexity Solution}

\begin{center}
$
X_i = \{x_1,...,x_n\}
$
\\ \vspace{2mm}
$
D_j := f \lbrack X_i \rbrack \rightarrow a_{o} \in \{\mathbb{T}, \mathbb{F}\} \hspace{3mm} \forall X_i
$
\\ \vspace{2mm}
$
s^+ := P \lbrack X_i \rbrack \rightarrow y_{o} : y_o = a_{o} \hspace{3mm} \forall X_i = 
$
\\ \vspace{2mm}
$
\{ s_1,s_2,...,s_{O_T \lbrack n \rbrack }, b_1, b_2,...,b_{O_S \lbrack n \rbrack},X_i,y_o \} = \{ \mathcal{L},\mathcal{M},X_i,y_o\}
$
\\ \vspace{3mm}
$
O_S[n] := |\mathcal{M}| = M
$
\\ \vspace{2mm}
$
s_{S}^+ := s^+ :
$
\\ \vspace{2mm}
$
\not \exists \hat{O_S} \lbrack n \rbrack < O_{S}[n] \hspace{3mm} \forall n, \hspace{1mm}  s^+ \in S_j^+
$
\end{center}






















% Total Polynomial Complexity; Set of all Polynomial Problems; Polynomial Order of Complexity
\newpage
\section{Polynomial Complexity}

\subsection{Definition}
Decision problem $D$ with solution $s^+$ has (optimal) total complexity $O[n]$ bounded by polynomial complexity if
\begin{center}
$\exists K,C,\lambda_1...\lambda_K \hspace{2mm} :$
\\ \vspace{2mm}
$O_{opt}[n] < (\lambda_K n)^K + (\lambda_{K-1} n)^{K-1}... + \lambda_1 n + C \hspace{3mm} \forall n$
\end{center}





\subsection{Polynomial Problems}
Define $\mathbb{P}$, the set of Decision Problems that can be solved with Polynomial Complexity
\begin{center}
$
\mathbb{P} := \{D_1,D_2,...\} : 
$
\\
$
\exists K,C,\lambda_1...\lambda_K : 
$
\\
$
O_{opt}[n] < (\lambda_K n)^K + (\lambda_{K-1} n)^{K-1}... + \lambda_1 n + C \hspace{4mm} \forall n, D_j \in \mathbb{P}
$
\end{center}





\subsection{Polynomial Order of Complexity}
Total complexity $O[n]$ is said to be of order $K_{opt}$
\begin{center}
$
 O[n] \sim K_{opt}
$
\\ \vspace{2mm}
$O_{opt}\lbrack n \rbrack := O \lbrack n \rbrack :$
\\ \vspace{2mm}
$ \not \exists \hat{O} \lbrack n \rbrack < O_{opt}\lbrack n \rbrack \hspace{2mm} \forall n$
\\ \vspace{2mm}
$O_{opt}[n] < (\lambda_{K_{opt}} n)^{K_{opt}} + (\lambda_{K_{opt}-1} n)^{K_{opt}-1}... + \lambda_1 n +  C \hspace{4mm} \forall n$
\\ \vspace{2mm}
$K_{opt} := K :$
\\ \vspace{2mm}
$\not \exists \hat{K} : O_{T}[n] < (\lambda_{\hat{K}} n)^{\hat{K}} + (\lambda_{\hat{K}-1} n)^{\hat{K}-1}... + \lambda_1 n +  C \hspace{4mm} \forall n,\hspace{1mm} \hat{K} < K$
\end{center}








\subsection{Corrolary of Optimal Complexity}
\begin{center}
$\not \exists s^+ \in S^+ :$
\\ \vspace{2mm}
$O_{T}[n] < (\lambda_{\hat{K}} n)^{\hat{K}} + (\lambda_{\hat{K}-1} n)^{\hat{K}-1}... + \lambda_1 n +  C \hspace{4mm} \forall n,\hspace{1mm} \hat{K} < K_{opt}$
\end{center}

\subsubsection{Proof}
Proof by contradiction; definition of optimal complexity









\subsection{Property of Polynomial  Complexity 1}
\begin{center}
$
lim_{n \rightarrow \infty} \frac{O[n+1]}{O[n]} = 1
$
\end{center}
\subsubsection{Proof WIP}
Show there exists no constant satisfying the decreasing limit condition
\begin{center}
$
O[n] < (\lambda_K n)^K + (\lambda_{K-1} n)^{K-1}... + \lambda_1 n + C
$
\\ \vspace{2mm}
$
O[n+1] < (\lambda_K (n+1))^K + (\lambda_{K-1} (n+1))^{K-1}... + \lambda_1 (n+1) + C
$
\\ \vspace{2mm}
$
O[n] \sim (\lambda n)^K; \hspace{2mm} O[n+1] \sim  (\lambda n)^K
$
\\ \vspace{2mm}
$
lim_{n \rightarrow \infty} \frac{(\lambda n)^K}{(\lambda n)^K} = 1
$
\end{center}







\subsection{Property of Polynomial Complexity 2}
\begin{center}
$
lim_{n \rightarrow \infty}( O[n+1] - O[n] ) \hspace{2mm} diverges
$
\end{center}
\subsubsection{Proof}
\begin{center}
$
O[n+1] < (\lambda_K (n+1))^K + (\lambda_{K-1} (n+1))^{K-1}... + \lambda_1 (n+1) + C
$
\\ \vspace{2mm}
$
O[n] < (\lambda_K n)^K + (\lambda_{K-1} n)^{K-1}... + \lambda_1 n + C
$

\end{center}






\subsection{Order of Complexity}
Total Complexity is said to be on the order of $K_{max}$
\begin{center}
$
O[n] < (\lambda_{K_{max}} n)^{K_{max}} + (\lambda_{K_{max}-1} n)^{K_{max}-1}... + \lambda_1 n + C
$
\\ \vspace{2mm}
$
O[n] \sim K_{max}
$
\end{center}




% Polynomial Complexity Time Based Proofs
\newpage
\section{Polynomial Time Complexity}

\subsection{Definition}
Decision problem $D$ with (optimal) Time Complexity $O_T[n]$ is bounded by polynomial time complexity if
\begin{center}
$\exists K,C,\lambda_1...\lambda_K \hspace{2mm} :$
\\ \vspace{2mm}
$O_T[n] < (\lambda_K n)^K + (\lambda_{K-1} n)^{K-1}... + \lambda_1 n + C \hspace{3mm} \forall n$
\end{center}








\subsection{Polynomial Time Problems}
Define $\mathbb{P}_{time}$, the set of Decision Problems that can be solved with polynomial time complexity
\begin{center}
$
\mathbb{P}_{time} := \{D_1,D_2,...\} : 
$
\\
$
\exists K,C,\lambda_1...\lambda_K : 
$
\\
$
O_T[n] < (\lambda_K n)^K + (\lambda_{K-1} n)^{K-1}... + \lambda_1 n + C \hspace{4mm} \forall n, D_j \in \mathbb{P}_{time}
$
\end{center}






\subsection{Total Polynomial Complexity Implies Time bounded Polynomial Complexity}
\begin{center}
\vspace{1mm}
$
D \in \mathbb{P} \Longrightarrow D \in \mathbb{P}_{time}
$
\end{center}

\subsubsection{Proof}
\begin{center}
$
O[n] < (\lambda_K n)^K + (\lambda_{K-1} n)^{K-1}... + \lambda_1 n + C \hspace{2mm} \forall n
$
\\ \vspace{2mm}
$
O[n] := O_T[n] + O_S[n]; \hspace{2mm} O_T[n] \leq O[n]
$
\\ \vspace{2mm}
$
\therefore O_T[n] < (\lambda_K n)^K + (\lambda_{K-1} n)^{K-1}... + \lambda_1 n + C \hspace{2mm} \forall n
$
\end{center}






\subsection{Time bounded Polynomial Complexity implies Total Polynomial Complexity?}
\subsection{Polynomial Time Complexity iff Polynomial Complexity?}




\subsection{Property of Polynomial Time Complexity 1}
\begin{center}
$
lim_{n \rightarrow \infty} \frac{O_T[n+1]}{O_T[n]} = 1
$
\end{center}
\subsubsection{Proof}


\subsection{Property of Polynomial Time Complexity 2}
\begin{center}
$
lim_{n \rightarrow \infty}( O_T[n+1] - O_T[n] ) \hspace{2mm} diverges
$
\end{center}
\subsubsection{Proof}




\subsection{Order of Complexity}
Time complexity $O_T[n]$ is said to be on the order of $K_{max}$
\begin{center}
$
O_T[n] < (\lambda_{K_{max}} n)^{K_{max}} + (\lambda_{K_{max}-1} n)^{K_{max}-1}... + \lambda_1 n + C
$
\\ \vspace{2mm}
$
O_T[n] \sim K_{max}
$
\end{center}


















% Polynomial Space Complexity
\newpage
\section{Polynomial Space Complexity}
\subsection{Defintion}
\subsection{Polynomial Space Problems}
\subsection{Total Polynomial Complexity Implies Space bounded Polynomial Complexity}
\subsection{Space Bounded Polynomial Complexity Implies Total Polynomial Complexity}
\subsection{Polynomial Space Complexity iff Polynomial Complexity}








\subsection{Property of Polynomial Space Complexity 1}
\begin{center}
$
lim_{n \rightarrow \infty} \frac{O_S[n+1]}{O_S[n]} = 1
$
\end{center}







\subsection{Property of Polynomial Space Complexity 2}
\begin{center}
$
lim_{n \rightarrow \infty}( O_S[n+1] - O_S[n] ) \hspace{2mm} diverges
$
\end{center}








\subsection{Order of Complexity}
Space complexity $O_S[n]$ is said to be on the order of $K_{max}$
\begin{center}
$
O_S[n] < (\lambda_{K_{max}} n)^{K_{max}} + (\lambda_{K_{max}-1} n)^{K_{max}-1}... + \lambda_1 n + C
$
\\ \vspace{2mm}
$
O_S[n] \sim K_{max}
$
\end{center}

















% Theorem of Polynomial Duality; Polynomial in time and space
\newpage
\section{Polynomial Duality}
\subsection{Proof of the existence of $O_{S_{opt}}$}
\subsection{Proof of the existence of $O_{T_{opt}}$}






%\subsection{Proof that $O_{T_{opt}}[n] \neq 0$}
%Proof by contradiction
%\subsection{Proof that $O_{S_{opt}}[n] \neq 0$}
%Proof by contradiction
%\subsection{Polynomial in time iff Polynomial in space?}






\subsection{Theorem Either OT or OS is on the order of Oopt}
Proof by contradiction










% Definition of a Duality Function
\subsection{Duality Functions}
\begin{center}
\vspace{1mm}
$
X_i = \{x_1,...,x_n\}
$
\\ \vspace{2mm}
$
D := f \lbrack X_i \rbrack \rightarrow a_{o} \in \{\mathbb{T}, \mathbb{F}\} \hspace{3mm} \forall X_i
$
\\ \vspace{2mm}
$
s^+ = s^+\lbrack n \rbrack := P \lbrack X_i \rbrack \rightarrow y_{o} : y_o = a_{o} \hspace{3mm} \forall X_i
$
\\ \vspace{2mm}
$
s^+ = \{ \mathcal{L},\mathcal{M},X_i,y_o\} \hspace{2mm}
$
\\ \vspace{6mm}
$
f_{\mathcal{L} \rightarrow \mathcal{M}} := f[\mathcal{L},\mathcal{M}] \rightarrow \hat{\mathcal{L}}, \hat{\mathcal{M}} :
$
\\ \vspace{2mm}
$
s_{\mathcal{L} \rightarrow \mathcal{M}}^+ = \{f_{\mathcal{L} \rightarrow \mathcal{M}}[\mathcal{L},\mathcal{M}],X_i,y_o\} =
$
\\ \vspace{2mm}
$
\{ \hat{\mathcal{L}},\hat{\mathcal{M}},X_i,y_o\}  \hspace{2mm} \forall s^+ \in S^+; \hspace{2mm} \hat{\mathcal{M}} \subseteq \mathcal{M}; \hspace{2mm} \mathcal{L} \subseteq \hat{\mathcal{L}}
$
\\ \vspace{6mm}
$
f_{\mathcal{M} \rightarrow \mathcal{L}} := f[\mathcal{L},\mathcal{M}] \rightarrow \hat{\mathcal{L}}, \hat{\mathcal{M}} :
$
\\ \vspace{2mm}
$
s_{\mathcal{M} \rightarrow \mathcal{L}}^+ = \{f_{\mathcal{L} \rightarrow \mathcal{M}}[\mathcal{L},\mathcal{M}],X_i,y_o\} =
$
\\ \vspace{2mm}
$
\{ \hat{\mathcal{L}},\hat{\mathcal{M}},X_i,y_o\}  \hspace{2mm} \forall s^+ \in S^+; \hspace{2mm} \hat{\mathcal{L}} \subseteq \mathcal{L}; \hspace{2mm} \mathcal{M} \subseteq \hat{\mathcal{M}}
$
\\ \vspace{7mm}
$
s_{\mathcal{L} \rightarrow \mathcal{M}}^+\lbrack n \rbrack := P_{\mathcal{L}\rightarrow \mathcal{M}} \lbrack X_i \rbrack \rightarrow y_{o} : y_o = a_{o} \hspace{3mm} \forall X_i
$
\\ \vspace{2mm}
$
s_{\mathcal{M} \rightarrow \mathcal{L}}^+\lbrack n \rbrack := P_{\mathcal{M}\rightarrow \mathcal{L}} \lbrack X_i \rbrack \rightarrow y_{o} : y_o = a_{o} \hspace{3mm} \forall X_i
$
\end{center}













% Inductive Function
\subsection{Inductive Function O[n+1]}
System of equations? Might be able to tie back to O[n]
\begin{center}
\vspace{2mm}
$
O[n] := O_T[n] + O_S[n]
$
\\ \vspace{2mm}
$
O[n+1] = O_T[n+1] + O_S[n+1]
$
\end{center}










% Inductive Function in relation to duality function
\subsubsection{Connection to duality functions}
System of equations? Might be able to tie back to O[n]
\begin{center}
\vspace{2mm}
$
O[n] := O_T[n] + O_S[n]
$
\\ \vspace{2mm}
$
O[n+1] = O_T[n+1] + O_S[n+1]
$
\\ \vspace{2mm}
$
O_T[n] := |\mathcal{L}| = N; \hspace{2mm} O_S[n] := |\mathcal{M}| = M
$
\\ \vspace{6mm}




$
s_{\mathcal{L} \rightarrow \mathcal{M}}^+\lbrack n \rbrack := P_{\mathcal{L}} \lbrack X_i \rbrack \rightarrow y_{o} : y_o = a_{o} \hspace{3mm} \forall X_i =
$
\\ \vspace{2mm}
$
\{f_{\mathcal{L} \rightarrow \mathcal{M}}[\mathcal{L},\mathcal{M}],X_i,y_o\} = \{ \hat{\mathcal{L}},\hat{\mathcal{M}},X_i,y_o\}  \hspace{3mm} \forall s^+ \in S^+; \hspace{3mm} \hat{\mathcal{M}} \subseteq \mathcal{M}; \hspace{2mm} \mathcal{L} \subseteq \hat{\mathcal{L}}
$
\\ \vspace{8mm}




$
s_{\mathcal{M} \rightarrow \mathcal{L}}^+\lbrack n \rbrack := P_{\mathcal{M}} \lbrack X_i \rbrack \rightarrow y_{o} : y_o = a_{o} \hspace{3mm} \forall X_i
$
\\ \vspace{2mm}
$
\{f_{\mathcal{M} \rightarrow \mathcal{L}}[\mathcal{L},\mathcal{M}],X_i,y_o\} = \{ \hat{\mathcal{L}},\hat{\mathcal{M}},X_i,y_o\}  \hspace{3mm} \forall s^+ \in S^+; \hspace{3mm} \hat{\mathcal{L}} \subseteq \mathcal{L}; \hspace{2mm} \mathcal{M} \subseteq \hat{\mathcal{M}}
$
\\ \vspace{6mm}







$
limit_{n \rightarrow \infty} \frac{O[n+1]}{O[n]} = 1
$




\end{center}









% Theorem of Polynomia? Duality
\subsection{Theorem of (Polynomial?) Duality}
For all Problems in P there exists a duality function\\
Formally define dynamic programming, Optimal polynomial complexity minimizes the difference between time and space complexity order
\begin{center}
\vspace{2mm}
$
D \in \mathbb{P}
$
\\ \vspace{2mm}
$
O[n] := O_T[n] + O_S[n]
$
\\ \vspace{2mm}
$
O_{opt}[n] := O[n] :
$
\\ \vspace{2mm}
$
\not \exists \hat{O} \lbrack n \rbrack < O[n] \hspace{3mm} \forall n
$
\\ \vspace{6mm}
$
O_T^+[n] := |\mathcal{L}| = N :
$
\\ \vspace{2mm}
$
\not \exists \hat{O_T} \lbrack n \rbrack < O^+_{T}[n] \hspace{3mm} \forall n
$
\\ \vspace{2mm}
$
O_S^+[n] := |\mathcal{M}| = M
$
\\ \vspace{2mm}
$
\not \exists \hat{O_S} \lbrack n \rbrack < O^+_{S}[n] \hspace{3mm} \forall n
$
\end{center}







% Polynomial Duality Proof
\subsection{Proof}
Prove that Order can be subtracted from Os or Ot and added to the other; double check cauchy schwartz inequal
\begin{center}
$
O[n] < (\lambda_K n)^K + (\lambda_{K-1} n)^{K-1}... + \lambda_1 n + C \hspace{4mm} \forall n, D_j \in \mathbb{P}
$
\\ \vspace{2mm}
$
O[n] := O_T[n] + O_S[n]
$
\\ \vspace{2mm}
$
O_T[n] + O_S[n] < (\lambda_K n)^K + (\lambda_{K-1} n)^{K-1}... + \lambda_1 n + C \hspace{4mm} \forall n, D_j \in \mathbb{P}
$
\end{center}







\subsection{There exists an optimal OT and OS on the order of Oopt}

\subsection{Even ordered decision problems}
M-N = N-M = 0
\subsection{Odd ordered decision problems}
N = M + 1 or M = N + 1









% N Sum Problem
\newpage
\subsection{N Sum Problem}

\subsubsection{Restate formal definition}
\begin{center}
\vspace{1.5mm}
$
X_i = \{x_1,...,x_n\}
$
\\ \vspace{2mm}
$
D := f \lbrack X_i \rbrack \rightarrow a_{o} \in \{\mathbb{T}, \mathbb{F}\} \hspace{3mm} \forall X_i
$
\\ \vspace{2mm}
$
s^+ = s^+\lbrack n \rbrack := P \lbrack X_i \rbrack \rightarrow y_{o} : y_o = a_{o} \hspace{3mm} \forall X_i
$
\\ \vspace{2mm}
$
s^+ = \{ s_1,s_2,...,s_{O_T \lbrack n \rbrack }, b_1, b_2,...,b_{O_S \lbrack n \rbrack},X_i,y_o \} = \{ \mathcal{L},\mathcal{M},X_i,y_o\}
$
\\ \vspace{6mm}
$
D = f[X_i] = \exists x_j,x_k \in X_i : x_j + x_k = N
$
\end{center}

\subsubsection{Express a formal solution}
\begin{center}
\vspace{1.5mm}
$
s^+ = \{ s_1,s_2,...,s_{O_T \lbrack n \rbrack }, b_1, b_2,...,b_{O_S \lbrack n \rbrack},X_i,y_o \} = \{ \mathcal{L},\mathcal{M},X_i,y_o\}
$
\\ \vspace{2mm}
$
b_1 = s_{O_S[n]} = y_o \leftarrow \mathbb{F};
$
\\ \vspace{2mm}
$
s_1 ... s_{O_T[n]} = s_1 ... s_{\frac{n(n-1)}{2}} = y_o \leftarrow y_o \cup (x_i + x_j == N)  \hspace{3mm}  \forall i,j > i
$
\\ \vspace{2mm}
$
s^+ = \{y_o \leftarrow \mathbb{F},y_o \leftarrow y_o \cup (x_i + x_j == N) \hspace{3mm} \forall i,j > i\}
$
\end{center}











\subsubsection{Express the order of complexity, order of time complexity, order of space complexity}
\begin{center}
\vspace{1.5mm}
$
O_T[n] = \frac{n(n-1)}{2} \sim n^2
$
\\ \vspace{2mm}
$
O_S[n] = 1 \sim n^0
$
\\ \vspace{2mm}
$
O[n] = \frac{n(n-1)}{2} + 1 \sim n^2
$
\end{center}





\subsubsection{Express O[n+1] in terms of O[n]}
\begin{center}
$
O[n] = n^2 - n + 2 
$
\\ \vspace{2mm}
$
O[n+1] = (n+1)^2 - n - 1  + 2 = n^2 + 2n + 1 - n - 1 + 2 = O[n] + 2n 
$
\end{center}



\subsubsection{Prove |$\mathbb{P}$| > 0}
\begin{center}
$
O[n] = n^2 - n + 2 
$
\\ \vspace{2mm}
$
O[n] = n^2 - n +2  < n^2 - n + 3 \hspace{3mm} \forall n
$
\\ \vspace{2mm}
$
\therefore D \in \mathbb{P}
$
\end{center}








% Alternate Solution
\newpage
\subsubsection{N Sum alternate solution}
\begin{center}
\vspace{1.5mm}
$
s_1 = (b_1 = y_o) \leftarrow \mathbb{F}
$
\\ \vspace{2mm}
$
s_2,s_4,...,s_{2n} = s_{2i} =  b_{i+1} \leftarrow N - x_i \hspace{3mm} i = 1...n
$
\\ \vspace{2mm}
$
s_3,s_5 ... s_{2n+1} = s_{2i+1} = y_o \leftarrow y_o \cup (x_i \in \mathcal{M})) \hspace{3mm} i = 1...n
$
\\ \vspace{2mm}
$
O_S[n] = n + 1; O_T[n] = n(1 + O_s[n])
$
\\ \vspace{2mm}
$
O[n] := O_T[n] + O_S[n] = n(1 + O_S[n]) + n + 1 = n(1 + n + 1) + n + 1
$
\\ \vspace{2mm}
$
O[n] = n^2 + 3n + 1
$
\end{center}


\subsubsection{Find a dual function of solution $s^+$}
\begin{center}
$
O_S[n] = n + 1; O_T[n] = n(1 + O_S[n])
$
\\ \vspace{2mm}
$
O[n] = O_T[n] + O_S[n] = n(1 + O_S[n]) + O_S[n] = n + O_S[n](1+n)
$
\end{center}

\subsubsection{Find an inductive function for $O_S[n]$}
\begin{center}
$
O_S[n] = n + 1; O_S[n+1] = n+2
$
\\ \vspace{2mm}
$
O_S[n+1] = n + 2 = (n+1) + 1 = O_S[n] + 1
$
\end{center}

% O[n+1] Alternate Solution
\subsubsection{Find an expression for O[n+1] as a function of O[n]}
\begin{center}
$
O[n] = O_T[n] + O_S[n] =n(1+O_S[n]) + O_S[n]
$
\\ \vspace{2mm}
$
O[n+1] = n+1(1+O_S[n]) + O_S[n]
$
\\ \vspace{2mm}
$
O[n] + inductive[n] = O[n+1]
$
\\ \vspace{2mm}
$
n(1+O_S[n]) + O_S[n] + inductive[n] =  (n+1)(1+O_S[n + 1]) + O_S[n + 1] 
$
\\ \vspace{2mm}
$
inductive[n] =  (n+1)(1+O_S[n + 1]) + O_S[n + 1]  - n(1+O_S[n]) - O_S[n]
$
\\ \vspace{2mm}
$
inductive[n] =  (n+1)(1+O_S[n] + 1) + O_S[n] + 1  - n(1+O_S[n]) - O_S[n]
$
\\ \vspace{2mm}
$
inductive[n] =  (n+1)(O_S[n] + 2) + 1  - n(1+O_S[n])
$
\\ \vspace{2mm}
$
inductive[n] =  nO_S[n] + 2n + O_S[n] + 2 + 1 - n - nO_S[n]
$
\\ \vspace{2mm}
$
inductive[n] =  2n + O_s[n] - n + 3 = 2n + n + 1 - n + 3 = 2n + 4
$
\\ \vspace{4mm}
$
O[n+1] = O[n] + inductive[n] = O[n] + 2n + 4
$

%$
%O[n] = n^2 + 3n + 1
%$
%\\ \vspace{2mm}
%$
%O[n+1] = (n+1)^2 + 3(n+1) + 1 = n^2 + 2n + 1 + 3n + 3 +1 = n^2 + 5n + 5
%$
%\\ \vspace{2mm}
%$
%O[n+1] = n^2 + 3n + 1 + 2n + 4 = O[n] + 2n + 4
%$
\end{center}



% Limit without an explicit O[n] N Sum
\subsubsection{Show the limit$_{n\rightarrow\infty} \frac{O[n+1]}{O[n]}$ = 1}
\begin{center}
$
limit_{n\rightarrow\infty} \frac
{O[n+1]}
{O[n]} =
$
\\ \vspace{4mm}
$
limit_{n\rightarrow\infty} \frac
{O[n] + 2n + 4}
{O[n]} =
$
\\ \vspace{4mm}
$
limit_{n\rightarrow\infty} 1 + \frac{2n+4}{O[n]}
$
\\ \vspace{4mm}
$
\not \exists K : 1 - 1 + \frac{2n + 4}{O[n]} > K \hspace{3mm} \forall n,K>0
$
\\ \vspace{4mm}
$
\therefore  limit_{n\rightarrow\infty} \frac{O[n+1]}{O[n]} = 1
$
\end{center}






\subsubsection{Prove |$\mathbb{P}$| > 0}
\begin{center}
$
D \in \mathbb{P} \Longleftrightarrow limit_{n\rightarrow\infty} \frac{O[n+1]}{O[n]} = 1
$
\\ \vspace{2mm}
$
\therefore D \in \mathbb{P}
$
\end{center}




\subsubsection{Criticism on interpretation of hashing solutions}
In the below solution to N sum problem, the solution is typically considered to be $O_T[n] \sim n$. However, the line "if element in M:" 
requires a search through an (indexed) dictionary. During element $x_i$; |M| = i . If M is pre-allocated to the total number of elements the search requires n lookups each iteration
where n = |$int\_list$|. Additional proofs required for optimized indexing. Prove the cost of storing and querying yields the same or different optimal order of complexity.\\
\\
def SumToN(int\_list,N): \\
output = False;\\
M = \{\}; \\
for element in int\_list: \\
if element in M:\\
output = True;\\
else:\\
M[N - element] = True;\\
output = output;\\
return output;














% NP Problems
\newpage
\section{Definition of Non-Polynomial Problems}
Define $\mathcal{N}$, the set of Decision Problems that cannot be solved with Polynomial Time Complexity
\begin{center}
$
\mathcal{N} := \{ D_1,D_2,...\} 
$
\\
$ 
\not \exists K,C,\lambda_1...\lambda_K : \hspace{2mm}
$
\\
$
\hspace{2mm} O_T[n] < (\lambda_K n)^K + (\lambda_{K-1} n)^{K-1}... + \lambda_1 n + C \hspace{4mm} \forall n, D_j \in \mathcal{N}
$
\end{center}






\section{Divergent Problems}
\subsection{Definition}
\begin{center}
$
\mathcal{\hat{D}} := \{ \hat{D}_1,\hat{D}_2,...\} 
$
\\ \vspace{2mm}
$
lim_{n \rightarrow \infty} \frac{\hat{O}[n+1]}{\hat{O}[n]} \hspace{2mm} diverges
$
\\ \vspace{2mm}
$
\forall s^+_{\hat{D}}
$
\end{center}
%\begin{center}
%$
%\mathcal{\hat{D}} := \{ D_1,D_2,...\} 
%$
%\\ \vspace{2mm}
%$
%\hspace{2mm} \hat{O}[n] \geq n^n \hspace{3mm} \forall n, D_j \in \mathcal{N}
%$
%\end{center}

\subsection{Property of Divergent Problem Complexity 1}
\begin{center}
$
lim_{n \rightarrow \infty} \frac{\hat{O}[n+1]}{\hat{O}[n]} \hspace{2mm} diverges
$
\end{center}
\subsubsection{Proof}


\subsection{Property of Divergent Problem Complexity 2}
\begin{center}
$
lim_{n \rightarrow \infty}( \hat{O}[n+1] - \hat{O}[n] ) \hspace{2mm} diverges
$
\end{center}
\subsubsection{Proof}








\subsection{Divergent Duality?}
\subsection{Divergent Induction Functions?}









% Fundamental Theorem of Computation
\newpage
\section{Fundamental Theorem of Computation}
Some solutions for Polynomial Problems are divergent; There exist no solution for divergent problems with polynomial bound
\begin{center}
$
\mathbb{P} = \{D_1,D_2,...\}
$
\\ \vspace{2mm}
$
S^+_{\mathbb{P}} = \{s_1,s_2,...\}
$
\\ \vspace{2mm}
$
\not \exists s \in S^+_{\mathbb{P}} : O[n] \geq n^n \hspace{4mm} \forall s_i \in S^+_{\mathbb{P}} 
$
\\ \vspace{2mm}
$
\mathbb{P} \cap \hat{D} = \emptyset
$
\\ \vspace{6mm}
$
limit_{n \rightarrow \infty} \nestedloop [s^+[n]] = n^n  \hspace{4mm} \forall s_i \in S^+_{\mathbb{P}} 
$
\end{center}

\subsection{Proof}
\begin{center}
$
X_i = \{x_1,...,x_n\}
$
\\ \vspace{2mm}
$
D := f \lbrack X_i \rbrack \rightarrow a_{o} \in \{\mathbb{T}, \mathbb{F}\} \hspace{3mm} \forall X_i
$
\\ \vspace{6mm}
$
Let \hspace{2mm} D \in \mathbb{P}
$
\\ \vspace{2mm}
$
s^+ := P \lbrack X_i \rbrack \rightarrow y_{o} : y_o = a_{o} \hspace{3mm} \forall X_i
$
\\ \vspace{2mm}
$
O[n] = O_T[n] + O_S[n] < (\lambda_K n)^K + (\lambda_{K-1} n)^{K-1}... + \lambda_1 n + C \hspace{3mm} \forall n
$
\end{center}

\subsection{Non-Polynomial implies Divergent}




















% Proof of Non-Polynomial Problems Traveling Salesman
\newpage
\section{Proof of the existence of $\hat{\mathcal{D}}$}
Non-trivial; Formalize the traveling salesman problem as a decision problem (any optimization problem)

\subsection{The Traveling Salesman Problem}
English description

\subsection{Formal Definition}
\begin{center}
$
X_i = \{ c_1,c_2,...,c_n \} :
$
\\ \vspace{2mm}
$
dim\lbrack c_i \rbrack = C > 1
$
\\ \vspace{4mm} 
$
\bar{X_i} = \{ c_1,c_2,...,c_n,\bar{P},\bar{f}[c_i,c_j] \}
$
\\ \vspace{2mm}
$
\bar{P} := \{c_k,...\} :
$
\\ \vspace{2mm}
$
\exists c_k \in \bar{P} \hspace{2mm} \forall c_k \in X_i
$
\end{center}

\subsection{Determine $s^+ \Longrightarrow O_{S_{opt}}$}
\subsection{Express $O[n],O_T[n],O_S[n] =  O_{S_{opt}}$}
\subsection{Revisit expressions properties inequalities connecting $O_{opt}; O_{T_{opt}}; O_{S_{opt}}$}
\subsection{Determine an alternate solution storing subpaths}
\subsection{Express $O[n],O_T[n],O_S[n]$}
\subsection{Determine a dual function}
\subsection{Show $limit_{n \rightarrow \infty} \frac{\hat{O}[n+1]}{\hat{O}[n]}$ \hspace{1mm} diverges}
\subsection{Let $\hat{s}^+$; a solution with $limit_{n \rightarrow \infty} \frac{\hat{O}[n+1]}{\hat{O}[n]}$ = 1}
\subsection{Show $\hat{s}^+$ implies a contradiction}





\newpage
\section{Proof of "P $\neq$ NP"}











% Theorem of Prime Numbers "Riemann Hypothesis"
\newpage
\section{Theorem of Prime Numbers "Riemann Hypothesis"}
Riemann Zeta Function
\begin{center}
$
\zeta (s) \equiv \sum_{n=1}^{\infty} \frac{1}{n^s} \hspace{3mm} \lbrack 2 \rbrack
$
\end{center}
"The prime number theorem determines the average distribution of the primes. The Riemann hypothesis tells us about the deviation from the average. Formulated in Riemann's 1859 paper, it asserts that all the 'non-obvious' zeros of the zeta function are complex numbers with real part 1/2." \lbrack 2\rbrack\\
\\
Prove the problem is divergent\\
There fore it can only be proven to a certain degree\\
The limit as n approaches infinity implies a real part of one half\\
Connection with the real and imaginary part of O $\lbrack n \rbrack$

\subsection{Determine a duality function for the Riemann Hypothesis}
\subsection{Determine an expression for O[n+1] as a function of O[n]}

\subsection{Prove $O_{opt}$ is performing $O_{opt}$ recursively for the ints less than square root of n}
Testing the primes less than sqrt(n)? double check \\
1. Optimal solution for n=1,2,3, everything else is a recursive optimal proof by induction\\
Time Complexity seems to be on the order of n log n... implies divergence or lack of bound? Add in the complexity of division.. probably approaches $n^n$

%\subsection{Show that $O_{opt}$ diverges with $n^n$, isn't bounded by $n^n$}
%Proves O is divergent

\subsection{Since divergent, no $s^{+}$ exists.. only rules}
Express as a limit

\subsection{Show that the limit as n $\rightarrow \infty$ implies the real part is 1/2}
1/2 ± 14.134725 i
1/2 ± 21.022040 i
1/2 ± 25.010858 i
1/2 ± 30.424876 i
1/2 ± 32.935062 i
1/2 ± 37.586178 i

Z = $\zeta(1/2 + it)$

\subsection{Notation, real imaginary parts of the problem}
Even numbers and numbers ending in 5 are automatically convergent\\
Testing numbers ending in 1,3,7,9 results in divergent expression\\
we can continue to add rules to a certain degree




















\newpage
\section{Divergent Problems}
Define $\hat{\mathbb{D}}$ the set of decision problems with no convergent?/finite? solution $\hat{D_j}$\\
\begin{center}
$
\hat{\mathbb{D}} := \{\hat{D}_j,...\}
$
\\ \vspace{2mm}
$
\hat{D_j} \in \mathbb{D} \hspace{3mm} \forall j
$
\\ \vspace{2mm}
$
\not \exists \hat{s}^{+} \in S^{+} : \hat{s}^{+}$ solves $\hat{D}_j \hspace{3mm} \forall \hat{D_j} \in \hat{\mathbb{D}} \Longleftrightarrow 
$
\\
$
\not \exists s^{+} \in S^{+} : O_j \lbrack n \rbrack < n^n  \hspace{3mm} \forall n,j
$
\end{center}
There exists no such solution such that O[n] < $n^n$, but there is a right and wrong answer\\
Either here or in the next chapter we'll prove you can only solve to a certain degree\\
 !!! There exists no such solution such that O[n] < $n^n$ \hspace{3mm} $\forall$ n
 
 \subsection{Definition}
 \begin{center}
$
\hat{O} \lbrack n \rbrack := n^n
$
\\ \vspace{2mm}
$
limit_{n \rightarrow \infty}\frac {O_{opt} \lbrack n \rbrack}{\hat{O}\lbrack n \rbrack} \hspace{2mm} diverges
$
\\ \vspace{2mm}
$
diverges \lbrack \hspace{1mm} O_{opt} \lbrack n \rbrack, \hat{O}\lbrack n \rbrack \hspace{1mm} \rbrack \rightarrow \mathbb{T}
$
\end{center}

\subsection{Theorem of Divergent Programs}
Prove that Divergent implies not in polynomial (trivial)\\
Prove that Divergent implies Non-polynomial (trial after proving above)\\
Show that there exists at least one member of Divergent\\


\section{Properties of Solvable and Divergent problems}

\subsection{Solvable and Divergent are disjoint}
Prove by contradiction



 \section{"Theorem of Divergent Programs"}
\subsection{Divergence Test}
1. Let $d_j \in D$\\
2. $d_j$ = ($d_j \in \hat{\mathcal{D}}$) $\cup$ ($d_j \in$ set of solvable problems) by disjoint condition of solvable and divergent\\
3. Let $O_{opt}\lbrack n \rbrack$, the optimal complexity of $d_j$\\
4. $\rightarrow s_i^{+}$ that solve $d_j$ have larger complexity $\forall i$\\
5. 2 implies  $O_{opt}\lbrack n \rbrack$ is either bounded by $n^n$ or not\\
6. $\hat{O} \lbrack n \rbrack \equiv n^n$\\
7. Easy Suppose $d_j\in $solvable$ limit_{n \rightarrow \infty}\frac{O_{opt}\lbrack n \rbrack}{\hat{O}\lbrack n \rbrack} = 0 $\\
8.Suppose $d_j\in \hat{\mathcal{D}} limit_{n \rightarrow \infty}\frac{O_{opt}\lbrack n \rbrack}{\hat{O}\lbrack n \rbrack} \neq 0 $ (by disjoint condition)\\
\begin{center}
$
 limit_{n \rightarrow \infty}\frac{O_{opt}\lbrack n \rbrack}{\hat{O}\lbrack n \rbrack} = 1 
$
\end{center}




\section{Connection to verification in polynomial time}






\newpage


\section{Fundamental Theorem of Computation}
$n^n$ or $\lambda n^n + C$ the universal bound to solvable computational complexity\\
$(\lambda n)^n$ + C ?\\






\subsection{Time Complexity Argument}
Suppose decision problem d with optimal time complexity $O_{T_{min}}\lbrack n \rbrack$ and solution $s^+$, an arbitrary decision problem in P with polynomial complexity\\
Assumptions\\
1. $d \in P$, $s^{+} \in S^{+}$\\
Assertions\\
2. $\exists K,C,\lambda_1...\lambda_K \hspace{2mm} : \hspace{2mm} O_{T_{min}}\lbrack n \rbrack < (\lambda_K n)^K + (\lambda_{K-1} n)^{K-1}... + \lambda_1 n + C, \hspace{3mm} \forall n\hspace{1mm}$
\\
3. Define $ f\lbrack K,C, \lambda_1,...\lambda_K \rbrack \equiv (\lambda_K n)^K + (\lambda_{K-1} n)^{K-1}... + \lambda_1 n + C$\\
4.  $\exists K,C,\lambda_1...\lambda_K : O_{T_{min}}\lbrack n \rbrack <   f\lbrack K,C, \lambda_1,...\lambda_K \rbrack \hspace{3mm} \forall n$\\
5. Let $\hat{s}^{+} \equiv \nestedloop \lbrack s^{+} \rbrack$\\
6. $O_T \lbrack n \rbrack <= \hat{O}_T \lbrack n\rbrack$ (by defition of nested loop)\\
\\
7.  $\hat{O}_{T_{min}}\lbrack n \rbrack < (\lambda_K n)^K + (\lambda_{K-1} n)^{K-1}... + \lambda_1 n + C$\\
8. $\hat{O}_{T_{min}} \lbrack n \rbrack < limit_{n \rightarrow \infty}  \nestedloop \lbrack s^{+} \rbrack$ (by definition of limit + definition of nested loop, expand to show full derivation, valid because this is a series, probably need to show limit applies)\\
9. $\therefore  O_{T_{min}}\lbrack n \rbrack < \hat{O}_{T_{min}} \lbrack n \rbrack < n^n  =  limit_{n \rightarrow \infty} \nestedloop \lbrack s^{+} \rbrack$
\\
I want to say forall n but seems refutable for n= 1,2.. but as n approach infinity it's a contradiction to say a solvable problem in P $\hat{O}_{T_{min}} = n^n \hspace{3mm} \forall n$\\
10. For "sufficiently large n"
\begin{center}
$
\not \exists \hat{s}^{+} \in S^{+} : | \hat{s}^{+} | \equiv O_{T_{min}} \lbrack n \rbrack <  n^n, \hspace{3mm} \forall n
$
\\
$
\hat{O} \lbrack n \rbrack \equiv n^n
$
\end{center}

\subsection{Space Argument}
Similar but additional notation required?






\newpage
\section{Divergence Criterion}
Necessary condition for divergent program, iff\\
 or you can show there exists no lambda, C such that O $\lbrack n \rbrack$ is $n^n$ is bounded by $\lambda n^n + C$ for all n

$limit_{n \rightarrow \infty}$ div / solvable > 1

Assumptions\\
1. Define the "Null Space of $\mathcal{D}$" or "Null Set" $O_\perp$
\begin{center}
$
O_\perp = \{\hat{d_1},\hat{d_2},...,\hat{d_j}\}, \hspace{3mm} j > 0
$
\\
$
 \hat{O_j} [n] \equiv (O[n])^n, \forall j
$
\end{center}
2. $O_P \cup O_\mathcal{N} =\mathcal{D} \hspace{3mm}$ (by definition)\\

Assertions\\
3. $O_P$ $\cap$ $O_\perp$ = $\emptyset$\\
4. Let $O_\mathcal{N}$ $\cap$ $O_\perp$ = $\hat{O}$ = $\{\hat{O}_i,...\}$, $i > 0$\\
5. Consider $D_j \in O_\mathcal{N}$\\
6. $D_j$ has finite complexity by definition\\
\begin{center}
 $O_j[n] = C$
\end{center}
7. $D_j$ has at least one optimal solution by the necessity of optimal solution (theorem Z)
\begin{center}
 $O_j[n] = C$
\end{center}
%$ $O_\perp$ = $\emptyset$

































\newpage

\section{Proof of "P $\neq$ NP"}
\subsection{Proof N implies D}
Is trivial by implication of Theorem x and Theorem y
\begin{center}
$
\mathcal{N} \equiv \{ d_j,.. \} \hspace{3mm} \forall j, \mathcal{N} \in D
$
\\
$ 
\not \exists K,C,\lambda_1...\lambda_K : \hspace{2mm}
$
\\
$
\hspace{2mm} O_T[n] < (\lambda_K n)^K + (\lambda_{K-1} n)^{K-1}... + \lambda_1 n + C, \hspace{3mm} \forall n, \hspace{1mm} \forall d_j \in \mathcal{N}
$
\end{center}
1. $\rightarrow \mathcal{P} \cap \mathcal{N} = \emptyset$ by definition of P,N\\
2. $d_i \in \hat{D} \lor d_i \in $solvable\\
3. 1 $\rightarrow d_i \not \in$ solvable\\
4. $\therefore d_i \in \hat{D}\hspace{3mm} \forall i$ (theorem y)
Show that Definition of Non-Polynomial Problems automatically implies Divergent\\
1. We've proven Solvable Union are disjoint and complete set P
2. N not in P by definition
3. therefore N in divergence by set theory

Currently we have only defined solvable problems and divergent problems\\
Additionally polynomial problem which the existance of is trivial\\
Plus we defined non-polynomial complexity\\
Prove the existence of $\mathcal{N}$ the set of non polynomial problems\\
\\

\subsection{Proof that D implies N}
\subsection{D iff N}

Show  $O \lbrack n \rbrack$ in the $\emptyset$ the set of problems with    $n^n$ > $O \lbrack n \rbrack$ > $n^k + c$\\
Proving there's Polynomial and Divergent, in the set of all decision problems\\


A neat follow up, tie in the definition of $\mathcal{N}$ implies membership to divergent problems


\newpage

\section{Prove the existance of D = N,The Traveling Salesman Problem}
Define the traveling salesman problem, prove it is divergent and has the same solution as current approaches\\
Consider proving with both definition and necessary condition\\

\subsection{Compute every sub path or recursive subpaths in memory}
Trade off between time and space, $O_{salesman}[n]$ diverges with a polynomial $O_P[n]$



\section{Prove Polynomial and Divergent problems are Complements}
Implied by the previous sections

\section{Solvable Union Divergent = all decision problems}
Trivial as a result of the previous section by definition of $\Omega$
\begin{center}
$\mathbb{P} \cup \hat{\mathbb{D}} = \mathbb{D}$
\end{center}





\newpage
\section{Theorem of Prime Numbers "Riemann Hypothesis"}
Riemann Zeta Function
\begin{center}
$
\zeta (s) \equiv \sum_{n=1}^{\infty} \frac{1}{n^s} \hspace{3mm} \lbrack 2 \rbrack
$
\end{center}
"The prime number theorem determines the average distribution of the primes. The Riemann hypothesis tells us about the deviation from the average. Formulated in Riemann's 1859 paper, it asserts that all the 'non-obvious' zeros of the zeta function are complex numbers with real part 1/2." \lbrack 2\rbrack\\
\\
Prove the problem is divergent\\
There fore it can only be proven to a certain degree\\
The limit as n approaches infinity implies a real part of one half\\
Connection with the real and imaginary part of O $\lbrack n \rbrack$

\subsection{Prove $O_{opt}$ is performing $O_{opt}$ recursively for the ints less than square root of n}
Testing the primes less than sqrt(n)? double check \\
1. Optimal solution for n=1,2,3, everything else is a recursive optimal proof by induction\\
Time Complexity seems to be on the order of n log n... implies divergence or lack of bound? Add in the complexity of division.. probably approaches $n^n$

%\subsection{Show that $O_{opt}$ diverges with $n^n$, isn't bounded by $n^n$}
%Proves O is divergent

\subsection{Since divergent, no $s^{+}$ exists.. only rules}
Express as a limit

\subsection{Show that the limit as n $\rightarrow \infty$ implies the real part is 1/2}
1/2 ± 14.134725 i
1/2 ± 21.022040 i
1/2 ± 25.010858 i
1/2 ± 30.424876 i
1/2 ± 32.935062 i
1/2 ± 37.586178 i

Z = $\zeta(1/2 + it)$

\subsection{Notation, real imaginary parts of the problem}
Even numbers and numbers ending in 5 are automatically convergent\\
Testing numbers ending in 1,3,7,9 results in divergent expression\\
we can continue to add rules to a certain degree


\newpage
\section*{Citations}
\lbrack 1\rbrack \hspace{1mm} $ https://www.claymath.org/millennium-problems$\\
\lbrack 2\rbrack \hspace{1mm} $ https://www.claymath.org/sites/default/files/official\_problem\_description.pdf$
\lbrack 3 \rbrack \hspace{1mm} $http://www.math.uchicago.edu/~may/VIGRE/VIGRE2011/REUPapers/Riffer-Reinert.pdf$


\end{document}