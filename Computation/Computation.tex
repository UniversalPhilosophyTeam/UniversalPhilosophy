\documentclass[11pt]{article}
\usepackage{amsfonts}
\usepackage[T1]{fontenc}
\usepackage{mathabx,graphicx}

\newcommand{\test}{\circlearrowright}
\def \loop {\ensuremath{\rotatebox[origin=c]{-90}{$\circlearrowright$}}}
\def \nestedloop {\ensuremath{\rotatebox[origin=c]{-90}{$\circlearrowright$}}^n}

\begin{document}

\section*{Ch. 5 Computation}





% Definition of Program
\section{Programs}
\subsection{Logical Instructions}
Define $\mathcal{L}$; an ordered set of logical operations $s_i$
\begin{center}
$
\mathcal{L} := \{ s_1,s_2,...,s_{N}\}
$
\end{center}





\subsection{Memory}
Define Memory $\mathcal{M}$; a set of elements
\begin{center}
$\mathcal{M} := \{b_1,b_2,...,b_M\}$
\end{center}




\subsection{State |}
Define state; the memory utilized to perform program P
\begin{center}
$
P := \{ s_1, s_2,...,s_{N} | b_1, b_2,...,b_M\} =
$
\\ \vspace{2mm}
$
\{ s_1, s_2,...,s_{N}, b_1, b_2,...,b_M\}
$
\end{center}





\subsection{Boolean Programs}
Define a boolean program; boolean programs can represent functions with inputs $x_i$ and boolean output $y_o$
\begin{center}
$
X = \{x_1,...,x_n\}
$
\\ \vspace{2mm}
$P = P\lbrack X \rbrack := \{ s_1,s_2,...,s_{N}\hspace{1mm}|\hspace{1mm} b_1, b_2,...,b_M,y_o\}=$
\\ \vspace{2mm}
$
P\lbrack X \rbrack \rightarrow y_o \in \{ \mathbb{T},\mathbb{F}\}
$
\end{center}





\subsection{Void Programs}
Define a void program; a program with inputs $x_i$ and no output
\begin{center}
$
X = \{x_1,...,x_n\}
$
\\ \vspace{2mm}
$P = P\lbrack X \rbrack := \{ s_1,s_2,...,s_{N}\hspace{1mm}|\hspace{1mm} b_1, b_2,...,b_M\}$
\end{center}





\subsection{Numerical Programs}
Define a numerical program; a program with inputs $x_i$ and real, rational output $y_o$
\begin{center}
$
X = \{x_1,...,x_n\}
$
\\ \vspace{2mm}
$P = P\lbrack X \rbrack := \{ s_1,s_2,...,s_{N}\hspace{1mm}|\hspace{1mm} b_1, b_2,...,b_M,y_o\}=$
\\ \vspace{2mm}
$
P\lbrack X \rbrack \rightarrow y_o \in \mathbb{Q} \hspace{2mm} y_o \geq 0
$
\end{center}





\subsection{System Programs}
Naming convention to be formalized; a program that outputs one or more elements
\begin{center}
$
X = \{x_1,...,x_n\}
$
\\ \vspace{2mm}
$P = P\lbrack X \rbrack := \{ s_1,s_2,...,s_{N}\hspace{1mm}|\hspace{1mm} b_1, b_2,...,b_M,Y_o\}=$
\\ \vspace{2mm}
$
P\lbrack X \rbrack \rightarrow Y_o = \{y_1,y_2,...,y_K\}
$
\end{center}





\subsection{Mathematical Programs}
Define a mathematical program; a program with inputs $x_i$ and numerical output $y_o$
\begin{center}
$
X = \{x_1,...,x_n\}
$
\\ \vspace{2mm}
$P = P\lbrack X \rbrack := \{ s_1,s_2,...,s_{N}\hspace{1mm}|\hspace{1mm} b_1, b_2,...,b_M,y_o\}=$
\\ \vspace{2mm}
$
P\lbrack X \rbrack \rightarrow y_o \in \mathbb{Q}
$
\end{center}




% No-op; While Loop; For Loop
\newpage
\section{No-op ;}

\subsection{Definition}
\begin{center}
$
; := \emptyset
$
\end{center}

\subsection{Property of No-op}
No-op can be inserted into any set with equality
\begin{center}
$
S= \{s_1,s_2,...,s_N\}
$
\\ \vspace{2mm}
$
S_; = insert \lbrack S,;, i \rbrack
$
\\ \vspace{2mm}
$
S_; = S_1 \hspace{2mm} \forall i
$
\\ \vspace{2mm}
$
|S_;| = |S| \hspace{2mm} \forall i
$
\end{center}


\subsection{Proof}
by definition of magnitude of null = 0 with Set And





%\section{For Loop \loop}
%\loop \lbrack startindex, endindex, condition\rbrack
%\subsection{Definition}








%\section{Nested For Loop $\nestedloop$}
%\loop \lbrack startindex, endindex, condition1,...condition n\rbrack
%\subsection{Definition}








% Decision Problems and Solutions
\newpage
\section{Decision Problems}

\subsection{Definition}
Define decision problem; a function with inputs $x_i$ and boolean output "answer" $a_o$
\begin{center}
$
X_i = \{x_1,...,x_n\}
$
\\ \vspace{2mm}
$
D := f \lbrack X_i \rbrack \rightarrow a_{o} \in \{\mathbb{T}, \mathbb{F}\} \hspace{3mm} \forall X_i
$
\end{center}







\section{Solutions}

\subsection{Definition}
Program P is a solution $s^{+}$ if P outputs answer $a_o$ for all inputs $X_i \hspace{2mm} \forall i$\\
$s^+$ is a function of the number of inputs n
\begin{center}
\vspace{1mm}
$
X_i = \{x_1,...,x_n\}; \hspace{2mm} \hat{X}_i = \{x_1,...,x_n,x_{n+1}\}
$
\\ \vspace{2mm}
$
D := f \lbrack X_i \rbrack \rightarrow a_{o} \in \{\mathbb{T}, \mathbb{F}\} \hspace{3mm} \forall X_i
$
\\ \vspace{2mm}
$
s^+ = s^+\lbrack n \rbrack := P :
$
\\ \vspace{2mm}
$
(P \lbrack X_i \rbrack \rightarrow y_{o} == a_{o} \hspace{3mm} \forall X_i) \hspace{2mm} \cap \hspace{2mm} (P[\hat{X}_i] \supseteq P[X_i] \hspace{3mm} \forall X_i,\hat{X}_i)
$
\\ \vspace{2mm}
$
P \lbrack X_i \rbrack = \{ s_1,s_2,...,s_N| b_1, b_2,...,b_M,y_o\}
$
\\ \vspace{3mm}
$
s^+ = P \lbrack X_i \rbrack = \{ s_1,s_2,...,s_{O_T \lbrack n \rbrack }, b_1, b_2,...,b_{O_S \lbrack n \rbrack},y_o \} \hspace{3mm} \forall X_i
$
\end{center}





\subsubsection{Property of No-op ;}
No-op ; can be added to any solution $S_i$ and remain a solution for all i anywhere in the order for all j
\begin{center}
$
s^+ = \{ s_1,s_2,...,s_{O_T \lbrack n \rbrack }, b_1, b_2,...,b_{O_S \lbrack n \rbrack},y_o\}
$
\\ \vspace{2mm}
$
\hat{s}^+ = insert \lbrack s^+,;,k \rbrack
$
\\ \vspace{2mm}
$
\hat{s}^+ = s^+ \hspace{2.5mm} \forall k
$
\end{center}





\subsection{Definition of $S^+$}
Define $S^+$; the set of solutions to decision problem D
\begin{center}
$
X_i = \{x_1,...,x_n\}; \hspace{2mm} \hat{X}_i = \{x_1,...,x_n,x_{n+1}\}
$
\\ \vspace{2mm}
$
D := f \lbrack X_i \rbrack \rightarrow a_{o} \in \{\mathbb{T}, \mathbb{F}\} \hspace{3mm} \forall X_i
$
\\ \vspace{2mm}
$
s_j^+ = s_j^+\lbrack n \rbrack := P :
$
\\ \vspace{2mm}
$
(P \lbrack X_i \rbrack \rightarrow y_{o} == a_{o} \hspace{3mm} \forall X_i) \hspace{2mm} \cap \hspace{2mm} (P[\hat{X}_i] \supseteq P[X_i] \hspace{3mm} \forall X_i,\hat{X}_i)
$
\\ \vspace{2mm}
$
S^+ := \{s^+_j,...\} \hspace{3mm} \forall j
$
\end{center}





\subsection{Definition of Solvable}
Define solvable
\begin{center}
$
X_i = \{x_1,...,x_n\}
$
\\ \vspace{2mm}
$
D := f \lbrack X_i \rbrack \rightarrow a_{o} \in \{\mathbb{T}, \mathbb{F}\} \hspace{3mm} \forall X_i
$
\\ \vspace{2mm}
$solvable : = solvable \lbrack D \rbrack \rightarrow b_o \in \{ \mathbb{T}, \mathbb{F} \} =$
\\ \vspace{2mm}
$\exists P : (P \lbrack X_i \rbrack \rightarrow y_{o} == a_{o} \hspace{3mm} \forall X_i) \hspace{2mm} \cap \hspace{2mm} (P[\hat{X}_i] \supseteq P[X_i] \hspace{3mm} \forall X_i,\hat{X}_i)
$
\end{center}





\section{The set of all Decision Problems $\mathbb{D}$}

\subsection{Definition}
Define the set of decision problems $\mathbb{D}$
\begin{center}
$
X_i = \{x_1,...,x_n\}
$
\\ \vspace{2mm}
$
D_j := f_j \lbrack X_i \rbrack \rightarrow a_{o} \in \{\mathbb{T}, \mathbb{F}\} \hspace{3mm} \forall X_i
$
\\ \vspace{2mm}
$\mathbb{D} := \{D_j,...\} \hspace{3mm} \forall j$
\end{center}





%\section{The Set of All Solutions to Decision Problems $\bold{S^+}$}

%\subsection{Definition}
%Define $\bold{S^+}$ the set of all solutions to decision problems
%\begin{center}
%$
%D_j \in \mathbb{D}
%$
%\\ \vspace{2mm}
%$
%D_j := f_j \lbrack X_i \rbrack \rightarrow a_{o} \in \{\mathbb{T}, \mathbb{F}\} \hspace{3mm} \forall X_i
%$
%\\ \vspace{2mm}
%$
%S^+_j := \{s^+_{1},s^+_{2},...\}
%$
%\\ \vspace{2mm}
%$
%\bold{S^+} := \{S^+_j,...\} \hspace{3mm} \forall j
%$
%\end{center}





\section{Instruction and Memory Notation}
Define $\mathcal{L}$ a set of logical operations\\
Define $\mathcal{M}$ a set of bits "memory"
\begin{center}
$
P \lbrack X_i \rbrack \rightarrow y_o = \{ s_1,s_2,...,s_{O_T \lbrack n \rbrack }, b_1, b_2,...,b_{O_S \lbrack n \rbrack},y_o \}
$
\\ \vspace{2mm}
$
\mathcal{L} := \{ s_1,s_2,...,s_{O_T \lbrack n \rbrack}\}
$
\\ \vspace{2mm}
$
\mathcal{M} := \{ b_1,b_2,...,b_{O_S \lbrack n \rbrack}\}
$
\\ \vspace{2mm}
$
P \lbrack X_i \rbrack = \{ \mathcal{L},\mathcal{M},y_o\}
$
\end{center}





















% Complexity; Time Complexity; Space Complexity; Total Complexity Dimension > 1
\section{Complexity}

\subsection{Time Complexity of a Decision Problem $O_T \lbrack n \rbrack$}
Define Time Complexity $O_T [n]$ of Decision Problem $D$ with solution $s^+$
\begin{center}
$
X_i = \{x_1,...,x_n\}
$
\\ \vspace{2mm}
$
D := f \lbrack X_i \rbrack \rightarrow a_{o} \in \{\mathbb{T}, \mathbb{F}\} \hspace{3mm} \forall X_i
$
\\ \vspace{2mm}
$
s^+ := P \lbrack X_i \rbrack \rightarrow y_{o} : y_o = a_{o} \hspace{3mm} \forall X_i = 
$
\\ \vspace{2mm}
$
\{ s_1,s_2,...,s_{O_T \lbrack n \rbrack }, b_1, b_2,...,b_{O_S \lbrack n \rbrack},y_o \} = \{ \mathcal{L},\mathcal{M},y_o\}
$
\\ \vspace{3mm}
$
O_T[n] := |\mathcal{L}| = N
$
\end{center}

\subsection{Space Complexity $O_S \lbrack n \rbrack$}
Define Space Complexity $O_S \lbrack n \rbrack$ of Decision Problem $D$ with solution $s^+$
\begin{center}
$
X_i = \{x_1,...,x_n\}
$
\\ \vspace{2mm}
$
D := f \lbrack X_i \rbrack \rightarrow a_{o} \in \{\mathbb{T}, \mathbb{F}\} \hspace{3mm} \forall X_i
$
\\ \vspace{2mm}
$
s^+ := P \lbrack X_i \rbrack \rightarrow y_{o} : y_o = a_{o} \hspace{3mm} \forall X_i =
$
\\ \vspace{2mm}
$
\{ s_1,s_2,...,s_{O_T \lbrack n \rbrack }, b_1, b_2,...,b_{O_S \lbrack n \rbrack},y_o \} = \{ \mathcal{L},\mathcal{M},y_o\}
$
\\ \vspace{2mm}
$
O_S[n] := |\mathcal{M}|+|y_o| = M
$
\end{center}





\section{Definition of Complexity}
Define Complexity $O[n]$ as a vector of dimension C
\begin{center}
$
\bold{O}[n] := \hspace{3mm} < O_T [n], O_S [n],O_3[n],O_4[n]...,O_C[n]>
$
\end{center}

\section{Total Complexity}
\begin{center}
$O[n] := O_T[n] + O_S[n] + \sum_{i=3}^{C} O_i[n]$
\end{center}
























% Simple Computational Complexity; Restate Time and Space Complexity; (Simple) Total Complexity
\newpage
\section{Simple Computational Complexity}
The remainder of this chapter assumes simple computational complexity of dimension 2

\subsection{Definition}
Define simple computational complexity of dimension 2
\begin{center}
$
\bold{O}[n] := \hspace{3mm} < O_T [n], O_S [n] >
$
\end{center}





\subsection{Time Complexity}
Restate definition of Time Complexity $O_T[n]$
\begin{center}
$
s^+ = \{ \mathcal{L},\mathcal{M},y_o\}
$
\\ \vspace{3mm}
$
O_T[n] := |\mathcal{L}| = N
$
\end{center}



\subsection{Space Complexity}
Restate definition of Time Complexity $O_S[n]$
\begin{center}
$
s^+ = \{ \mathcal{L},\mathcal{M},y_o\}
$
\\ \vspace{2mm}
$
O_S[n] := |\mathcal{M}| + |y_o| = M
$
\end{center}





\subsection{Total Complexity}
\begin{center}
$O[n] := O_T[n] + O_S[n]$
\\ \vspace{2mm}
$= |\mathcal{L}| + |\mathcal{M}| + |y_o| = N + M$
\end{center}


\subsection{$O_S[n] \neq 0$}
\subsubsection{Proof}
By definition of decision problem; Proof by contradiction; $y_o$ must be set to TF by definition; Suppose yo = 0; then yo is empty set; contradicts definition of D

\subsection{$O_T[n] \neq 0$}
\subsubsection{Proof}
By definition of decision problem; Proof by contradiction; $y_o$ must be set to TF by definition; Suppose |L| = 0; yo <- TF cap L is null by definition of empty set; implies yo emptyset (doesnt exist)

\subsection{O[n] = $O_T[n] + O_S[n] \neq$ 0}
\subsubsection{Proof}

\subsection{$O[n] > O_T[n]$}
\subsubsection{Proof}

\subsection{$O[n] > O_S[n]$}
\subsubsection{Proof}


\subsection{$O[n+1] \geq O[n]$}





% Optimal Complexity; Optimal Solutions; Optimal Time Solution; Optimal Space Solution
\newpage
\section{Optimal Complexity}

\subsection{Definition}
Define Optimal Complexity; the minimum total complexity required to solve a decision problem
\begin{center}
$O_{opt}[n] :=$
\\ \vspace{2mm}
$\not \exists \hat{O} \lbrack n \rbrack : \hat{O} \lbrack n \rbrack < O_{opt}[n] \hspace{2mm} \forall n$
\end{center}

\subsection{Proof of Existence}
Prove the existence of at least one $O_{min}[n]$ by induction/contradiction








\section{Optimal solution}
Define an optimal solution $s_{opt}^+$

\subsection{Definition}
\begin{center}
$
X_i = \{x_1,...,x_n\}
$
\\ \vspace{2mm}
$
D_j := f \lbrack X_i \rbrack \rightarrow a_{o} \in \{\mathbb{T}, \mathbb{F}\} \hspace{3mm} \forall X_i
$
\\ \vspace{2mm}
$
s^+ := P \lbrack X_i \rbrack \rightarrow y_{o} : y_o = a_{o} \hspace{3mm} \forall X_i
$
\\ \vspace{2mm}
$
s_{opt}^+ := s^+ :
$
\\ \vspace{2mm}
$
\not \exists \hat{O} \lbrack n \rbrack < O_{opt}[n] \hspace{3mm} \forall n, \hspace{1mm}  s^+ \in S_j^+
$
\end{center}





\subsection{Optimal Time Complexity Solution}
\begin{center}
$
X_i = \{x_1,...,x_n\}
$
\\ \vspace{2mm}
$
D_j := f \lbrack X_i \rbrack \rightarrow a_{o} \in \{\mathbb{T}, \mathbb{F}\} \hspace{3mm} \forall X_i
$
\\ \vspace{2mm}
$
s^+ := P \lbrack X_i \rbrack \rightarrow y_{o} : y_o = a_{o} \hspace{3mm} \forall X_i = 
$
\\ \vspace{2mm}
$
\{ s_1,s_2,...,s_{O_T \lbrack n \rbrack }, b_1, b_2,...,b_{O_S \lbrack n \rbrack},X_i,y_o \} = \{ \mathcal{L},\mathcal{M},X_i,y_o\}
$
\\ \vspace{3mm}
$
O_T[n] := |\mathcal{L}| = N
$
\\ \vspace{2mm}
$
s_{T}^+ := s^+ :
$
\\ \vspace{2mm}
$
\not \exists \hat{O_T} \lbrack n \rbrack < O_{T}[n] \hspace{3mm} \forall n, \hspace{1mm}  s^+ \in S_j^+
$
\end{center}




\subsection{Optimal Space Complexity Solution}

\begin{center}
$
X_i = \{x_1,...,x_n\}
$
\\ \vspace{2mm}
$
D_j := f \lbrack X_i \rbrack \rightarrow a_{o} \in \{\mathbb{T}, \mathbb{F}\} \hspace{3mm} \forall X_i
$
\\ \vspace{2mm}
$
s^+ := P \lbrack X_i \rbrack \rightarrow y_{o} : y_o = a_{o} \hspace{3mm} \forall X_i = 
$
\\ \vspace{2mm}
$
\{ s_1,s_2,...,s_{O_T \lbrack n \rbrack }, b_1, b_2,...,b_{O_S \lbrack n \rbrack},X_i,y_o \} = \{ \mathcal{L},\mathcal{M},X_i,y_o\}
$
\\ \vspace{3mm}
$
O_S[n] := |\mathcal{M}| = M
$
\\ \vspace{2mm}
$
s_{S}^+ := s^+ :
$
\\ \vspace{2mm}
$
\not \exists \hat{O_S} \lbrack n \rbrack < O_{S}[n] \hspace{3mm} \forall n, \hspace{1mm}  s^+ \in S_j^+
$
\end{center}







\subsection{Conjecture of Optimal Solutions}
$O_{T_{min}}$ subject to $O_{S_{opt}} = 1$
\begin{center}
$
\exists s^+_{opt} :
$
\\ \vspace{2mm}
$
O_{opt}[n] = 1 + O_T[n] = O_{S_{opt}} + O_T[n] \hspace{3mm} \forall s^+ \in S^+
$
\end{center}
\subsubsection{Proof}




\subsection{Theorem of Optimal Solutions}
Alternate way of expressing $O_{opt}[n]$ possibly with efficiency function and equivalence functions?\\
$f_{S \rightarrow T}[n,O_S[n]]$ as a function of space complexity order K?\\
Efficiency function ($f_{S \rightarrow T}[n,O_S[n]]$ as a function of space complexity order K) is strictly decreasing for Polynomial Functions\\
Or there's an inflection point\\
Efficiency function might have a general pattern for all problems in D








% Total Polynomial Complexity; Set of all Polynomial Problems; Polynomial Order of Complexity
\newpage
\section{Polynomial Complexity}

\subsection{Definition}
Decision problem $D$ with solution $s^+$ has polynomial total complexity $O[n]$ if
\begin{center}
$\exists K,C,\lambda_1...\lambda_K \hspace{2mm} :$
\\ \vspace{2mm}
$O[n] = (\lambda_K n)^K + (\lambda_{K-1} n)^{K-1}... + \lambda_1 n + C \hspace{3mm} \forall n$
\end{center}





\subsection{Polynomial Problems}
Define $\mathbb{P}$, the set of Decision Problems that can be solved with Polynomial Complexity
\begin{center}
$
\mathbb{P} := \{D_1,D_2,...\} : 
$
\\
$
\exists K,C,\lambda_1...\lambda_K : 
$
\\
$
O[n] = (\lambda_K n)^K + (\lambda_{K-1} n)^{K-1}... + \lambda_1 n + C \hspace{4mm} \forall n, D_i \in \mathbb{P}
$
\end{center}





\subsection{Polynomial Order of Complexity}
Total complexity $O[n]$ is said to be of order $K_{opt}$
\begin{center}
$
 O[n] \sim K_{opt}
$
\\ \vspace{2mm}
$O_{opt}\lbrack n \rbrack := O \lbrack n \rbrack :$
\\ \vspace{2mm}
$ \not \exists \hat{O} \lbrack n \rbrack < O_{opt}\lbrack n \rbrack \hspace{2mm} \forall n$
\\ \vspace{2mm}
$O_{opt}[n] < (\lambda_{K_{opt}} n)^{K_{opt}} + (\lambda_{K_{opt}-1} n)^{K_{opt}-1}... + \lambda_1 n +  C \hspace{4mm} \forall n$
\\ \vspace{2mm}
$K_{opt} := K :$
\\ \vspace{2mm}
$\not \exists \hat{K} : O_{T}[n] < (\lambda_{\hat{K}} n)^{\hat{K}} + (\lambda_{\hat{K}-1} n)^{\hat{K}-1}... + \lambda_1 n +  C \hspace{4mm} \forall n,\hspace{1mm} \hat{K} < K$
\end{center}








\subsection{Corrolary of Optimal Complexity}
\begin{center}
$\not \exists s^+ \in S^+ :$
\\ \vspace{2mm}
$O_{T}[n] < (\lambda_{\hat{K}} n)^{\hat{K}} + (\lambda_{\hat{K}-1} n)^{\hat{K}-1}... + \lambda_1 n +  C \hspace{4mm} \forall n,\hspace{1mm} \hat{K} < K_{opt}$
\end{center}

\subsubsection{Proof}
Proof by contradiction; definition of optimal complexity






\subsection{Property of Polynomial  Complexity 1}
\begin{center}
$
lim_{n \rightarrow \infty} \frac{O[n+1]}{O[n]} = 1
$
\end{center}
\subsubsection{Proof WIP}
FIX!!!
Show there exists no constant satisfying the decreasing limit condition
\begin{center}
$
O[n] < (\lambda_K n)^K + (\lambda_{K-1} n)^{K-1}... + \lambda_1 n + C
$
\\ \vspace{2mm}
$
O[n+1] < (\lambda_K (n+1))^K + (\lambda_{K-1} (n+1))^{K-1}... + \lambda_1 (n+1) + C
$
\\ \vspace{2mm}
$
O[n] \sim (\lambda n)^K; \hspace{2mm} O[n+1] \sim  (\lambda n)^K
$
\\ \vspace{2mm}
$
lim_{n \rightarrow \infty} \frac{(\lambda n)^K}{(\lambda n)^K} = 1
$
\end{center}







\subsection{Property of Polynomial Complexity 2}
\begin{center}
$
\exists K,C,\lambda_1,...,\lambda_K :
$
\\ \vspace{2mm}
$
( O[n+1] - O[n] ) = f_{n+1}[n] = (\lambda_K n)^K + (\lambda_{K-1} n)^{K-1}... + \lambda_1 n + C \hspace{3mm} \forall n
$
\end{center}
\subsubsection{Proof FIX!!!}
\begin{center}
$
O[n+1] < (\lambda_K (n+1))^K + (\lambda_{K-1} (n+1))^{K-1}... + \lambda_1 (n+1) + C
$
\\ \vspace{2mm}
$
O[n] < (\lambda_K n)^K + (\lambda_{K-1} n)^{K-1}... + \lambda_1 n + C
$

\end{center}







\subsection{Total Polynomial Complexity Implies Time bounded Polynomial Complexity}
\begin{center}
\vspace{1mm}
$
D \in \mathbb{P} \Longrightarrow O_T[n] < ...
$
\end{center}

\subsubsection{Proof FIX!!!}
\begin{center}
$
O[n] < (\lambda_K n)^K + (\lambda_{K-1} n)^{K-1}... + \lambda_1 n + C \hspace{2mm} \forall n
$
\\ \vspace{2mm}
$
O[n] := O_T[n] + O_S[n]; \hspace{2mm} O_T[n] < O[n]
$
\\ \vspace{2mm}
$
\therefore O_T[n] < (\lambda_K n)^K + (\lambda_{K-1} n)^{K-1}... + \lambda_1 n + C \hspace{2mm} \forall n
$
\end{center}










\subsection{Total Polynomial Complexity Implies Space bounded Polynomial Complexity}
\begin{center}
\vspace{1mm}
$
D \in \mathbb{P} \Longrightarrow O_S[n] < ...
$
\end{center}

\subsubsection{Proof FIX!!!}
\begin{center}
$
O[n] < (\lambda_K n)^K + (\lambda_{K-1} n)^{K-1}... + \lambda_1 n + C \hspace{2mm} \forall n
$
\\ \vspace{2mm}
$
O[n] := O_T[n] + O_S[n]; \hspace{2mm} O_S[n] < O[n]
$
\\ \vspace{2mm}
$
\therefore O_S[n] < (\lambda_K n)^K + (\lambda_{K-1} n)^{K-1}... + \lambda_1 n + C \hspace{2mm} \forall n
$
\end{center}










\subsection{Total Polynomial Complexity iff Time and Space bounded by Polynomial Complexity}
Use limit definition













\subsection{Order of Complexity}
ERROR in second condition\\
Total Complexity is said to be on the order of $K_{max}$
\begin{center}
$
O[n] \sim K_{max}
$
\\ \vspace{2mm}
$
K_{max} := K :
$
\\ \vspace{2mm}
$
O[n] < (\lambda_{K_{max}} n)^{K_{max}} + (\lambda_{K_{max}-1} n)^{K_{max}-1}... + \lambda_1 n + C \hspace{3mm} \forall n
$
\\ \vspace{2mm}
$
\not \exists O[n] < (\lambda_{\hat{K}_{max}} n)^{\hat{K}_{max}} + (\lambda_{\hat{K}_{max}-1} n)^{\hat{K}_{max}-1}... + \lambda_1 n + C \hspace{3mm} \forall n,\hat{K} < K_{max}
$
\end{center}





\subsection{Theorem Either OT or OS is on the order of Oopt}
Proof by contradiction








% Polynomial Complexity Time Based Proofs
\newpage
\section{Polynomial Time Complexity}

\subsection{Definition}
Decision problem $D$ with (optimal) Time Complexity $O_T[n]$ is bounded by polynomial time complexity if
\begin{center}
$\exists K,C,\lambda_1...\lambda_K \hspace{2mm} :$
\\ \vspace{2mm}
$O_T[n] < (\lambda_K n)^K + (\lambda_{K-1} n)^{K-1}... + \lambda_1 n + C \hspace{3mm} \forall n$
\end{center}








\subsection{Polynomial Time Solutions}
Define $\mathbb{S}^+_{time}$, the set of solutions that can be solved with polynomial time complexity
\begin{center}
$
\mathbb{S}^+_{time} := \{s^+_1,s^+_2,...\} : 
$
\\
$
\exists K,C,\lambda_1...\lambda_K : 
$
\\
$
O_T[n] < (\lambda_K n)^K + (\lambda_{K-1} n)^{K-1}... + \lambda_1 n + C \hspace{4mm} \forall n, s_i \in \mathbb{S}^+_{time}
$
\end{center}








\subsection{Property of Polynomial Time Complexity 1}
\begin{center}
$
lim_{n \rightarrow \infty} \frac{O_T[n+1]}{O_T[n]} = 1
$
\end{center}
\subsubsection{Proof}


\subsection{Property of Polynomial Time Complexity 2}
\begin{center}
$
\exists K,C,\lambda_1,...,\lambda_K :
$
\\ \vspace{2mm}
$
(O_T[n+1] - O_T[n]) <  (\lambda_K n)^K + (\lambda_{K-1} n)^{K-1}... + \lambda_1 n + C \hspace{2mm} \forall n
$
\end{center}
\subsubsection{Proof}




\subsection{Order of Complexity}
Time complexity $O_T[n]$ is said to be on the order of $K_{max}$
\begin{center}
$
O_T[n] < (\lambda_{K_{max}} n)^{K_{max}} + (\lambda_{K_{max}-1} n)^{K_{max}-1}... + \lambda_1 n + C
$
\\ \vspace{2mm}
$
O_T[n] \sim K_{max}
$
\end{center}


\subsection{Proof of the existence of $O_{T_{opt}}$}
















% Polynomial Space Complexity
\newpage
\section{Polynomial Space Complexity}


% Definition of Polynomial Space Complexity
\subsection{Defintion}
Decision problem $D$ with (optimal) Time Complexity $O_S[n]$ is bounded by polynomial time complexity if
\begin{center}
$\exists K,C,\lambda_1...\lambda_K \hspace{2mm} :$
\\ \vspace{2mm}
$O_S[n] < (\lambda_K n)^K + (\lambda_{K-1} n)^{K-1}... + \lambda_1 n + C \hspace{3mm} \forall n$
\end{center}







\subsection{Polynomial Space Problems}
Define $\mathbb{S}^+_{space}$, the set of solutions that can be solved with polynomial time complexity
\begin{center}
$
\mathbb{S}^+_{space} := \{s_1,s_2,...\} : 
$
\\
$
\exists K,C,\lambda_1...\lambda_K : 
$
\\
$
O_S[n] < (\lambda_K n)^K + (\lambda_{K-1} n)^{K-1}... + \lambda_1 n + C \hspace{4mm} \forall n, s_i \in \mathbb{S}^+_{time}
$
\end{center}







\subsection{Total Polynomial Complexity Implies Space bounded Polynomial Complexity}
\subsection{Space Bounded Polynomial Complexity Implies Total Polynomial Complexity}
\subsection{Polynomial Space Complexity iff Polynomial Complexity}








\subsection{Property of Polynomial Space Complexity 1}
\begin{center}
$
lim_{n \rightarrow \infty} \frac{O_S[n+1]}{O_S[n]} = 1
$
\end{center}
\subsubsection{Proof}







\subsection{Property of Polynomial Space Complexity 2}
\begin{center}
$
\exists K,C,\lambda_1,...,\lambda_K :
$
\\ \vspace{2mm}
$
( O_S[n+1] - O_S[n] ) < (\lambda_K n)^K + (\lambda_{K-1} n)^{K-1}... + \lambda_1 n + C \hspace{2mm} \forall n \hspace{2mm} \forall n
$
\end{center}
\subsubsection{Proof}








\subsection{Order of Complexity}
Space complexity $O_S[n]$ is said to be on the order of $K_{max}$
\begin{center}
$
O_S[n] < (\lambda_{K_{max}} n)^{K_{max}} + (\lambda_{K_{max}-1} n)^{K_{max}-1}... + \lambda_1 n + C
$
\\ \vspace{2mm}
$
O_S[n] \sim K_{max}
$
\end{center}




\subsection{Proof of the existence of $O_{S_{opt}}$}












% Theorem of Computational Duality; Polynomial in time and space
\newpage
\section{Inductive Functions}














% Inductive Function
\subsection{Inductive Function $f_{n+1}$}
\begin{center}
\vspace{2mm}
$
O[n] := O_T[n] + O_S[n]
$
\\ \vspace{2mm}
$
O[n+1] = O_T[n+1] + O_S[n+1]
$
\\ \vspace{4mm}
$
f_{n+1}[n] := f[n] :
$
\\ \vspace{2mm}
$
O[n+1] = f[n] + O[n] \hspace{3mm} \forall n
$
\end{center}

\subsubsection{Proof of existence}
Algebraic Proof












\subsection{Inductive Space and Time Formulas}
\begin{center}
$
f^T_{n+1} := O_T[n+1] - O_T[n]
$
\\ \vspace{2mm}
$
O_T[n+1] = O_T[n] + f^T_{n+1}
$
\\ \vspace{2mm}
$
f^S_{n+1} := O_S[n+1] - O_S[n]
$
\\ \vspace{2mm}
$
O_S[n+1] = O_S[n] + f^S_{n+1}
$

\end{center}

\subsubsection{Proof of existence}
Algebraic Proof




% Theorem of Polynomia? Duality
\subsection{Inductive Function Expressions}
Relate $f_{n+1}[n]$ to equivalence functions
\begin{center}
$
D \in \mathbb{P}
$
\\ \vspace{2mm}
$
O[n] := O_T[n] + O_S[n]
$
\\ \vspace{2mm}
$
O[n+1] = O_T[n+1] + O_S[n+1] = O[n] + f_{n+1}[n]
$
\\ \vspace{2mm}
$
O_T[n] = O[n] - O_S[n]
$
\\ \vspace{2mm}
$
O_S[n] = O[n] - O_T[n]
$
\\ \vspace{8mm}
$
f_{n+1} = O[n+1] - O[n]
$
\\ \vspace{2mm}
$
f_{n+1} = O_T[n+1] + O_S[n+1] - O[n]
$
\\ \vspace{2mm}
$
f_{n+1} = O_T[n+1] - O_T[n] + O_S[n+1] - O_S[n]
$
\\ \vspace{2mm}
$
f_{n+1} = O[n+1] - O_T[n] - O_S[n]
$
\\ \vspace{2mm}
$
f_{n+1}[n] =  f^T_{n+1}[n] +  f^S_{n+1}[n]
$
\end{center}






\subsection{Zero Order Inductive Function}
\begin{center}
$
Let \hspace{2mm} O_S[n] \sim n^0
$
\\ \vspace{2mm}
$
f_{n+1} = O_T[n+1] - O_T[n] + O_S[n+1] - O_S[n] = O_T[n+1] - O_T[n]
$
\end{center}


\subsection{Property of Polynomial Complexity}
$f_{n+1}[n]$ has order less than O[n]\\
$f_{n+1}[n]$ is bound by $K_{max}$ - 1
\subsubsection{Proof}
Proof by contradiction; limit doesn't converge














% Equivalence Functions of OT and OS
\newpage
\section{Duality of $O_T[n]$, $O_S[n]$?}
\subsubsection{$O_T$ to $O_S$}
Define equivalence function $f_{T \rightarrow S}$; a function converting logical operations into memory elements
\begin{center}
$
f_{T \rightarrow S} := f :
$
\\ \vspace{2mm}
$
O_S[n] = f[n,O_T[n]] \hspace{3mm} \forall n, s^+ \in S^+
$
\end{center}

\subsubsection{$O_S$ to $O_T$}
Define equivalence function $f_{S \rightarrow T}$; a function converting memeory elements into logical operations
\begin{center}
$
f_{S \rightarrow T} := f :
$
\\ \vspace{2mm}
$
O_T[n] = f[n,O_S[n]] \hspace{3mm} \forall n, s^+ \in S^+
$
\end{center}


\subsubsection{Invertibility?}
\subsubsection{Polynomial Bounded?}




\subsection{Efficiency Function?}
Function relating the decrease in O[n] as $O_S[n]$ increases in order\\
$f_{S \rightarrow T}[n,O_S[n]]$ as a function of space complexity order K?


\section{Theorem of Computational Duality?}
For all Problems in P there exists a duality function\\
Formally define dynamic programming, Optimal polynomial complexity minimizes the difference between time and space complexity order
\begin{center}
\vspace{2mm}
$
D \in \mathbb{P}
$
\\ \vspace{2mm}
$
O[n] := O_T[n] + O_S[n]
$
\\ \vspace{2mm}
$
limit_{n \rightarrow \infty} \frac{O[n+1]}{O[n]} = 1 \hspace{3mm} \forall s^+ \in S_{\mathbb{P}}^+
$
\\ \vspace{7mm}
$
O_T[n] = f_{S \rightarrow T}[n,O_S[n]]
$
\\ \vspace{2mm}
$
O_S[n] = f_{T \rightarrow S}[n,O_T[n]]
$
\\ \vspace{7mm}
$
limit_{n \rightarrow \infty} \frac{O_T[n+1] + O_S[n+1]}{O_T[n] + O_S[n]} = 1 \hspace{3mm} \forall s^+ \in S_{\mathbb{P}}^+
$
\\ \vspace{7mm}
$
limit_{n \rightarrow \infty} \frac{ f_{S \rightarrow T}[n+1,O_S[n+1]] + O_S[n+1]}{ f_{S \rightarrow T}[n,O_S[n]] + O_S[n]} = 1 \hspace{3mm} \forall s^+ \in S_{\mathbb{P}}^+
$
\\ \vspace{3mm}
$
limit_{n \rightarrow \infty} \frac{O_T[n+1] + f_{T \rightarrow S}[n+1,O_T[n+1]]}{O_T[n] + f_{T \rightarrow S}[n,O_T[n]]} = 1 \hspace{3mm} \forall s^+ \in S_{\mathbb{P}}^+
$
\end{center}


\newpage
\section{Subfunctions}

\subsection{Restate the subfunction condition of solutions}
\subsection{Show O[n] is a strictly increasing function}
\subsection{Theorem of Polynomial Subfunctions}
\subsection{Theorem of Divergent Subfunctions}






% N Sum Problem
\newpage
\section{Sum to N Problem with 2 integers}
% Zero Order Space Complexity Solution
\subsection{State formal definition of Sum to N : $x_i + x_j == N$}
\begin{center}
\vspace{1.5mm}
$
X_i = \{x_1,...,x_n,N\}
$
\\ \vspace{2mm}
$
D := f \lbrack X_i \rbrack \rightarrow a_{o} \in \{\mathbb{T}, \mathbb{F}\} \hspace{3mm} \forall X_i
$
\\ \vspace{2mm}
$
s^+ = P :
$
\\ \vspace{2mm}
$
(P \lbrack X_i \rbrack \rightarrow y_{o} == a_{o} \hspace{3mm} \forall X_i) \hspace{2mm} \cap \hspace{2mm} (P[\hat{X}_i] \supseteq P[X_i] \hspace{3mm} \forall X_i,\hat{X}_i)
$
\\ \vspace{2mm}
$
s^+ = \{ s_1,s_2,...,s_{O_T \lbrack n \rbrack }, b_1, b_2,...,b_{O_S \lbrack n \rbrack},y_o \} = \{ \mathcal{L},\mathcal{M},y_o\}
$
\\ \vspace{6mm}
$
D = f[X_i] = \exists x_j,x_k \in X_i : x_j + x_k == N
$
\end{center}



\subsection{Express a formal solution : $O_S[n] \sim n^0$}
\begin{center}
\vspace{1.5mm}
$
s^+ = \{ s_1,s_2,...,s_{O_T \lbrack n \rbrack }, b_1, b_2,...,b_{O_S \lbrack n \rbrack},y_o \} = \{ \mathcal{L},\mathcal{M},y_o\}
$
\\ \vspace{2mm}
$
s_1 = y_o \leftarrow \mathbb{F};
$
\\ \vspace{2mm}
$
\forall i,j > i
$
\\ \vspace{2mm}
$
s_2,s_3,s_8,s_9,...,s_{3ij-4},s_{3ij-3}...,s_{3n(n-1)-4},s_{3n(n-1)-3} =  b_1 \leftarrow x_i + x_j 
$
\\ \vspace{2mm}
$
s_4,s_5,s_{10},s_{11},...,s_{3ij-2},s_{3ij-1}...,s_{3n(n-1)-2},s_{3n(n-1)-1} = b_1 \leftarrow b_1 == N
$
\\ \vspace{2mm}
$
s_6,s_7,s_{12},s_{13}...,s_{3ij},s_{3ij+1}...,s_{3n(n-1)},s_{3n(n-1)+1} = y_o \leftarrow y_o \lor b_1
$
\\ \vspace{2mm}
$
s^+ = \{y_o \leftarrow \mathbb{F},y_o \leftarrow y_o \lor (x_i + x_j == N) \hspace{3mm} \forall i,j > i \hspace{1mm}| \hspace{1mm} b_1,y_o\}
$
\end{center}



\subsection{Show $s^+$ satisfies the subfunction condition of solutions: $P[\hat{X}_i] \supseteq P[X_i] \hspace{3mm} \forall \hat{X}_i,X_i$}
\begin{center}
\vspace{2mm}
$
X_i = \{x_1,...,x_n,N\}; \hspace{2mm} \hat{X}_i = \{x_1,...,x_n,x_{n+1},N\}
$
\\ \vspace{2mm}
$
s^+ = \{ s_1,s_2,...,s_{O_T \lbrack n \rbrack }, b_1, b_2,...,b_{O_S \lbrack n \rbrack},y_o \} = \{ \mathcal{L},\mathcal{M},y_o\}
$
\\ \vspace{2mm}
$
s_{n+1}^+ = s^+ \cup \hat{s}^+ 
$
\\ \vspace{2mm}
$
s_1 = y_o \leftarrow \mathbb{F};
$
\\ \vspace{2mm}
$
\forall i,j > i
$
\\ \vspace{2mm}
$
s_2,s_3,s_8,s_9,...,s_{3ij-4},s_{3ij-3}...,s_{3n(n-1)-4},s_{3n(n-1)-3} =  b_1 \leftarrow x_i + x_j 
$
\\ \vspace{2mm}
$
s_4,s_5,s_{10},s_{11},...,s_{3ij-2},s_{3ij-1}...,s_{3n(n-1)-2},s_{3n(n-1)-1} = b_1 \leftarrow b_1 == N
$
\\ \vspace{2mm}
$
s_6,s_7,s_{12},s_{13}...,s_{3ij},s_{3ij+1}...,s_{3n(n-1)},s_{3n(n-1)+1} = y_o \leftarrow y_o \lor b_1
$
\\ \vspace{2mm}
$
\forall k
$
\\ \vspace{2mm}
$
s... = b_1 \leftarrow x_k + x_{n+1}
$
\\ \vspace{2mm}
$
s... = b_1 \leftarrow b_1 == N
$
\\ \vspace{2mm}
$
s... = y_o \leftarrow y_o \lor b_1
$
\\ \vspace{8mm}
$
s^+ = \{y_o \leftarrow \mathbb{F},y_o \leftarrow y_o \lor (x_i + x_j == N) \hspace{3mm} \forall i,j > i \hspace{1mm}| \hspace{1mm} b_1,y_o\}
$
\\ \vspace{4mm}
$
\hat{s}^+ = \{y_o \leftarrow y_o \lor (x_k + x_{n+1} == N) \hspace{2mm} \forall k < n+1 | \hspace{1mm} b_1,y_o \}
$
\\ \vspace{4mm}
$
s^+_{n+1} = \{y_o \leftarrow \mathbb{F},y_o \leftarrow y_o \lor (x_i + x_j == N) \hspace{3mm} \forall i,j > i \hspace{1mm}| \hspace{1mm} b_1,y_o\} \hspace{2mm}\cup 
$
\\ \vspace{3mm}
$
\{y_o \leftarrow y_o \lor (x_k + x_{n+1} == N) \hspace{2mm} \forall k < n+1 | \hspace{1mm} b_1,y_o \}
$
\\ \vspace{6mm}
$
s_{n+1}^+ = P[\hat{X}_i] \supseteq P[X_i] = s^+
$
\end{center}











\subsection{Determine $O[n], O_S[n], O_T[n],\hat{O}[n],\hat{O}_T[n],\hat{O}_S[n]$ for the above solution}
\begin{center}
$
O_S[n] = |y_o| + |b_1| = 2
$
\\ \vspace{2mm}
$
O_T[n] = 3n(n-1) + 1 = 3n(n-1) -1 + O_S[n]
$
\\ \vspace{2mm}
$
O[n] = 3n(n-1) + 3 = 3n^2 -3n + 3
$
\\ \vspace{4mm}
$
\hat{O}_S[n] = 0
$
\\ \vspace{2mm}
$
\hat{O}_T[n] = 6n
$
\\ \vspace{2mm}
$
\hat{O}[n] = \hat{O}_S[n] + \hat{O}_T[n]
$
\end{center}


\subsection{Verify $O[n+1]$ = $O[n]$ + $\hat{O}[n]$}
\begin{center}
$
O[n+1] = O[n] + \hat{O}[n]
$
\\ \vspace{2mm}
$
3(n+1)^2 -3(n+1) + 3 =  3n^2 -3n + 3 + 6n
$
\\ \vspace{2mm}
$
3n^2 + 6n + 3 - 3n -3 + 3 = 3n^2 + 3n + 3
$
\\ \vspace{2mm}
$
3n^2 + 3n+ 3 = 3n^2 + 3n + 3
$
\end{center}







\subsection{Show $s^+$ has Polynomial Complexity by the definition of Total Polynomial Complexity}
\begin{center}
\vspace{2mm}
$
O[n] = 3n^2 -3n + 3
$
\end{center}

\subsection{Show $s^+$ has Polynomial Complexity by showing limit$_{n \rightarrow \infty}\frac{O[n+1]}{O[n]}$ = 1}
\begin{center}
$
limit_{n \rightarrow \infty}\frac{O[n+1]}{O[n]} =
$
\\ \vspace{2mm}
$
limit_{n \rightarrow \infty}\frac{3n^2 + 3n + 3}{3n^2 - 3n + 3} =
$
\\ \vspace{2mm}
$
limit_{n \rightarrow \infty}(\frac{3n^2 - 3n + 3}{3n^2 - 3n + 3} + \frac{6n}{3n^2 - 3n + 3}) =
$
\\ \vspace{2mm}
$
limit_{n \rightarrow \infty}(1 + \frac{6n }{3n^2 - 3n + 3}) = 1
$
\end{center}















% Divergent Problems
\newpage
\section{Divergent Problems}
\subsection{Definition}
\begin{center}
$
\mathcal{\hat{D}} := \{ \hat{D}_1,\hat{D}_2,...\} :
$
\\ \vspace{2mm}
$
lim_{n \rightarrow \infty} \frac{O[n+1]}{O[n]} \hspace{2mm} diverges \hspace{3mm} \forall \hat{D} \in \mathcal{\hat{D}}
$
\end{center}



\subsection{Theorem of Divergent Subfunctions}
If an (inductive) subfunction of $s^+$ diverges, the solution is divergent
\begin{center}
$
f_{n+1} = \sum g_{n+1}
$
\\ \vspace{2mm}
$
limit \frac{g[n+1]}{g[n]} diverges
$
\\ \vspace{3mm}
$
\exists g_{n+1} : limit \frac{g_{n+1}}{O[n]} diverges \Longrightarrow limit \frac{O[n+1]}{O[n]} diverges
$
\end{center}

\subsection{Proof}



\subsection{The Set of Polynomial Solutions and the Set of Divergent Solutions are disjoint}
\begin{center}
\vspace{2mm}
$
\mathbb{P} \cap \hat{D} = \emptyset
$
\end{center}

\subsection{Proof}
Proof by contradiction; Let $s^+ \in \mathbb{P}, \hat{D}; s^+ \in \mathbb{P} \cap \hat{D}$
\begin{center}
$
X_i = \{x_1,...,x_n\}
$
\\ \vspace{2mm}
$
D := f \lbrack X_i \rbrack \rightarrow a_{o} \in \{\mathbb{T}, \mathbb{F}\} \hspace{3mm} \forall X_i
$
\\ \vspace{6mm}
$
Let \hspace{2mm} D \in \mathbb{P}
$
\\ \vspace{2mm}
$
s^+ := P \lbrack X_i \rbrack \rightarrow y_{o} : y_o = a_{o} \hspace{3mm} \forall X_i
$
\\ \vspace{2mm}
$
O[n] = O_T[n] + O_S[n] < (\lambda_K n)^K + (\lambda_{K-1} n)^{K-1}... + \lambda_1 n + C \hspace{3mm} \forall n
$
\end{center}








% Proof of Non-Polynomial Problems Traveling Salesman
\newpage
\section*{Traveling Salesman Problem of Dimension 2}
\section{Proof of the existence of $\hat{\mathcal{D}}$}

\subsection{The Traveling Salesman Problem of Dimension 2}
English description

\subsection{Formal Definition}
\begin{center}
$
X_i = \{l_1,l_2,...,l_n,C\}
$
\\ \vspace{5mm}
$
l_i = \{x_i,y_i\} \hspace{2mm} \forall i
$
\\ \vspace{2mm}
$l_i$ denotes the 2D coordinates of location i
\\ \vspace{5mm}
$
C = \{d_{proposed},p_{decimal}\}
$
\\ \vspace{2mm}
$
d_{proposed}$ denotes the suggested shortest distance
\\ \vspace{1.5mm}
$
p_{decimal}$ is the decimal precision
\\ \vspace{5mm}
$
L\lbrack l_i,l_j \rbrack := \sqrt{(y_j - y_i)^2 + (x_j - x_i)^2}
$
\\ \vspace{2mm}
Let $L\lbrack l_i,l_j \rbrack$ denote the distance between location $l_i$ and $l_j$
\\ \vspace{5mm}
$
\tilde{L} \lbrack l_i,l_j \rbrack :=d_{trunc} : -p_{decimal} < d_{trunc} - L\lbrack l_i,l_j \rbrack < p_{decimal}
$
\\ \vspace{2mm}
Let $\tilde{L}\lbrack l_i,l_j \rbrack$ denote a truncated decimal representation of $L\lbrack l_i,l_j \rbrack$
\\ \vspace{5mm}
$
R_i := \{r_1,r_2,...,r_n,r_1\} \hspace{1.5mm} : \hspace{1.5mm} r_i \in X_i \hspace{2mm} \forall i; \hspace{2mm} r_i \neq r_j
$
\\ \vspace{2mm}
Let $R_i$ denote route $i$
\\ \vspace{5mm}
$
L_{Total}\lbrack R_i \rbrack := (\sum_{i=1}^{n-1} \tilde{L}\lbrack r_i,r_{i+1}\rbrack) + \tilde{L}\lbrack r_n,r_1\rbrack
$
\\ \vspace{2mm}
Let $L_{Total}\lbrack R_i \rbrack$ denote the sum of truncated lengths of route $R_i$
\\ \vspace{5mm}
$
D := f \lbrack X_i \rbrack \rightarrow a_o \in \{\mathbb{T},\mathbb{F}\} \hspace{2mm} \forall X_i
$
\\ \vspace{2mm}
$
a_o =
$
\\ \vspace{2mm}
$
(\exists R_k : L_{total}\lbrack R_k\rbrack == d_{proposed}) \cap (\not \exists R_j :  L_{total}\lbrack R_j \rbrack < d_{proposed})
$
\end{center}





\newpage
\section*{Traveling Salesman Problem of Dimension 2}
\subsection{Define subpath, subpath distance, subpath storage}
\vspace{2mm}
$\tilde{L} \lbrack l_i,l_j \rbrack$ denotes "the distance of a subpath of length 1"
\begin{center}
$
\tilde{L} \lbrack l_i,l_j \rbrack :=d_{trunc} : -p_{decimal} < d_{trunc} - L\lbrack l_i,l_j \rbrack < p_{decimal} 
$
\\ \vspace{2mm}
$
= abs(d_{trunc} - L\lbrack l_i,l_j \rbrack) < p_{decimal} 
$
\end{center}
\vspace{4mm}
$
\tilde{R}$ denotes a subpath of length k
\begin{center}
$
\tilde{R} = \{\tilde{r}_1,\tilde{r}_2,...,\tilde{r}_k\} \hspace{1.5mm} : \hspace{1.5mm} \tilde{r}_i \in X_i \hspace{2mm} \forall i, r_i \neq r_j
$
\end{center}
\vspace{4mm}
$\tilde{L}_k\lbrack \tilde{R} \rbrack$ denotes "the distance of a subpath of length k"
\begin{center}
$
\tilde{L}_k\lbrack \tilde{R} \rbrack := \sum_{i=1}^{k} \tilde{L}\lbrack \tilde{r},\tilde{r}_{i+1}\rbrack
$ 
\end{center}
\vspace{4mm}
Let $\mathcal{M}_1$ denote the memory reserved for subpaths distances of length 1
\begin{center}
$
\mathcal{M}_1= \{\hat{b}_{1;1},\hat{b}_{1;2},\hat{b}_{1;3},...,\hat{b}_{startindex;finishindex},...,\hat{b}_{n-1;n}\}^*
$
\\ \vspace{2mm}
$
\mathcal{M} \supseteq \mathcal{M}_1
$
\end{center}
\vspace{3mm}* Note $\hat{b}_{i;j} = \hat{b}_{j;i}$\\
$\sqrt{(y_j - y_i)^2 + (x_j - x_i)^2} = \sqrt{(y_i - y_j)^2 + (x_i - x_j)^2}$



% Traveling Salesman First Order Space Complexity Solution
\subsection{Define the following functions}
\subsubsection{$sqrt\lbrack x,p_{decimal} \rbrack = \sqrt{x}$\hspace{2mm}[1]}
\subsubsection{$pow\lbrack x,2,p_{decimal} \rbrack = x^2$ \hspace{1mm}[2]}


\subsection{Define the following subfunctions}
\subsubsection{loadM1Subpaths$\lbrack X\rbrack$}
\vspace{1mm}
// Computes all subpaths of length 1 and stores in $\mathcal{M}_1 = \{\hat{b}_{1;1},\hat{b}_{1,;2},...,\hat{b}_{n-1;n}\}$
\begin{center}
\vspace{2mm}
$
// X_i = \{l_1,l_2,...,l_n,C\}
$
\\ \vspace{2mm}
$
//\hspace{.5mm} l_i = \{x_i,y_i\} \hspace{2mm} \forall i
$
\\ \vspace{5mm}
$
// \mathcal{M} = \{b_1,b_2,...,b_M,\hat{b}_{1;1},\hat{b}_{1,;2},...,\hat{b}_{n-1;n},y_o\} = \{b_1,b_2,...,b_M,\mathcal{M}_1,y_o\} = \{\mathcal{M},\mathcal{M}_1,y_o\}
$
\\ \vspace{5mm}
$
\forall i,j > i
$
\\ \vspace{2mm}
$
b_3 \leftarrow y_i - y_j
$
\\ \vspace{2mm}
$
b_4 \leftarrow x_i - x_j
$
\\ \vspace{2mm}
$
b_3 \leftarrow b_3^2
$
\\ \vspace{2mm}
$
b_4 \leftarrow b_4^2
$
\\ \vspace{2mm}
$
b_3 \leftarrow b_3 + b_4
$
\\ \vspace{2mm}
$
\hat{b}_{i;j}\leftarrow \sqrt{b_3}^*
$
\end{center}
\vspace{4mm}
$^* \hat{b}_{i;j} = \tilde{L}\lbrack l_i,l_j \rbrack$



\subsubsection{computeAllRoutes$\lbrack X \rbrack$}
\vspace{1mm} // Computes all complete routes, checks for a route == $d_{proposed}$, sets $y_o$ to false if the current route is shorter than $d_{proposed}$
\begin{center}
$
\forall i,j \neq i, k \neq i,j ...\hspace{.5mm},q \neq i,j,...,m
$
\\ \vspace{2mm}
$
b_3 \leftarrow \hat{b}_{1;j}+ \hat{b}_{j;k}
$
\\ \vspace{2mm}
$
b_3 \leftarrow b_3 + \hat{b}_{k;l}
$
\\ \vspace{2mm}
$
...
$
\\ \vspace{2mm}
$
b_3 \leftarrow b_3 + \hat{b}_{m;q}
$
\\ \vspace{2mm}
$
b_3 \leftarrow b_3 + \hat{b}_{q;1}
$
\\ \vspace{2mm}
$
b_4 \leftarrow b_3 == b_2
$
\\ \vspace{2mm}
$
b_1 \leftarrow b_1 \lor b_4
$
\\ \vspace{2mm}
$
b_4 \leftarrow b_2 \leq b_3
$
\\ \vspace{2mm}
$
y_o \leftarrow y_o \land b_4
$
\end{center}




\subsection{Express a solution using subfunctions, storing subpaths of length 1 in memory}
\begin{center}
\vspace{4mm}
// $d_{proposed}$ is the shortest path\\
$y_o \leftarrow \mathbb{T}$
\\ \vspace{3mm}
//$d_{proposed}$ exists as a path\\
$ b_1 \leftarrow \mathbb{F}$
\\ \vspace{3mm}
// shortest path register\\
$ b_2 \leftarrow d_{proposed}$
\\ \vspace{3mm}
$loadM1Subpaths\lbrack X\rbrack$
\\ \vspace{1mm}
$computeAllRoutes\lbrack X \rbrack$
\end{center}












\subsection{Show each subfunction satisfies the subfunction condition of solutions $: P[\hat{X}_i] \supseteq P[X_i] \hspace{3mm} \forall \hat{X}_i,X_i, \hspace{2mm} \hat{X}_i \supseteq X_i$}
Let
\begin{center}
$
\mathcal{M}_0 = \{b_1,b_2,b_3,b_4,y_o\}
$
\\ \vspace{2mm}
$
\mathcal{M}_1 = \{\hat{b}_{1;1},\hat{b}_{1,;2},...,\hat{b}_{n-1;n}\} 
$
\end{center}

\subsubsection{$loadM1Subpaths\lbrack X\rbrack \rightarrow \mathcal{M}_1$}
Let
\begin{center}
$
// X = \{l_1,l_2,...,l_n,C\}; \hspace{2mm} \hat{X} = \{l_1,l_2,...,l_n,l_{n+1},C\}
$
\\ \vspace{4mm}
$
loadM1Subpaths\lbrack X, \mathcal{M}\rbrack \rightarrow \mathcal{M}_1 = Sub_1\lbrack X, \mathcal{M} \rbrack \rightarrow \mathcal{M}_1
$
\\ \vspace{8mm}
$
Sub_1 \lbrack X, \mathcal{M} \rbrack = \{ \mathcal{L},\mathcal{M}\}
$
\\ \vspace{2mm}
$
= \{ \hat{b}_{i;j} \leftarrow \tilde{L}\lbrack l_i,l_j \rbrack \hspace{2mm} \forall i,j>i | b_3,b_4,\hat{b}_{1;1},\hat{b}_{1;2},...,\hat{b}_{n-1;n}\}
$
\\ \vspace{3mm}
$
Sub_1 \lbrack X_i, \mathcal{M} \rbrack = \{ \hat{b}_{i;j} \leftarrow \tilde{L}\lbrack l_i,l_j \rbrack \hspace{2mm} \forall i,j>i | b_3,b_4,\mathcal{M}_1\}
$
\\ \vspace{8mm}
$
Sub_1 \lbrack \hat{X}, \mathcal{M} \rbrack = \{ \hat{\mathcal{L}},\hat{\mathcal{M}}\}
$
\\ \vspace{2mm}
$
= \{ \hat{b}_{i;j} \leftarrow \tilde{L}\lbrack l_i,l_j \rbrack \hspace{2mm} \forall i,j>i | b_3,b_4,\hat{b}_{1;1},\hat{b}_{1;2},...,\hat{b}_{n;n+1}\}
$
\\ \vspace{2mm}
$
= \{ \mathcal{L}, \hat{b}_{i;j} \leftarrow \tilde{L}\lbrack l_i,l_j \rbrack \hspace{2mm} \forall i,j=n+1 | \mathcal{M}, \hat{b}_{1;n+1},\hat{b}_{2;n+1},...,\hat{b}_{n;n+1}\}
$
\\ \vspace{2mm}
$
Sub_1 \lbrack \hat{X}, \mathcal{M} \rbrack = \{ \mathcal{L}, \mathcal{L}_{n+1} | \mathcal{M}, \mathcal{M}_{n+1}\}
$
\\ \vspace{3mm}
$
Sub_1 \lbrack \hat{X}, \mathcal{M} \rbrack = \{ \mathcal{L}, \mathcal{L}_{n+1} | \mathcal{M}, \mathcal{M}_{n+1}\} \supseteq \{ \mathcal{L} | \mathcal{M}\} = Sub_1 \lbrack X, \mathcal{M} \rbrack 
$
\end{center}




\subsubsection{$computeAllRoutes\lbrack X\rbrack$}
Let
\begin{center}
$
computeAllRoutes\lbrack X\rbrack = Sub_2\lbrack X \rbrack
$
\\ \vspace{2mm}
$
Sub_2 \lbrack X \rbrack = \{ \mathcal{L},\mathcal{M},y_o\}
$
\\ \vspace{2mm}
$
= \{\hat{b}_{1;i_2}+\hat{b}_{i_2;i_3}+\hat{b}_{i_3;i_4}+...+\hat{b}_{i_{n};1} \hspace{2mm} \forall i_2,i_3 \neq i_2, i_4 \neq i_2,i_3 ... i_n \neq i_2,i_3...,i_{n-1} \hspace{2mm}
$
\\ \vspace{1mm}
$
| b_1,b_2,b_3,b_4,\mathcal{M}_1,y_o \}
$
\\ \vspace{6mm}
$
Sub_2 \lbrack\hat{X} \rbrack = \{ \hat{\mathcal{L}},\hat{\mathcal{M}},y_o\}
$
\\ \vspace{2mm}
$
= \{\hat{b}_{1;i_2}+\hat{b}_{i_2;i_3}+\hat{b}_{i_3;i_4}+...+\hat{b}_{i_{n+1};1} \hspace{2mm} \forall i_2,i_3 \neq i_2, i_4 \neq i_2,i_3 ... i_{n+1} \neq i_2,i_3...,i_{n} \hspace{2mm}
$
\\ \vspace{1mm}
$
| \mathcal{M},\mathcal{M}_{n+1},y_o \}
$
\end{center}
\vspace{6mm}Let
\begin{center}
$
insert\_subpath\lbrack \mathcal{L} \rbrack =
$
\\ \vspace{8mm}
$
Sub_2 \lbrack\hat{X} \rbrack =  \{ insert\_subpath\lbrack \mathcal{L},\hat{b}_{i_{n+1};j},j \rbrack \hspace{2mm} \forall j \neq n+1 | \mathcal{M},\mathcal{M}_{n+1},y_o \}
$
\end{center}




\subsubsection{Show the overall solution storing subpaths of length 1 satisfies the subfunction condition of solutions $: P[\hat{X}_i] \supseteq P[X_i] \hspace{3mm} \forall \hat{X}_i,X_i$}





\subsection{Express $O[n]$ in terms of subfunction complexities}
\begin{center}
$
O_{sub1}\lbrack n \rbrack = O_{T_{sub1}}\lbrack n \rbrack + O_{S_{sub1}}\lbrack n \rbrack
$
\\ \vspace{2mm}
$
O_{sub2}\lbrack n \rbrack = O_{T_{sub2}}\lbrack n \rbrack + O_{S_{sub2}}\lbrack n \rbrack
$
\\ \vspace{2mm}
$
O\lbrack n \rbrack = O_{sub1}\lbrack n \rbrack + O_{sub2}\lbrack n \rbrack + 3
$
\end{center}









\subsection{Find an expression for $O_+[n]$ := the number of $\tilde{L} \lbrack l_i,l_j \rbrack$ + $\tilde{L} \lbrack l_j,l_k \rbrack$  length 1 subpath additions}
\begin{center}
\vspace{2mm}
$
O_+[n] =
$
\end{center}







\subsection{Prove $O_+[n]$ is a subfunction of all $s^+$ by contradiction}
\subsection{Express the solution that stores subpaths of length 1 in memory in terms of $O_+[n]$}
\subsection{Express $O[n],O_T[n],O_S[n],f_{n+1}$ in terms of subfunction complexities including $O_+[n]$ as a subfunction}
\subsection{Show $O_+[n]$ diverges}
\subsection{Prove D diverges by the theorem of divergent subfunctions}




% Universal Bound of Computation
\newpage
\section{Universal Bound of Computation?}
maybe
\begin{center}
$
O[n] < n^n \hspace{3mm} \forall s^+
$
\end{center}

\subsection{Show Polynomial Union Divergent Solutions represent the universe of solutions}
\subsection{Show Polynomial Solutions are bounded by $n^n$}
\subsection{Show Divergent Solutions are bounded by $n^n$?}









% Theorem of Prime Numbers "Riemann Hypothesis"
\newpage
\section{Theorem of Prime Numbers "Riemann Hypothesis"}
Riemann Zeta Function
\begin{center}
$
\zeta (s) \equiv \sum_{n=1}^{\infty} \frac{1}{n^s} \hspace{3mm} \lbrack 2 \rbrack
$
\end{center}
"The prime number theorem determines the average distribution of the primes. The Riemann hypothesis tells us about the deviation from the average. Formulated in Riemann's 1859 paper, it asserts that all the 'non-obvious' zeros of the zeta function are complex numbers with real part 1/2." \lbrack 3\rbrack\\
\\
Prove the problem is divergent\\
There fore it can only be proven to a certain degree\\
The limit as n approaches infinity implies a real part of one half\\
Connection with the real and imaginary part of O $\lbrack n \rbrack$

\subsection{Determine a duality function for the Riemann Hypothesis}
\subsection{Determine an expression for O[n+1] as a function of O[n]}

\subsection{Prove $O_{opt}$ is performing $O_{opt}$ recursively for the ints less than square root of n}
Testing the primes less than sqrt(n)? double check \\
1. Optimal solution for n=1,2,3, everything else is a recursive optimal proof by induction\\
Time Complexity seems to be on the order of n log n... implies divergence or lack of bound? Add in the complexity of division.. probably approaches $n^n$

%\subsection{Show that $O_{opt}$ diverges with $n^n$, isn't bounded by $n^n$}
%Proves O is divergent

\subsection{Since divergent, no $s^{+}$ exists.. only rules}
Express as a limit

\subsection{Show that the limit as n $\rightarrow \infty$ implies the real part is 1/2}
1/2 ± 14.134725 i
1/2 ± 21.022040 i
1/2 ± 25.010858 i
1/2 ± 30.424876 i
1/2 ± 32.935062 i
1/2 ± 37.586178 i

Z = $\zeta(1/2 + it)$

\subsection{Notation, real imaginary parts of the problem}
Even numbers and numbers ending in 5 are automatically convergent\\
Testing numbers ending in 1,3,7,9 results in divergent expression\\
we can continue to add rules to a certain degree




















\newpage
\section{Divergent Problems}
Define $\hat{\mathbb{D}}$ the set of decision problems with no convergent?/finite? solution $\hat{D_j}$\\
\begin{center}
$
\hat{\mathbb{D}} := \{\hat{D}_j,...\}
$
\\ \vspace{2mm}
$
\hat{D_j} \in \mathbb{D} \hspace{3mm} \forall j
$
\\ \vspace{2mm}
$
\not \exists \hat{s}^{+} \in S^{+} : \hat{s}^{+}$ solves $\hat{D}_j \hspace{3mm} \forall \hat{D_j} \in \hat{\mathbb{D}} \Longleftrightarrow 
$
\\
$
\not \exists s^{+} \in S^{+} : O_j \lbrack n \rbrack < n^n  \hspace{3mm} \forall n,j
$
\end{center}
There exists no such solution such that O[n] < $n^n$, but there is a right and wrong answer\\
Either here or in the next chapter we'll prove you can only solve to a certain degree\\
 !!! There exists no such solution such that O[n] < $n^n$ \hspace{3mm} $\forall$ n
 
 \subsection{Definition}
 \begin{center}
$
\hat{O} \lbrack n \rbrack := n^n
$
\\ \vspace{2mm}
$
limit_{n \rightarrow \infty}\frac {O_{opt} \lbrack n \rbrack}{\hat{O}\lbrack n \rbrack} \hspace{2mm} diverges
$
\\ \vspace{2mm}
$
diverges \lbrack \hspace{1mm} O_{opt} \lbrack n \rbrack, \hat{O}\lbrack n \rbrack \hspace{1mm} \rbrack \rightarrow \mathbb{T}
$
\end{center}

\subsection{Theorem of Divergent Programs}
Prove that Divergent implies not in polynomial (trivial)\\
Prove that Divergent implies Non-polynomial (trial after proving above)\\
Show that there exists at least one member of Divergent\\


\section{Properties of Solvable and Divergent problems}

\subsection{Solvable and Divergent are disjoint}
Prove by contradiction



 \section{"Theorem of Divergent Programs"}
\subsection{Divergence Test}
1. Let $d_j \in D$\\
2. $d_j$ = ($d_j \in \hat{\mathcal{D}}$) $\cup$ ($d_j \in$ set of solvable problems) by disjoint condition of solvable and divergent\\
3. Let $O_{opt}\lbrack n \rbrack$, the optimal complexity of $d_j$\\
4. $\rightarrow s_i^{+}$ that solve $d_j$ have larger complexity $\forall i$\\
5. 2 implies  $O_{opt}\lbrack n \rbrack$ is either bounded by $n^n$ or not\\
6. $\hat{O} \lbrack n \rbrack \equiv n^n$\\
7. Easy Suppose $d_j\in $solvable$ limit_{n \rightarrow \infty}\frac{O_{opt}\lbrack n \rbrack}{\hat{O}\lbrack n \rbrack} = 0 $\\
8.Suppose $d_j\in \hat{\mathcal{D}} limit_{n \rightarrow \infty}\frac{O_{opt}\lbrack n \rbrack}{\hat{O}\lbrack n \rbrack} \neq 0 $ (by disjoint condition)\\
\begin{center}
$
 limit_{n \rightarrow \infty}\frac{O_{opt}\lbrack n \rbrack}{\hat{O}\lbrack n \rbrack} = 1 
$
\end{center}




\section{Connection to verification in polynomial time}






\newpage


\section{Fundamental Theorem of Computation}
$n^n$ or $\lambda n^n + C$ the universal bound to solvable computational complexity\\
$(\lambda n)^n$ + C ?\\






\subsection{Time Complexity Argument}
Suppose decision problem d with optimal time complexity $O_{T_{min}}\lbrack n \rbrack$ and solution $s^+$, an arbitrary decision problem in P with polynomial complexity\\
Assumptions\\
1. $d \in P$, $s^{+} \in S^{+}$\\
Assertions\\
2. $\exists K,C,\lambda_1...\lambda_K \hspace{2mm} : \hspace{2mm} O_{T_{min}}\lbrack n \rbrack < (\lambda_K n)^K + (\lambda_{K-1} n)^{K-1}... + \lambda_1 n + C, \hspace{3mm} \forall n\hspace{1mm}$
\\
3. Define $ f\lbrack K,C, \lambda_1,...\lambda_K \rbrack \equiv (\lambda_K n)^K + (\lambda_{K-1} n)^{K-1}... + \lambda_1 n + C$\\
4.  $\exists K,C,\lambda_1...\lambda_K : O_{T_{min}}\lbrack n \rbrack <   f\lbrack K,C, \lambda_1,...\lambda_K \rbrack \hspace{3mm} \forall n$\\
5. Let $\hat{s}^{+} \equiv \nestedloop \lbrack s^{+} \rbrack$\\
6. $O_T \lbrack n \rbrack <= \hat{O}_T \lbrack n\rbrack$ (by defition of nested loop)\\
\\
7.  $\hat{O}_{T_{min}}\lbrack n \rbrack < (\lambda_K n)^K + (\lambda_{K-1} n)^{K-1}... + \lambda_1 n + C$\\
8. $\hat{O}_{T_{min}} \lbrack n \rbrack < limit_{n \rightarrow \infty}  \nestedloop \lbrack s^{+} \rbrack$ (by definition of limit + definition of nested loop, expand to show full derivation, valid because this is a series, probably need to show limit applies)\\
9. $\therefore  O_{T_{min}}\lbrack n \rbrack < \hat{O}_{T_{min}} \lbrack n \rbrack < n^n  =  limit_{n \rightarrow \infty} \nestedloop \lbrack s^{+} \rbrack$
\\
I want to say forall n but seems refutable for n= 1,2.. but as n approach infinity it's a contradiction to say a solvable problem in P $\hat{O}_{T_{min}} = n^n \hspace{3mm} \forall n$\\
10. For "sufficiently large n"
\begin{center}
$
\not \exists \hat{s}^{+} \in S^{+} : | \hat{s}^{+} | \equiv O_{T_{min}} \lbrack n \rbrack <  n^n, \hspace{3mm} \forall n
$
\\
$
\hat{O} \lbrack n \rbrack \equiv n^n
$
\end{center}

\subsection{Space Argument}
Similar but additional notation required?






\newpage
\section{Divergence Criterion}
Necessary condition for divergent program, iff\\
 or you can show there exists no lambda, C such that O $\lbrack n \rbrack$ is $n^n$ is bounded by $\lambda n^n + C$ for all n

$limit_{n \rightarrow \infty}$ div / solvable > 1

Assumptions\\
1. Define the "Null Space of $\mathcal{D}$" or "Null Set" $O_\perp$
\begin{center}
$
O_\perp = \{\hat{d_1},\hat{d_2},...,\hat{d_j}\}, \hspace{3mm} j > 0
$
\\
$
 \hat{O_j} [n] \equiv (O[n])^n, \forall j
$
\end{center}
2. $O_P \cup O_\mathcal{N} =\mathcal{D} \hspace{3mm}$ (by definition)\\

Assertions\\
3. $O_P$ $\cap$ $O_\perp$ = $\emptyset$\\
4. Let $O_\mathcal{N}$ $\cap$ $O_\perp$ = $\hat{O}$ = $\{\hat{O}_i,...\}$, $i > 0$\\
5. Consider $D_j \in O_\mathcal{N}$\\
6. $D_j$ has finite complexity by definition\\
\begin{center}
 $O_j[n] = C$
\end{center}
7. $D_j$ has at least one optimal solution by the necessity of optimal solution (theorem Z)
\begin{center}
 $O_j[n] = C$
\end{center}
%$ $O_\perp$ = $\emptyset$

































\newpage

\section{Proof of "P $\neq$ NP"}
\subsection{Proof N implies D}
Is trivial by implication of Theorem x and Theorem y
\begin{center}
$
\mathcal{N} \equiv \{ d_j,.. \} \hspace{3mm} \forall j, \mathcal{N} \in D
$
\\
$ 
\not \exists K,C,\lambda_1...\lambda_K : \hspace{2mm}
$
\\
$
\hspace{2mm} O_T[n] < (\lambda_K n)^K + (\lambda_{K-1} n)^{K-1}... + \lambda_1 n + C, \hspace{3mm} \forall n, \hspace{1mm} \forall d_j \in \mathcal{N}
$
\end{center}
1. $\rightarrow \mathcal{P} \cap \mathcal{N} = \emptyset$ by definition of P,N\\
2. $d_i \in \hat{D} \lor d_i \in $solvable\\
3. 1 $\rightarrow d_i \not \in$ solvable\\
4. $\therefore d_i \in \hat{D}\hspace{3mm} \forall i$ (theorem y)
Show that Definition of Non-Polynomial Problems automatically implies Divergent\\
1. We've proven Solvable Union are disjoint and complete set P
2. N not in P by definition
3. therefore N in divergence by set theory

Currently we have only defined solvable problems and divergent problems\\
Additionally polynomial problem which the existance of is trivial\\
Plus we defined non-polynomial complexity\\
Prove the existence of $\mathcal{N}$ the set of non polynomial problems\\
\\

\subsection{Proof that D implies N}
\subsection{D iff N}

Show  $O \lbrack n \rbrack$ in the $\emptyset$ the set of problems with    $n^n$ > $O \lbrack n \rbrack$ > $n^k + c$\\
Proving there's Polynomial and Divergent, in the set of all decision problems\\


A neat follow up, tie in the definition of $\mathcal{N}$ implies membership to divergent problems


\newpage

\section{Prove the existance of D = N,The Traveling Salesman Problem}
Define the traveling salesman problem, prove it is divergent and has the same solution as current approaches\\
Consider proving with both definition and necessary condition\\

\subsection{Compute every sub path or recursive subpaths in memory}
Trade off between time and space, $O_{salesman}[n]$ diverges with a polynomial $O_P[n]$



\section{Prove Polynomial and Divergent problems are Complements}
Implied by the previous sections

\section{Solvable Union Divergent = all decision problems}
Trivial as a result of the previous section by definition of $\Omega$
\begin{center}
$\mathbb{P} \cup \hat{\mathbb{D}} = \mathbb{D}$
\end{center}





\newpage
\section{Theorem of Prime Numbers "Riemann Hypothesis"}
Riemann Zeta Function
\begin{center}
$
\zeta (s) \equiv \sum_{n=1}^{\infty} \frac{1}{n^s} \hspace{3mm} \lbrack 2 \rbrack
$
\end{center}
"The prime number theorem determines the average distribution of the primes. The Riemann hypothesis tells us about the deviation from the average. Formulated in Riemann's 1859 paper, it asserts that all the 'non-obvious' zeros of the zeta function are complex numbers with real part 1/2." \lbrack 2\rbrack\\
\\
Prove the problem is divergent\\
There fore it can only be proven to a certain degree\\
The limit as n approaches infinity implies a real part of one half\\
Connection with the real and imaginary part of O $\lbrack n \rbrack$

\subsection{Prove $O_{opt}$ is performing $O_{opt}$ recursively for the ints less than square root of n}
Testing the primes less than sqrt(n)? double check \\
1. Optimal solution for n=1,2,3, everything else is a recursive optimal proof by induction\\
Time Complexity seems to be on the order of n log n... implies divergence or lack of bound? Add in the complexity of division.. probably approaches $n^n$

%\subsection{Show that $O_{opt}$ diverges with $n^n$, isn't bounded by $n^n$}
%Proves O is divergent

\subsection{Since divergent, no $s^{+}$ exists.. only rules}
Express as a limit

\subsection{Show that the limit as n $\rightarrow \infty$ implies the real part is 1/2}
1/2 ± 14.134725 i
1/2 ± 21.022040 i
1/2 ± 25.010858 i
1/2 ± 30.424876 i
1/2 ± 32.935062 i
1/2 ± 37.586178 i

Z = $\zeta(1/2 + it)$

\subsection{Notation, real imaginary parts of the problem}
Even numbers and numbers ending in 5 are automatically convergent\\
Testing numbers ending in 1,3,7,9 results in divergent expression\\
we can continue to add rules to a certain degree









\newpage
\section*{Citations}

\lbrack1\rbrack \hspace{1mm} $chatgpt$\\
\lbrack2\rbrack \hspace{1mm} $https://stackoverflow.com/questions/3518973/floating-point-exponentiation-without-power-function$\\
\lbrack3 \rbrack \hspace{1mm} $https://stackoverflow.com/questions/27086195/linear-index-upper-triangular-matrix$\\
\lbrack3\rbrack \hspace{1mm} $http://www.math.uchicago.edu/~may/VIGRE/VIGRE2011/REUPapers/Riffer-Reinert.pdf$









\newpage
\section*{Appendix}

\section*{Traveling Salesman Problem of Dimension 2}
\subsection{Express a formal solution with $\mathcal{M}_{n+1}$}
\subsection{Show the $\mathcal{M}_{n+1}$ solution satisfies the  subfunction condition of solutions$: P[\hat{X}_i] \supseteq P[X_i] \hspace{3mm} \forall \hat{X}_i,X_i$}
\subsection{Show the $\mathcal{M}_{n+1}$ solution has additions greater than or equal to $O_{+}[n]$}
\subsection{Express the $\mathcal{M}_{n+1}$ solution with $O_{+}[n]$ as a subfunction}
\subsection{Express O[n] for all solutions with $O_{+}[n]$ as a subfunction}


\newpage
\section*{Traveling Salesman Problem of Dimension 2}
\subsection{Express a formal, general solution $O_S[n] \sim n^0$}
\subsection{Express the zero order Space Complexity Solution as the union of subfunctions}
\subsection{Show the $O_S[n] \sim n^0$ solution satisfies the  subfunction condition of solutions$: P[\hat{X}_i] \supseteq P[X_i] \hspace{3mm} \forall \hat{X}_i,X_i$}
\subsection{Find $O[n], O_T[n], O_S[n],f_{n+1}$ for the zero order Space Complexity Solution subfunctions}
\begin{center}
$
O_S[n] = |y_o| + |\{b_1,b_2,b_3\}| = 4
$
\\ \vspace{2mm}
$
O_T[n] = 
$
\end{center}

\subsection{Find $O[n], O_T[n], O_S[n],f_{n+1}$ for the overall zero order Space Complexity Solution}




% Traveling Salesman Zero Order Solution
\newpage
\section*{Traveling Salesman Problem of Dimension 2}
\subsection{Express a formal, general $\mathcal{M}_1$ solution}
\begin{center}
\vspace{2mm}Traveling Salesman Solution $\hspace{3mm} n == N \hspace{3mm} O_S[n] \sim n^0
$
\\ \vspace{2mm}
$
X_i = \{x_1,x_2,...,x_n,C\}
$
\\ \vspace{2mm}
$
C = \{d_{proposed},p_{decimal}\}; \hspace{2mm} |C| = |\{d_{proposed},p_{decimal}\}| = 2
$
\\ \vspace{2mm}
$
y_o \hspace{2mm} d_{proposed}$ is the shortest distance
\\ \vspace{2mm}
$
b_1 \hspace{2mm} d_{proposed}$ the proposed shortest distance
\\ \vspace{2mm}
$
b_2$ current path length
\\ \vspace{2mm}
$
b_3$ intermediate path length or $d_{proposed}$ < current path
\end{center}
// Set $y_o$ and $b_1$\\
$s_1 = b_1 \leftarrow d_{proposed}$\\
$s_2 = y_o \leftarrow \mathbb{T}$\\ \\
// Calculate $\mathcal{M}_1$ subpaths := $x_i$ to  $x_j \hspace{2mm} \forall i,j > i$\\
// Calculate (n P (n-1))/2 routes\\
// Let current route = $\{x_{c1},x_{c2},...,x_{cn}\}, \hspace{2mm} x_{ci} \in X_i \hspace{2mm} \forall x_{ci}$\\
$s,s = b_2 \leftarrow L \lbrack {x}_{c1},{x}_{c2} \rbrack$\\
$s,s = b_3 \leftarrow L \lbrack {x}_{c2},{x}_{c3} \rbrack$\\
$s,s = b_2 \leftarrow b_2 + b_3$\\
$s,s = b_3 \leftarrow L \lbrack {x}_{c3},{x}_{c4} \rbrack$\\
$s,s = b_2 \leftarrow b_2 + b_3$\\
...\\
$s,s = b_3 \leftarrow L \lbrack {x}_{c(n-1)}, {x}_{cn} \rbrack$\\
$s,s = b_2 \leftarrow b_2 + b_3$\\
$s,s = b_3 \leftarrow L \lbrack {x}_{cn}, {x}_{c1}\rbrack$\\
$s,s = b_2 \leftarrow b_2 + b_3$\\
$s,s = b_3 \leftarrow b_1 < b_2$\\
$s = y_o \leftarrow y_o \land b_3$







\subsection{Express the $\mathcal{M}_1$ Solution as the union of subfunctions}
\subsection{Show the  $\mathcal{M}_1$ solution satisfies the subfunction condition of solutions$: P[\hat{X}_i] \supseteq P[X_i] \hspace{3mm} \forall \hat{X}_i,X_i$}
\subsection{Find $O[n], O_T[n], O_S[n],f_{n+1}$ for the $\mathcal{M}_1$ Solution Subfunctions}
\begin{center}
$
O_S[n] = |y_o| + |\{b_1,b_2,b_3\}| = 4
$
\\ \vspace{2mm}
$
O_T[n] = 
$
\end{center}


\subsection{Find $O[n], O_T[n], O_S[n],f_{n+1}$ for the $\mathcal{M}_1$ Overall Solution}



\subsection{Find an expression for the number of additions $x_i + x_j \hspace{2mm} \forall i,j>1 \hspace{2mm} := O_{+}[n]$ for all solutions $s^+$}
\subsection{Prove $O_{+}[n]$ is a subfunction of all solutions $s^+$}
Proof by contradiction
\subsection{Show $O_{+}[n]$ diverges}
\subsection{Prove The Traveling Salesman Problem is in $\hat{D}$ using the Theorem of Divergent Subfunctions}
















% Sum to N Old
\section{Sum to N Problem with 2 integers Old}

% Zero Order Space Complexity Solution
\subsection{State formal definition of Sum to N : $x_i + x_j == N$}
\begin{center}
\vspace{1.5mm}
$
X_i = \{x_1,...,x_n,N\}
$
\\ \vspace{2mm}
$
D := f \lbrack X_i \rbrack \rightarrow a_{o} \in \{\mathbb{T}, \mathbb{F}\} \hspace{3mm} \forall X_i
$
\\ \vspace{2mm}
$
s^+ = s^+\lbrack n \rbrack := P \lbrack X_i \rbrack \rightarrow y_{o} : y_o = a_{o} \hspace{3mm} \forall X_i
$
\\ \vspace{2mm}
$
s^+ = \{ s_1,s_2,...,s_{O_T \lbrack n \rbrack }, b_1, b_2,...,b_{O_S \lbrack n \rbrack},y_o \} = \{ \mathcal{L},\mathcal{M},y_o\}
$
\\ \vspace{6mm}
$
D = f[X_i] = \exists x_j,x_k \in X_i : x_j + x_k == N
$
\end{center}

\subsection{Express a formal solution : $O_S[n] \sim n^0$}
\begin{center}
\vspace{1.5mm}
$
s^+ = \{ s_1,s_2,...,s_{O_T \lbrack n \rbrack }, b_1, b_2,...,b_{O_S \lbrack n \rbrack},y_o \} = \{ \mathcal{L},\mathcal{M},y_o\}
$
\\ \vspace{2mm}
$
s_1 = y_o \leftarrow \mathbb{F};
$
\\ \vspace{2mm}
$
\forall i,j > i
$
\\ \vspace{2mm}
$
s_2,s_3,s_8,s_9,...,s_{3ij-4},s_{3ij-3}...,s_{3n(n-1)-4},s_{3n(n-1)-3} =  b_1 \leftarrow x_i + x_j 
$
\\ \vspace{2mm}
$
s_4,s_5,s_{10},s_{11},...,s_{3ij-2},s_{3ij-1}...,s_{3n(n-1)-2},s_{3n(n-1)-1} = b_1 \leftarrow b_1 == N
$
\\ \vspace{2mm}
$
s_6,s_7,s_{12},s_{13}...,s_{3ij},s_{3ij+1}...,s_{3n(n-1)},s_{3n(n-1)+1} = y_o \leftarrow y_o \lor b_1
$
\\ \vspace{2mm}
$
s^+ = \{y_o \leftarrow \mathbb{F},y_o \leftarrow y_o \lor (x_i + x_j == N) \hspace{3mm} \forall i,j > i \hspace{1mm}| \hspace{1mm} b_1,y_o\}
$
\end{center}

\subsection{Determine $O[n], O_S[n], O_T[n]$ for the above solution}
\begin{center}
$
O_S[n] = |y_o| + |b_1| = 2
$
\\ \vspace{2mm}
$
O_T[n] = 3n(n-1) + 1 = 3n(n-1) -1 + O_S[n]
$
\\ \vspace{2mm}
$
O[n] = 3n(n-1) + 3 = 3n^2 -3n + 3
$
\end{center}

\subsection{Show $s^+$ is bounded by Polynomial Complexity by the definition of Total Polynomial Complexity}
\begin{center}
$
O[n] = 3n(n-1) + 3 = 3n^2 -3n + 3
$
\\ \vspace{2mm}
$
O[n] = 3n^2 -3n + 3 < 3n^2 -3n +4 \hspace{2mm} \forall n
$
\\ \vspace{2mm}
$
\therefore s^+ \in S^+_{\mathbb{P}}
$
\end{center}

\subsection{Show $s^+$ is bounded by Polynomial Complexity by showing limit$_{n \rightarrow \infty}\frac{O[n+1]}{O[n]}$ = 1}
\begin{center}
$
limit_{n \rightarrow \infty}\frac{O[n+1]}{O[n]} =
$
\\ \vspace{2mm}
$
limit_{n \rightarrow \infty}\frac{3n^2 + 3n + 3}{3n^2 - 3n + 3} =
$
\\ \vspace{2mm}
$
limit_{n \rightarrow \infty}(\frac{3n^2 - 3n + 3}{3n^2 - 3n + 3} + \frac{6n }{3n^2 - 3n + 3}) =
$
\\ \vspace{2mm}
$
limit_{n \rightarrow \infty}(1 + \frac{6n }{3n^2 - 3n + 3}) = 1
$
\end{center}




% First Order Space Complexity Solution
\subsection{Express a formal solution : $O_S[n] \sim n^1$}
\begin{center}
$
s^+ = \{ s_1,s_2,...,s_{O_T \lbrack n \rbrack }, b_1, b_2,...,b_{O_S \lbrack n \rbrack},y_o \} = \{ \mathcal{L},\mathcal{M},y_o\}
$
\\ \vspace{2mm}
$
s_1 = y_o \leftarrow \mathbb{F};
$
\\ \vspace{2mm}
$
\forall x_i
$
\\ \vspace{2mm}
$
s_2,s_3,... = b_i \leftarrow N - x_i
$
\\ \vspace{2mm}
$
\forall b_j \in \mathcal{M} \hspace{2mm} j \leq i
$
\\ \vspace{2mm}
$
s_4,s_5,... = b_{n+1} \leftarrow b_j == x_i
$
\\ \vspace{2mm}
$
s_6,s_7,... = y_o \leftarrow y_o \lor b_{n+1}
$
\end{center}








\subsection{Determine $O[n], O_S[n], O_T[n]$ for the above first order space complexity solution}
\begin{center}
$
O_S[n] = |y_o| + |b_1...b_n| + |b_{n+1}| = n + 2
$
\\ \vspace{2mm}
$
O_T[n] = 1 + 2n + 4\sum_{i=1}^n i = 1 + 2n + 2n(n+1) = 2n^2 + 4n + 1 
$
\\ \vspace{2mm}
$
O[n] = 2n^2 + 4n + n + 1 + 2 = 2n^2 + 5n + 3
$
\end{center}

\subsection{Determine inductive function $f_{n+1}$ using time and space inductive functions $f^{T}_{n+1}$, $f^{S}_{n+1}$}
\begin{center}
$
f_{n+1}[n] = f^{T}_{n+1} + f^{S}_{n+1}
$
\\ \vspace{2mm}
$
f_{n+1}[n] = O_T[n+1] - O_T[n] +  O_S[n+1] - O_S[n]
$
\\ \vspace{2mm}
$
f_{n+1}[n] =  1 + 2(n+1) + 4\sum_{i=1}^{n+1}i - 1 - 2n - 4\sum_{i=1}^{n}i +  n + 1 + 2 - n - 2
$
\\ \vspace{2mm}
$
f_{n+1}[n] =   2(1) + 4\sum_{i=n+1}^{n+1}i  + 1 = 4(n+1) + 3 = 4n + 7
$
\end{center}


\subsection{Show $s^+$ is bounded by Polynomial Complexity by showing limit$_{n \rightarrow \infty}\frac{O[n+1]}{O[n]}$ = 1}
By showing the order $f_{n+1}[n]$ is less than O[n]
\begin{center}
$
limit_{n \rightarrow \infty}\frac{O[n+1]}{O[n]} = 
$
\\ \vspace{2mm}
$
limit_{n \rightarrow \infty}\frac{O[n]+ f_{n+1}[n]}{O[n]} = 
$
\\ \vspace{2mm}
$
limit_{n \rightarrow \infty} 1 + \frac{f_{n+1}[n]}{O[n]} = 
$
\\ \vspace{2mm}
$
limit_{n \rightarrow \infty} 1 + \frac{4n+7}{2n^2 + 5n + 3} = 1
$
\end{center}









% Sum to N General Solution
\newpage
\section{Sum to N Problem: q == 3}
\subsection{State formal definition of Sum to N with 3 integers : $x_j + x_k + x_l == N$}
\begin{center}
$
X_i = \{x_1,...,x_n,N\}
$
\\ \vspace{2mm}
$
D := f \lbrack X_i \rbrack \rightarrow a_{o} \in \{\mathbb{T}, \mathbb{F}\} \hspace{3mm} \forall X_i
$
\\ \vspace{2mm}
$
s^+ = s^+\lbrack n \rbrack := P \lbrack X_i \rbrack \rightarrow y_{o} : y_o = a_{o} \hspace{3mm} \forall X_i
$
\\ \vspace{2mm}
$
s^+ = \{ s_1,s_2,...,s_{O_T \lbrack n \rbrack }, b_1, b_2,...,b_{O_S \lbrack n \rbrack},y_o \} = \{ \mathcal{L},\mathcal{M},y_o\}
$
\\ \vspace{6mm}
$
D = f[X_i] = \exists x_j,x_k,x_l \in X_i : x_j + x_k + x_l == N
$
\end{center}

\subsection{Determine a solution of zero order Space Complexity}
\begin{center}
\vspace{1.5mm}
$
s^+ = \{ s_1,s_2,...,s_{O_T \lbrack n \rbrack }, b_1, b_2,...,b_{O_S \lbrack n \rbrack},y_o \} = \{ \mathcal{L},\mathcal{M},y_o\}
$
\\ \vspace{2mm}
$
s_1 = y_o \leftarrow \mathbb{F};
$
\\ \vspace{2mm}
$
\forall j, k \neq j, l \neq k,j
$
\\ \vspace{2mm}
$
s_2,s_3,... =  b_1 \leftarrow x_j + x_k
$
\\ \vspace{2mm}
$
s_4,s_5,... =  b_1 \leftarrow b_1 + x_l
$
\\ \vspace{2mm}
$
s_6,s_7,... = b_1 \leftarrow b_1 == N
$
\\ \vspace{2mm}
$
s_8,s_9,... = y_o \leftarrow y_o \lor b_1
$
\\ \vspace{2mm}
$
s^+ = \{y_o \leftarrow \mathbb{F},y_o \leftarrow y_o \lor (x_j + x_k + x_l == N) \hspace{3mm} \forall j, k \neq j, l \neq k,j \hspace{1mm}| \hspace{1mm} b_1,y_o\}
$
\end{center}

\subsection{Determine $O[n], O_S[n], O_T[n]$ for the above solution}
\begin{center}
$
O_S[n] = |y_o| + |b_1| = 2
$
\\ \vspace{2mm}
$
O_T[n] = \frac{8n(n-1)(n-2)}{6}+ 1
$
\\ \vspace{2mm}
$
O[n] = \frac{8n(n-1)(n-2)}{6}+ 3
$
\end{center}


\subsection{Find the order of $s^+$; Show $s^+$ is bounded by Polynomial Complexity}
\begin{center}
$
O[n] = \frac{8n(n-1)(n-2)}{6}+ 3
$
\\ \vspace{2mm}
$
O[n] \sim n^3
$
\\ \vspace{2mm}
$
O[n] < \frac{8n(n-1)(n-2)}{6} + 4
$
\end{center}







% Sum to N 3 ints; 1st order Space Complexity Solution
\subsection{Determine a solution of first order Space Complexity}
\begin{center}

\end{center}

\subsection{Determine $O[n], O_S[n], O_T[n]$ for the above first order Space Complexity Solution}
\begin{center}

\end{center}

\subsection{Determine inductive functions  $f^T_{n+1}[n],  f^S_{n+1}[n]$}
\begin{center}

\end{center}

\subsection{Determine inductive function $f_{n+1}[n] =  f^T_{n+1}[n] +  f^S_{n+1}[n]$}
\begin{center}

\end{center}

\subsection{Show $s^+$ is bounded by Polynomial Complexity by showing limit$_{n \rightarrow \infty}\frac{O[n+1]}{O[n]} = 1$}
Satisfying the criteria that $f_{n+1}$[n] has lower order than O[n] 
\begin{center}

\end{center}





\section{Sum to N problem: general q}
\subsection{State general definition of Sum to N with q integers : $x_j + x_k \hspace{1mm} ... \hspace{1mm} + x_q == N$}
\begin{center}
$
X_i = \{x_1,...,x_n,N\}
$
\\ \vspace{2mm}
$
D := f \lbrack X_i \rbrack \rightarrow a_{o} \in \{\mathbb{T}, \mathbb{F}\} \hspace{3mm} \forall X_i
$
\\ \vspace{2mm}
$
s^+ = s^+\lbrack n \rbrack := P \lbrack X_i \rbrack \rightarrow y_{o} : y_o = a_{o} \hspace{3mm} \forall X_i
$
\\ \vspace{2mm}
$
s^+ = \{ s_1,s_2,...,s_{O_T \lbrack n \rbrack }, b_1, b_2,...,b_{O_S \lbrack n \rbrack},y_o \} = \{ \mathcal{L},\mathcal{M},y_o\}
$
\\ \vspace{6mm}
$
D = f[X_i] = \exists x_j,x_k,x_l...,x_q \in X_i : x_j + x_k + ... + x_q == N
$
\end{center}


\subsection{Determine a general solution of zero order Space Complexity}
\begin{center}
\vspace{1.5mm}
$
s^+ = \{ s_1,s_2,...,s_{O_T \lbrack n \rbrack }, b_1, b_2,...,b_{O_S \lbrack n \rbrack},y_o \} = \{ \mathcal{L},\mathcal{M},y_o\}
$
\\ \vspace{2mm}
$
s_1 = y_o \leftarrow \mathbb{F};
$
\\ \vspace{2mm}
$
\forall j; k \neq j; l \neq k,j; ... ; q \neq j,k,...,p
$
\\ \vspace{2mm}
$
s_2,s_3,... =  b_1 \leftarrow x_j + x_k
$
\\ \vspace{2mm}
$
s_4,s_5,... =  b_1 \leftarrow b_1 + x_l
$
\\ \vspace{2mm}
$
s_6,s_7,... = b_1 \leftarrow b_1 == N
$
\\ \vspace{2mm}
$
s_8,s_9,... = y_o \leftarrow y_o \lor b_1
$
\\ \vspace{2mm}
$
s^+ = \{y_o \leftarrow \mathbb{F},y_o \leftarrow y_o \lor (x_j + x_k + x_l == N) \hspace{3mm} \forall j, k \neq j, l \neq k,j \hspace{1mm}| \hspace{1mm} b_1,y_o\}
$
\end{center}






% Proof of Non-Polynomial Problems Traveling Salesman
\newpage
\section{Proof of the existence of $\hat{\mathcal{D}}$}
Non-trivial; Formalize the traveling salesman problem as a decision problem (any optimization problem)

\subsection{The Traveling Salesman Problem}
English description

\subsection{Formal Definition}
\begin{center}
$
X_i = \{ c_1,c_2,...,c_n \} :
$
\\ \vspace{2mm}
$
dim\lbrack c_i \rbrack = C > 1
$
\\ \vspace{4mm} 
$
\bar{X_i} = \{ c_1,c_2,...,c_n,\bar{P},\bar{f}[c_i,c_j] \}
$
\\ \vspace{2mm}
$
\bar{P} := \{c_k,...\} :
$
\\ \vspace{2mm}
$
\exists c_k \in \bar{P} \hspace{2mm} \forall c_k \in X_i
$
\end{center}



% Traveling Salesman Zero Order Solution
\newpage
\subsection{Express a formal, general solution $O_S[n] \sim n^0$}
\begin{center}
\vspace{2mm}Traveling Salesman Solution $\hspace{3mm} n == N \hspace{3mm} O_S[n] \sim n^0
$
\\ \vspace{2mm}
$
X_i = \{x_1,x_2,...,x_n,L \lbrack x_i,x_j \rbrack,R\}
$
\\ \vspace{2mm}
$
R = \{\hat{x}_1,\hat{x}_2,...,\hat{x}_n\}
$
\\ \vspace{2mm}
$
y_o \hspace{2mm}$ R is the shortest path
\\ \vspace{2mm}
$
b_1$ length of path R
\\ \vspace{2mm}
$
b_2$ current path length
\\ \vspace{2mm}
$
b_3$ intermediate path length or R < current path
\end{center}
\vspace{4mm}// Calculate Proposed Route Distance\\
$s_1,s_2 = b_1 \leftarrow L \lbrack \hat{x_1},\hat{x_2} \rbrack$\\
$s_3,s_4 = b_2 \leftarrow L \lbrack \hat{x_2},\hat{x_3} \rbrack$\\
$s_5,s_6 = b_1 \leftarrow b_1 + b_2$\\
$s_7,s_8 = b_2 \leftarrow L \lbrack \hat{x_3},\hat{x_4} \rbrack$\\
$s_9,s_{10} = b_1 \leftarrow b_1 + b_2$\\
...\\
$s,s = b_2 \leftarrow L \lbrack \hat{x}_{n-1}, \hat{x}_n \rbrack$\\
$s,s = b_1 + b_2$\\
$s,s = b_2 \leftarrow L \lbrack \hat{x}_{n}, \hat{x}_1 \rbrack$\\
$s,s = b_1 + b_2$\\
$s = y_o \leftarrow \mathbb{T}$\\ \\
// Calculate other (n P (n-1))/2 routes\\
// Let current route = $\{x_{c1},x_{c2},...,x_{cn}\}, \hspace{2mm} x_{ci} \in X_i \hspace{2mm} \forall x_{ci}$\\
$s,s = b_2 \leftarrow L \lbrack {x}_{c1},{x}_{c2} \rbrack$\\
$s,s = b_3 \leftarrow L \lbrack {x}_{c2},{x}_{c3} \rbrack$\\
$s,s = b_2 \leftarrow b_2 + b_3$\\
$s,s = b_3 \leftarrow L \lbrack {x}_{c3},{x}_{c4} \rbrack$\\
$s,s = b_2 \leftarrow b_2 + b_3$\\
...\\
$s,s = b_3 \leftarrow L \lbrack {x}_{c(n-1)}, {x}_{cn} \rbrack$\\
$s,s = b_2 \leftarrow b_2 + b_3$\\
$s,s = b_3 \leftarrow L \lbrack {x}_{cn}, {x}_{c1}\rbrack$\\
$s,s = b_2 \leftarrow b_2 + b_3$\\
$s,s = b_3 \leftarrow b_1 < b_2$\\
$s = y_o \leftarrow y_o \land b_3$







\subsection{Express the zero order Space Complexity Solution as the union of two subfunctions}



\subsection{Show the $O_S[n] \sim n^0$ solution satisfies the  subfunction condition of solutions$: P[\hat{X}_i] \supseteq P[X_i] \hspace{3mm} \forall \hat{X}_i,X_i$}


\subsection{Find $O[n], O_T[n], O_S[n],f_{n+1}$ for the zero order Space Complexity Solution}
\begin{center}
$
O_S[n] = |y_o| + |\{b_1,b_2,b_3\}| = 4
$
\\ \vspace{2mm}
$
O_T[n] = 
$
\end{center}




\subsection{Find a formal expression for the minimum number of additions $O_{+}[n]$ for all solutions $s^+$}
Proof by contradiction
\subsection{Show the $O_S[n] \sim n^0$ solution has additions greater than or equal to $O_{+}[n]$}
\subsection{Express the $O_S[n] \sim n^0$ solution with $O_{+}[n]$ as a subfunction}
\subsection{Introduce $\mathcal{M}_x$ notation}
\subsection{Express a formal solution with $\mathcal{M}_{n+1}$}
\subsection{Show the $\mathcal{M}_{n+1}$ solution satisfies the  subfunction condition of solutions$: P[\hat{X}_i] \supseteq P[X_i] \hspace{3mm} \forall \hat{X}_i,X_i$}
\subsection{Show the $\mathcal{M}_{n+1}$ solution has additions greater than or equal to $O_{+}[n]$}
\subsection{Express the $\mathcal{M}_{n+1}$ solution with $O_{+}[n]$ as a subfunction}
\subsection{Express O[n] for all solutions with $O_{+}[n]$ as a subfunction}
\subsection{Prove The Traveling Salesman Problem is in $\hat{D}$ using the Theorem of Divergent Subfunctions}







% Proof of Non-Polynomial Problems Traveling Salesman
\newpage
\section*{Previous Material}
\section{Proof of the existence of $\hat{\mathcal{D}}$}
Non-trivial; Formalize the traveling salesman problem as a decision problem (any optimization problem)

\subsection{The Traveling Salesman Problem}
English description

\subsection{Formal Definition}
\begin{center}
$
X_i = \{ c_1,c_2,...,c_n \} :
$
\\ \vspace{2mm}
$
dim\lbrack c_i \rbrack = C > 1
$
\\ \vspace{4mm} 
$
\bar{X_i} = \{ c_1,c_2,...,c_n,\bar{P},\bar{f}[c_i,c_j] \}
$
\\ \vspace{2mm}
$
\bar{P} := \{c_k,...\} :
$
\\ \vspace{2mm}
$
\exists c_k \in \bar{P} \hspace{2mm} \forall c_k \in X_i
$
\end{center}

\subsection{Determine $s^+ \Longrightarrow O_{S_{opt}}$}
\subsection{Express $O[n],O_T[n],O_S[n] =  O_{S_{opt}}$}
\subsection{Revisit expressions properties inequalities connecting $O_{opt}; O_{T_{opt}}; O_{S_{opt}}$}
\subsection{Determine an alternate solution storing subpaths}
\subsection{Express $O[n],O_T[n],O_S[n]$}
\subsection{Determine a dual function}
\subsection{Show $limit_{n \rightarrow \infty} \frac{\hat{O}[n+1]}{\hat{O}[n]}$ \hspace{1mm} diverges}
\subsection{Let $\hat{s}^+$; a solution with $limit_{n \rightarrow \infty} \frac{\hat{O}[n+1]}{\hat{O}[n]}$ = 1}
\subsection{Show $\hat{s}^+$ implies a contradiction}


\subsection{Determine $s^+_{S_{opt}}$}
\subsection{Determine equivalence function}
\subsection{Determine an expression for $\frac{\hat{O_{opt}}[n+1]}{\hat{O_{opt}}[n]}$}

\subsection{Show inductive function diverges for all orders of $O_S[n]$}


\newpage
\section{Proof of "P $\neq$ NP"}

\end{document}