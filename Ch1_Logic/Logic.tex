\documentclass[11pt]{article}
\usepackage{amsfonts}
\usepackage[T1]{fontenc}
\usepackage{mathabx,graphicx}


\newcommand{\test}{\circlearrowright}
\def \loop {\ensuremath{\rotatebox[origin=c]{-90}{$\circlearrowright$}}}
\def \nestedloop {\ensuremath{\rotatebox[origin=c]{-90}{$\circlearrowright$}}^n}



\begin{document}
\section*{Logic}


\section{Statement $s$}
\begin{center}
$
"Statement" := s:
$
\\ \vspace{2mm}
$
\mathbb{T} \doteq s\hspace{2mm} \oplus \hspace{2mm} \mathbb{F} \doteq s
$
\end{center}




\section{Definition of Positive Implication}
\begin{center}
$
"Positive$ $Implication" := s_{in} \cup s_{result}:
$
\\ \vspace{2mm}
$
\mathbb{T} \doteq s_{in} \Rightarrow \mathbb{T} \doteq s_{result}
$
\end{center}



\section{Definition of Negative Implication}
\begin{center}
$
"Negative$ $Implication" := s_{in} \cup s_{result}:
$
\\ \vspace{2mm}
$
\mathbb{F} \doteq s_{in} \Rightarrow \mathbb{F} \doteq s_{result}
$
\end{center}




\section{Definition of Preventative Implication}
\begin{center}
$
"Preventative$ $Implication" := s_{in} \cup s_{result}:
$
\\ \vspace{2mm}
$
\mathbb{T} \doteq s_{in} \Rightarrow \mathbb{F} \doteq s_{result}
$
\end{center}



\section{Definition of Consequential Implication}
\begin{center}
$
"Consesquential$ $Implication" := s_{in} \cup s_{result}:
$
\\ \vspace{2mm}
$
\mathbb{F} \doteq s_{in} \Rightarrow \mathbb{T} \doteq s_{result}
$
\end{center}







\section{Fact $t$}
\subsection{Definition of "Fact" $t$}
\begin{center}
$
"Fact" := t:
$
\\ \vspace{2mm}
$
\mathbb{T} \doteq t
$
\end{center}



\subsection{Prove a fact is a statement}



\subsection{English Examples}
Let $t=$ Five plus five is equal to ten
\begin{center}
Five plus five is equal to ten is a true statement\\ \vspace{2mm}
Flive plus five is equal to ten is a fact
\end{center}




\section{Lie $l$}
\subsection{Definition of "lie" $l$}
\begin{center}
$
"lie" := l:
$
\\ \vspace{2mm}
$
\mathbb{F} \doteq l
$
\end{center}



\subsection{Prove a "lie" is a statement}




\subsection{English Example}
Consider the statement five plus five is equal to nine
\begin{center}
Five plus five is equal to nine is a false statement\\ \vspace{2mm}
Five plus five is equal to nine is a lie
\end{center}






\section{Not s $\lnot$ s}
\subsection{Definition of "Not s" $\lnot$ s}
\vspace{2mm}
\begin{center}
$
\lnot s :=
$
\end{center}
$\mathbb{T} \doteq s$
\begin{center}
$
\mathbb{F} \doteq \lnot s
$
\end{center}
\vspace{2mm}
$\mathbb{F} \doteq s$
\begin{center}
$
\mathbb{T} \doteq \lnot s
$
\end{center}







\subsection{English Example}
Let
\begin{center}
$s=$ Humans can travel faster than the speed of light.\\ \vspace{2mm}
$\lnot s=$ Humans can not travel faster than the speed of light.\\ \vspace{2mm}
$\mathbb{F} \doteq s$\\
s is false\\ \vspace{2mm}
$\mathbb{T} \doteq \lnot s$\\
$\lnot s$ is a fact
\end{center}





\section{Definition of Or $\lor$}
The law of excluded middle
\begin{center}
\end{center}
\subsection{The Law of Total Equivalence}


\section{Definition of And $\land$}
The law of non-contradiction
\begin{center}
$
s \land \lnot s = \mathbb{F}
$
\end{center}
\subsection{Definition of Contradiction}
\begin{center}
$
s \land \lnot s = \mathbb{T}
$
\end{center}








\section{Remaining 2 Variable Logical Definitions}
Express explicitly; Express in terms of the above definitions
\subsection{XOR}
\subsection{NOR}
\subsection{XNOR}
\subsection{NAND}






\section{Universality of Logical Expressions}
\subsection{Universality of Not $\lnot$;\hspace{2mm}Logical Or $\lor$;\hspace{2mm} Logical $\land$}




















\newpage
\section*{Appendix}

\section{"-ness"}
Happy is an adjective\\
Happyiness is a noun\\
The dog is happy\\
The dog has happiness\\
Happiness is a (current or permanent) quality of the dog

\section{is vs is a}
The cat is a feline\\
The cat is a member of the set of felines\\
vs\\
The cat is hairy\\
The cat has hair. The cat has the quality of hairyness\\
vs\\
The cat is a hairy cat\\
The cat is a member of the set of cats having the quality of hairyness\\


\section{Overloaded "is"}
Object c is a cat\\
c = cat\\
\\
That cat is a feline\\
cat = feline?\\
cat $\Rightarrow$ feline\\
cat inherits felineness\\
\\
That cat is hairy\\
cat = hairy?\\
The cat has hairyness; the quality of having hair


\section{Criticism of "Is True" and "Is False"}
Consider true statement t\\
"Statement t is true" is equivalent to saying "Statement t equals True"\\
Statement t is not equivalent to True. Statement has the property of truth.\\
\\
Consider false statement f\\
"Statement f is false" is equivalent to saying "Statement f equals False"\\
False statement f is not equivalent to False. Statement f has the property of falsehood.\\




\section{English Translation of Logical Or $\lor$}
$b_1$ Logical Or $b_2$ is spoken in English as "at least one of the following is true. $b_1$. $b_2.$"
\subsection{English Example}
At least one of the following is true.\\
Most dogs have four legs.\\
Two plus three is equal to 5.
\subsection{Criticism of "Logical Or" In Computer Science}
In Computer Science, $b_1$ Logical Or $b_2$ is often spoken as "$b_1$ or $b_2$". "$b_1$ or $b_2$" can lead to inconsistent statements.\\
\begin{center}
$b_1 = ($int $3 \subset [1,2,3])$\\
\vspace{2mm}
$b_2 = ((2 + 2) == 5)$\\
\end{center}
\vspace{2mm}
The following is a valid expression in Computer Science
\begin{center}
$
b_1 \lor b_2 = \mathbb{T}
$
\end{center}
The expression is read in English as "3 is in the list 1 2 3 or 2 plus 2 is equal to 5". The expression is True by definition but $b_1$ or $b_2$
do not necessarily imply Truth.\\\\
\begin{center}
$b_1 = ($int $3 \subset [1,2,3])$\\
\vspace{2mm}
$b_2 = ($int $3 \not \subset [1,2,3])$\\
\end{center}
\vspace{2mm}
The following is a valid expression in Computer Science
\begin{center}
$
 b_1 \lor b_2 =  b_1 \lor \lnot b_1 = \mathbb{T}
$
\end{center}
The expression is read in English as "3 is in the list 1 2 3 or 3 is not in the list 1 2 3 is True". The expression is necessarily true.\\
\\Now consider
\begin{center}
$
 b_1 \land b_2 =  b_1 \land \lnot b_1 = \mathbb{F}
$
\end{center}
The expression is read in English as "3 is in the list 1 2 3 and 3 is not in the list 1 2 3 is True". The expression in computer science evaluates to false


\section{English Translation of "Exclusive Or" XOR}
"$b_1$ Exclusive Or $b_2$" is read in English as "Either $b_1$ Or $b_2$"
\subsection{Example}
Either\\
Three plus four is equal to seven\\
or\\
Three plus four is equal to eight
\subsection{Criticism of English Expression of "Exclusive Or"}
"Exclusive Or" is often expressed as "Or" in English.\\
I can order the salad for lunch\\
or\\
I can order tofu for lunch

\section{Commentary on "Logical And"}
In English, "$b_1$ Logical And $b_2$" is read as "And"
\subsection{Example}
Most dogs have four legs\\
and\\
Most cats have four legs


\section{Criticism logical union, set union, logical and, set and}
- logical or is a function logical and is a function\\
- language muks up our understanding\\

Logical or $\lor$ is different from $\cup$
Logical and $\land$ is different from $\cap$

Logical or, only one has to be true\\
Logical and, both have to be true --> I'll take the intersection\\

Set and, I'll take bag 1 and bag 2 i'll take both --> I'll take the union\\
set or, I'll take bag 1 or bag 2 I'll take just one\\

Do we ever confuse set union, set and with logical or, and?\\
(Don't we describe set union $\cup$ as "or")






\end{document}