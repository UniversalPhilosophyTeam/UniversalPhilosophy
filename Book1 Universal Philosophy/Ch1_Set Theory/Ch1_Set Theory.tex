\documentclass[11pt]{article}
\usepackage{geometry}                
\geometry{letterpaper}                 
\usepackage[parfill]{parskip}        
\usepackage{graphicx}
\usepackage{subfigure}
\usepackage{amssymb}
\usepackage{amssymb}
\usepackage{amsmath}
\usepackage{epstopdf}
\usepackage{verbatim}
\usepackage{float}
\usepackage{grffile}
\usepackage{fullpage}
\usepackage{enumerate}
\usepackage{amsmath}
\usepackage{hyperref}
\usepackage[utf8]{inputenc}
\usepackage{gensymb}
\usepackage[T1]{fontenc}
\usepackage[hang,small]{caption}
\DeclareGraphicsRule{.tif}{png}{.png}{`convert #1 `dirname #1`/`basename #1 .tif`.png}

\graphicspath{ {C:/Users/Nate/Documents/School/EECS351/Nates Discussion Material/Discussion 4} }
\usepackage{listings}
\usepackage{color}
\usepackage{textcomp}
\definecolor{listinggray}{gray}{0.9}
\definecolor{lbcolor}{rgb}{1,1,1}
\lstset{
	backgroundcolor=\color{lbcolor},
	tabsize=4,
	rulecolor=,
	language=matlab,
	basicstyle= \scriptsize,
	upquote=true,
	aboveskip={1.5\baselineskip},
	columns=fixed,
        	showstringspaces=false,
        	extendedchars=true,
        	breaklines=true,
        	prebreak = \raisebox{0ex}[0ex][0ex]{\ensuremath{\hookleftarrow}},
        	frame=single,
        	showtabs=false,
        	showspaces=false,
        	showstringspaces=false,
        	identifierstyle=\ttfamily,
        	keywordstyle=\color[rgb]{0,0,1},
        	commentstyle=\color[rgb]{0.133,0.545,0.133},
        	stringstyle=\color[rgb]{0.627,0.126,0.941},
}


\begin{document}



\section*{Ch. 2 Set Theory}

\section{Universal Set}

\subsection{Definition}

\begin{center}
$
\Omega \equiv
\Omega \subseteq  S_i , \forall i \hspace{5mm} (Definition \hspace{1mm}  1.1)
$
\end{center}

\subsection{Proof of Existence Universal Set $\Omega$}

\subsection{Properties of Universal Set $\Omega$}
Proofs in Appendix\\ \\
$
2. \vspace{3mm} \hspace{3mm} \Omega \cap \Omega = \Omega \hspace{5mm}( Theorem \hspace{1mm} 1.1) \\
3. \vspace{3mm} \hspace{3mm} \Omega \cup \Omega = \Omega \hspace{5mm}( Theorem \hspace{1mm}  1.2) \\
4.  \vspace{3mm} \hspace{3mm} \Omega \cap  S_i = S_i, \forall i \hspace{5mm} ( Theorem \hspace{1mm}  1.3) \\
5. \vspace{3mm} \hspace{3mm} \Omega \cup S_i = \Omega, \forall i \hspace{5mm} ( Theorem \hspace{1mm} 1.4)  \\
$

\section{Empty Set}
\subsection{Definition}

The "empty set" $\emptyset$ is defined as the complement of the Universal Set $\Omega$
\begin{center}
$
\emptyset \equiv \Omega_\perp
$
\end{center}

\subsection{Proof of Existence Empty Set}

\subsection{Properties of Empty Set}
$
1. \vspace{3mm} \hspace{3mm} \Omega \subset \emptyset \hspace{5mm}( Theorem \hspace{1mm}  2.1) \\
2.  \vspace{3mm} \hspace{3mm}\emptyset \not\subseteq \Omega \hspace{5mm}  ( Theorem \hspace{1mm} 2.2) \\
3. \vspace{3mm} \hspace{3mm} \emptyset \cup \emptyset = \emptyset \hspace{5mm}  ( Theorem \hspace{1mm} 2.3) \\
4. \vspace{3mm} \hspace{3mm} \emptyset \cap \emptyset = \emptyset \hspace{5mm}( Theorem \hspace{1mm} 2.4) \\
5.  \vspace{3mm} \hspace{3mm} \emptyset \cap  S_i = 0, \forall i \hspace{5mm} ( Theorem \hspace{1mm}  2.5) \\
6. \vspace{3mm} \hspace{3mm} \emptyset \cup S_i = S_i, \forall i \hspace{5mm} ( Theorem \hspace{1mm} 2.6)\\
7. \vspace{3mm} \hspace{3mm} \Omega \cup \emptyset = \Omega \hspace{5mm} ( Theorem \hspace{1mm}  2.7) \\
8. \vspace{3mm} \hspace{3mm} \Omega \cap \emptyset = \emptyset \hspace{5mm} ( Theorem \hspace{1mm} 2.8) \\
$

\section{Appendix}
\subsection{Proofs}
$
1. \vspace{3mm} \hspace{3mm} \Omega \subset \emptyset \\
2. \vspace{3mm} \hspace{3mm} \Omega \cap \Omega = \Omega \\
3. \vspace{3mm} \hspace{3mm} \Omega \cup \Omega = \Omega \\
4. \vspace{3mm} \hspace{3mm} \Omega \cup \emptyset = \Omega \\
5. \vspace{3mm} \hspace{3mm} \Omega \cap \emptyset = \emptyset \\
6. \vspace{3mm} \hspace{3mm} \Omega \cap  S_i = S_i, \forall i\\
7. \vspace{3mm} \hspace{3mm} \Omega \cup S_i = \Omega, \forall i \\
8. \vspace{3mm} \hspace{3mm} \emptyset \not\subseteq \Omega \\
9. \vspace{3mm} \hspace{3mm} \emptyset \cup \emptyset = \emptyset \\
10. \vspace{3mm} \hspace{3mm} \emptyset \cap \emptyset = \emptyset \\
11. \vspace{3mm} \hspace{3mm} \emptyset \cap  S_i = 0, \forall i \\
12. \vspace{3mm} \hspace{3mm} \emptyset \cup S_i = S_i, \forall i \\
$


\end{document}