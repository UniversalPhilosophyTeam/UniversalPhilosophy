\documentclass[11pt]{article}
\usepackage{amsfonts}
\usepackage[T1]{fontenc}
\usepackage{mathabx,graphicx}


\newcommand{\test}{\circlearrowright}
\def \loop {\ensuremath{\rotatebox[origin=c]{-90}{$\circlearrowright$}}}
\def \nestedloop {\ensuremath{\rotatebox[origin=c]{-90}{$\circlearrowright$}}^n}



\begin{document}
\section*{Logic}



\section{Law of Non-Contradiction}
Let $a$ denote a boolean statement
\begin{center}
$a$ = $a$
\end{center}
\subsection{Commentary}
In digital logic, the Law of Non-Contradiction is often referred to as a buffer




\section{Not $\lnot$}
\subsection{Definition}
\begin{center}
$\lnot a$ = $\lnot \lbrack a \rbrack$ :=\\
\vspace{2mm}
$\lnot \lbrack \mathbb{T} \rbrack \rightarrow \mathbb{F}$\\
\vspace{2mm}
$\lnot \lbrack \mathbb{F} \rbrack \rightarrow \mathbb{T}$
\end{center}





\section{Contradiction}
\begin{center}
$a$ = $\lnot a \Rightarrow$ contradiction
\end{center}
\subsection{Proof by Contradiction}




\newpage
\section{Logical Function}
\subsection{Definition}
Define logical function with boolean input(s) $a_1,...,a_n$ and a boolean output $b$
\begin{center}
$f_{logical}$ $:=$\\
$f[a_1,...,a_n] \rightarrow b$
\end{center}




\section{Non-Trivial Logical Functions}
\subsection{Definition}
\subsection{Proof All Logical Functions Can Be Built from Not with Any Non-Trivial Logical Function}



\newpage
\section{Logical And $\land$}
\subsection{Definition}
\begin{center}
$a \land b$ = $\land \lbrack a,b \rbrack$ :=\\
$\lnot (\exists \mathbb{F} \in \{a,b\})$ =\\
$\not \exists \mathbb{F} \in \{a,b\}$
\end{center}
\subsection{Alternate Definition}
\begin{center}
$\land \lbrack \mathbb{F},\mathbb{F} \rbrack = \mathbb{F}$\\
$\land \lbrack \mathbb{F},\mathbb{T} \rbrack = \mathbb{F}$\\
$\land \lbrack \mathbb{T},\mathbb{F} \rbrack = \mathbb{F}$\\
$\land \lbrack \mathbb{T},\mathbb{T} \rbrack = \mathbb{T}$
\end{center}


\section{Logical Or $\lor$}
\subsection{Definition}
\begin{center}
$a \land b$ = $\land \lbrack a,b \rbrack$ :=\\
$\exists \mathbb{T} \in \{a,b\})$
\end{center}
\subsection{Alternate Definition}
\begin{center}
$\land \lbrack \mathbb{F},\mathbb{F} \rbrack = \mathbb{F}$\\
$\land \lbrack \mathbb{F},\mathbb{T} \rbrack = \mathbb{T}$\\
$\land \lbrack \mathbb{T},\mathbb{F} \rbrack = \mathbb{T}$\\
$\land \lbrack \mathbb{T},\mathbb{T} \rbrack = \mathbb{T}$
\end{center}



\newpage
\section{Exclusive Or (Xor)}
\subsection{Definition}
\subsection{Alternate Definition}


\section{Not Or (Nor)}
\subsection{Definition}
\subsection{Alternate Definition}





\newpage
\section{Exclusive Nor (Xnor)}
\subsection{Definition}
\subsection{Alternate Definition}

\section{Not And (Nand)}
\subsection{Definition}
\subsection{Alternate Definition}




\newpage
\section*{Appendix}
\section{Criticism logical union, set union, logical and, set and}
- logical or is a function logical and is a function\\
- language muks up our understanding\\

Logical or $\lor$ is different from $\cup$
Logical and $\land$ is different from $\cap$

Logical or, only one has to be true\\
Logical and, both have to be true --> I'll take the intersection\\

Set and, I'll take bag 1 and bag 2 i'll take both --> I'll take the union\\
set or, I'll take bag 1 or bag 2 I'll take just one\\

Do we ever confuse set union, set and with logical or, and?\\
(Don't we describe set union $\cup$ as "or")






\end{document}