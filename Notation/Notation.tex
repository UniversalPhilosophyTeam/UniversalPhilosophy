\documentclass[11pt]{article}
\usepackage{amsfonts}
\usepackage[T1]{fontenc}
\usepackage{mathabx,graphicx}


\newcommand{\test}{\circlearrowright}
\def \loop {\ensuremath{\rotatebox[origin=c]{-90}{$\circlearrowright$}}}
\def \nestedloop {\ensuremath{\rotatebox[origin=c]{-90}{$\circlearrowright$}}^n}



\begin{document}

\section*{Notation}



\section*{Logical Symbols}
$\mathbb{T}$ is read as "True"\\
$\mathbb{F}$ is read as "False"\\
$\lnot$ is read as "not"\\
$\land$ is read as "logical and"\\
== is read as "is equal to"\\
: is read as "satisfying the condition"\\
$\cup$ is read as "union" (sometimes read as "and")\\
$\cap$ is read as "intersection"



\section*{Set Theory Symbols}
$\in$ is read as "in"\\
$\exists$ is read as "there exists"\\
$\not \exists$ is read as "there does not exist"\\
$\forall$ is read as "for all"\\



\section*{Computation Symbols}
$\leftarrow$ is read as "assignment"\\





\section*{Function Symbols}
$x_i$ denotes "inputs"\\
$y_i$ denotes "outputs"\\
$f[x_1,x_2,...x_n] \rightarrow y_1, y_2,...y_n$ \hspace{1.5mm} is read as "function f with inputs $x_1,x_2,...x_n$ outputs $y_1,y_2,...y_n$"\\










\section*{Mathematical Symbols}
= is read as "equals"\\
+ is read as "plus"\\
$\perp$ is read as "orthogonal"



\end{document}