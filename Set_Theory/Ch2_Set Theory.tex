\documentclass[11pt]{article}
\usepackage{amsfonts}
\usepackage[T1]{fontenc}
\usepackage{mathabx,graphicx}


\newcommand{\test}{\circlearrowright}
\def \loop {\ensuremath{\rotatebox[origin=c]{-90}{$\circlearrowright$}}}
\def \nestedloop {\ensuremath{\rotatebox[origin=c]{-90}{$\circlearrowright$}}^n}



\begin{document}


Comma commonly denotes or and and
Defintion of a Set presupposes 1,2,...,n-1,n which is the definition of a set
Must define the relationship between elements and sets first

\section*{Ch. 2 Set Theory}





\section{Element}
Define element $l$
\begin{center}
$
l
$
\end{center}







\section{Identity}
Define identity of element $l$\\

\begin{center}
$
|l| := 1
$
\end{center}



\section{Set}
\subsection{Definition}
Define set $S$ as an ordered union of elements $s_i$

\begin{center}
$
S :=   s_1 \cup  s_2 \cup\ ... \cup s_{n-1} \cup\ s_n = \{s_1,s_2,..s_n\}
$
\end{center}
\subsection{Alternate Notation}
\begin{center}
$
S := s_i \in  S :  i=1,2,...,n-1,n \hspace{5mm}$
\\$
S = \{s_1,s_2,...,s_{n-1},s_n\}
$

\end{center}





\subsection{Magnitude of a Set}
\begin{center}
$
|S| = |\hspace{.5mm} \{x_1,...,x_N\}| = N
$
\end{center}




\subsection{Definition Unordered Set}
Set $S$ is unordered if
\begin{center}
$
S = \{x_1,x_2,..x_n\}
$
\\
$
S = S_1 = S_2 = S_N = \{x_{i_1},x_{i_2},...,x_{i_n}\} \hspace{3mm} 
$
\\
$
\forall i_1,i_2,...,i_n,
$
\\
$
\Longleftrightarrow x_i,x_j \in S, x_i = x_j, \hspace{3mm} \forall i,j \hspace{4mm} i\neq j\hspace{5mm} (Theorem)
$
\end{center}




\subsection{Definition of Unique Set}
\begin{center}
$
a_i, a_j \in S
$
\\
$
a_i \neq a_j \hspace{3mm} \forall i,j \neq i
$
\end{center}



%\subsection{Definition of Non-Unique Set}
%\begin{center}
%$
%a_i, a_j \in S
%$
%\\
%$
%\exists j, j\neq i \hspace{3mm} \ni  a_i = a_j \hspace{3mm}
%$
%\end{center}






\subsection{Definition of Countable/Uncountable set}
Potentially just a line?


\section{Topology; Elements to Sets}
Every element is a set, but not all sets are elements

\section{Comma ,}
The comma symbol '','' is an overloaded symbol
\subsection{Comma in Set Theory}
\subsection{Comma in Counting}

Definition of "," = or   and $\&$ = and
Comma might be a union (spoken as "and" but logicall represents or)



\section{Universal Set}

\subsection{Definition}
Define Universe, the "Universal Set" containing all elements
\begin{center}
$
\Omega := s_i \in \Omega, \forall i \hspace{5mm}
$
\end{center}

%\subsection{Proof of Existence Universal Set $\Omega$}









\section{Empty Set}
\subsection{Definition}
Define Empty Set, the set containing no elements
\begin{center}
$
\emptyset \equiv \{\}
$
\end{center}





\section{Definition Counting}
1,2,3,4,5,...,N



\section{Define a line $\mathbb{L}$}
Define line $\mathbb{L}$
\begin{center}
$
\mathbb{L} = \{l_0,l_{1},l_{2},...,l_{N-1},l_N,l_{N+1},...
$
\end{center}


\section{Definition of Span}
A function mapping to every element of L?





\section{Containment}
\subsection{Contains}
\subsection{Equals = }
Define set equivalence  =
\begin{center}
$
S_1 \subseteq S_2 ; \hspace{3mm} S_2 \subseteq S_1 \hspace{1mm} \Longleftrightarrow S_1 = S_2 \hspace{3mm}
$
\end{center}


\subsection{Subset}



\subsection{Proper Subset Citation}




\subsection{Definition of Complement}
\begin{center}
$
S = \{ s_1,s_2,...,s_N \}
$
\\
$
S^C :=
$
\\
$
s_j: \{ s_j \in \Omega\} \cap \{s_j \not \in S\}; \hspace{2mm} \forall j
$
\end{center}
\subsection{Alternate Notation}
Wikipedia definition of complement
\begin{center}
$
S^C = U - S = \{x \in \Omega : x \not \in S\} \hspace{3mm} 
$
https://en.wikipedia.org/wiki/Complement\_(set\_theory)
\end{center}



\section*{Set Operators}
\section{$\leftarrow$ "Assignment"}
\subsection{Definition}

\section{Insertion}

\section{Append}

\section{"Deletion"}
\subsection{Definition}


\section{Iteration \loop}
\subsection{Definition}
Define iteration \loop

\section{Definition of No-op}
Define ";" , the no-op (The C representation of a null statement or no-op)
\subsection{Properties of No-op}
1. Can be added to any solution $S_i$ and remain a solution for all i anywhere in the order for all j











\section{Appendix}


\subsection{Proofs and Properties}
$
1. \vspace{3mm} \hspace{3mm} \Omega \subset \emptyset \\
2. \vspace{3mm} \hspace{3mm} \Omega \cap \Omega = \Omega \\
3. \vspace{3mm} \hspace{3mm} \Omega \cup \Omega = \Omega \\
4. \vspace{3mm} \hspace{3mm} \Omega \cup \emptyset = \Omega \\
5. \vspace{3mm} \hspace{3mm} \Omega \cap \emptyset = \emptyset \\
6. \vspace{3mm} \hspace{3mm} \Omega \cap  S = S \\
7. \vspace{3mm} \hspace{3mm} \Omega \cup S = \Omega \\
8. \vspace{3mm} \hspace{3mm} \emptyset \not\subseteq \Omega \\
9. \vspace{3mm} \hspace{3mm} \emptyset \cup \emptyset = \emptyset \\
10. \vspace{3mm} \hspace{3mm} \emptyset \cap \emptyset = \emptyset \\
11. \vspace{3mm} \hspace{3mm} \emptyset \cap  S = \emptyset \\
12. \vspace{3mm} \hspace{3mm} \emptyset \cup S = S\\
13. \vspace{3mm} \hspace{3mm} \emptyset = \emptyset\\
14. \vspace{3mm} \hspace{3mm}\emptyset = \Omega^C\\
15. \vspace{3mm} \hspace{3mm} \Omega \subseteq  S
$





\end{document}